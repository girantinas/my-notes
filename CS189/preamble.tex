\documentclass[dvipsnames]{article}

\usepackage{amsmath}
\usepackage{amssymb}

\usepackage[dvipsnames]{xcolor}
\usepackage{anyfontsize}
\usepackage{newtxtext}
\usepackage{newtxmath}
\usepackage{mathrsfs}
\usepackage[utf8]{inputenc} 
\usepackage{bbm}

\usepackage{array}
\usepackage{tabstackengine}
\usepackage{float}

\usepackage{algorithm}
\usepackage[noend]{algpseudocode} % Algorithms

\usepackage{fancyhdr}
\usepackage{enumitem}
\usepackage{icomma}
\usepackage[standard, thmmarks]{ntheorem}
\usepackage[most, listings]{tcolorbox}
\usepackage{ragged2e}
\usepackage{hyperref}

\usepackage{tikz}
\usetikzlibrary{shapes, arrows, positioning, graphs}
\usepackage{caption}
\usepackage{tikz-qtree}
\usepackage{tikz-cd}
\graphicspath{ {./images/} }
\usepackage{graphicx}
\usepackage{rotating}
\usepackage{lipsum}
\usepackage{epigraph}
\usepackage{xpatch}

\usepackage{listings}
\usepackage{physics}
\usepackage{siunitx}
\usepackage{mathtools}

\def\coursenumber{CS 189}
\def\class{Introduction to Machine Learning}
\def\semester{Spring 2022}
\def\name{Rohit Agarwal}
\def\title{Notes}

\textheight=9in
\textwidth=6.5in
\addtolength{\voffset}{-1in} 
\oddsidemargin=0in
\parindent=0pt
\parskip=5pt
\itemsep=-1pt
\floatsep 9pt plus 2pt minus 3pt
\intextsep 9pt plus 2pt minus 3pt
\textfloatsep=9pt plus 2pt minus 3pt

%\makeatletter
%\def\section{\@startsection{section}{1}{\z@}{-3.0ex plus -.6ex minus -.2ex}{1.3ex plus .1ex}{\dunha}}
%\def\subsection{\@startsection{subsection}{2}{\z@}{-1.6ex plus -.5ex minus -.1ex}{0.9ex plus .1ex}{\dunha}}
%\def\subsubsection{\@startsection{subsubsection}{3}{\z@}{-1.0ex plus -.4ex minus -.1ex}{0.5ex plus .1ex}{\dunha}}
%\makeatother

% Page heading
\pagestyle{fancy}
\fancyhf{}
\setlength{\headheight}{30pt}
\lhead{\coursenumber, \semester}
\chead{\title}
\rhead{\name}
\lfoot{\class}
\rfoot{\thepage}

% Table of contents hyperlinks
\hypersetup{
    colorlinks,
    citecolor=black,
    filecolor=black,
    linkcolor=black,
    urlcolor=black
}

% New command to make matrices using the tabstackengine
\stackMath
\setstackEOL{;}
\setstackTAB{,}
\setstacktabbedgap{1ex}
\setstackgap{L}{1.25\normalbaselineskip}
\fixTABwidth{T}
\let\nmatrix\bracketMatrixstack

% Match default array spacing with \nmatrix
\renewcommand{\arraystretch}{1.25}

% Modify matrix environments to accept augments
\makeatletter
\renewcommand*\env@matrix[1][*\c@MaxMatrixCols c]{%
    \hskip -\arraycolsep
    \let\@ifnextchar\new@ifnextchar
\array{#1}}
\makeatother

% Make an environment for theorems and definitions
\theorembodyfont{\normalfont}
\theoremstyle{break} 
\renewtheorem{definition}{Definition}[section]
\renewtheorem{theorem}{Theorem}[section]
\renewtheorem*{theorem*}{Theorem}
\renewtheorem{example}{Example}[section]
\renewtheorem{proof}{Proof}[section]
\newtheorem{algothm}{Algorithm}[section]
\renewtheorem*{proof*}{Proof}
\newtheorem{note}{Note}[section]
\tcolorboxenvironment{definition}{
    enhanced, 
    boxrule=0pt, 
    frame hidden, 
    breakable, 
    sharp corners, 
    colback=blue!10, 
    before skip=10pt, 
    after skip=10pt
}
\tcolorboxenvironment{theorem}{
    enhanced, 
    boxrule=0pt, 
    frame hidden, 
    breakable, 
    sharp corners, 
    colback=orange!10, 
    before skip=10pt, 
    after skip=10pt
}
\tcolorboxenvironment{theorem*}{
    enhanced, 
    boxrule=0pt, 
    frame hidden, 
    breakable, 
    sharp corners, 
    colback=orange!10, 
    before skip=10pt, 
    after skip=10pt
}
\tcolorboxenvironment{algothm}{
    enhanced, 
    boxrule=0pt, 
    frame hidden, 
    breakable, 
    sharp corners, 
    colback=purple!10, 
    before skip=10pt, 
    after skip=10pt
}
\tcolorboxenvironment{proof}{
    enhanced, 
    boxrule=0pt, 
    frame hidden, 
    breakable, 
    sharp corners, 
    colback=orange!10, 
    before skip=10pt, 
    after skip=10pt
}
\tcolorboxenvironment{example}{
    enhanced, 
    boxrule=0pt, 
    frame hidden, 
    breakable, 
    sharp corners, 
    colback=green!10, 
    before skip=10pt, 
    after skip=10pt
}
\tcolorboxenvironment{note}{
    enhanced, 
    boxrule=0pt, 
    frame hidden, 
    breakable, 
    sharp corners, 
    colback=purple!10, 
    before skip=10pt, 
    after skip=10pt
}

% Fix the mathcal alphabet
\DeclareMathAlphabet{\mathcal}{OMS}{cmsy}{m}{n}

% Set default enumerate labels
\setlist[enumerate, 1]{label*=\arabic*.}
\setlist[enumerate, 2]{label=(\roman*)}


%%% Linear Algebra
% Operators
\DeclareMathOperator{\proj}{proj}  % Projection
\DeclareMathOperator{\Span}{span}  % Vector/Matrix Span
\DeclareMathOperator{\range}{range}  % Range (span of columns)
\DeclareMathOperator{\Null}{null }  % Null space
\DeclareMathOperator{\Cols}{C}  % Col space
\DeclareMathOperator{\corr}{corr}  % correlation
%\DeclareMathOperator{\det}{det}
% Commands
\newcommand{\inner}[1]{\left\langle #1 \right\rangle}  % Inner product
\renewcommand{\va}[1]{\vec{\bm{#1}}}  % Vector with italic bold
\newcommand{\mat}[1]{\left[ #1 \right]}  % Matrix variable
\newcommand{\linearcombination}{a_1v_1 + a_2v_2 + \cdots + a_nv_n}
\newcommand{\listofvectors}{v_1, v_2, \ldots, v_n}
\newcommand{\listofscalars}{a_1, a_2, \ldots, a_n}
\newcommand{\setofvectors}{\set{\listofvectors}}
\newcommand{\listofnames}[2]{#1_1, #1_2, \ldots, #1_{#2}}
\newcommand{\spanof}{\operatorname{span}}
\newcommand{\Lin}[1]{\mathcal{L} \left( #1 \right)}
\newcommand{\Polys}[2]{\mathcal{P}_{#2}(#1)}
\newcommand{\PolysAll}[1]{\mathcal{P}(#1)}
\newcommand{\Matof}[1]{\mathcal{M}(#1)}

%%% Probability Theory
% Operators
\DeclareMathOperator{\iid}{\sim_{IID}}
% Commands
\newcommand{\Bern}[2]{#1 \sim \text{Bernoulli}\left( #2 \right)}
\newcommand{\Bin}[3]{#1 \sim \text{Binomial}\left( #2, #3 \right)}
\newcommand{\Geo}[2]{#1 \sim \text{Geometric}\left( #2 \right)}
\newcommand{\Pois}[2]{#1 \sim \text{Poisson}\left( #2 \right)}
\newcommand{\PoisX}[2]{\dfrac{{#1}^{#2}}{#2 !} e^{- {#1}}}
\newcommand{\Uniform}[3]{#1 \sim \text{Uniform}\left( #2, #3 \right)}
\newcommand{\Expo}[2]{#1 \sim \text{Exp}\left( #2 \right)}
\newcommand{\Normal}[3]{#1 \sim \mathcal{N}\left( #2, #3 \right)}
\newcommand{\StdNormal}[1]{#1 \sim \mathcal{N}\left( 0, 1 \right)}
\newcommand{\PP}[1]{\mathrm{PP}\left( #1 \right)}
\renewcommand{\Pr}[1]{\mathbb{P}\left[ #1 \right]}
\renewcommand{\P}[1]{\text{P}\left( #1 \right)}
\newcommand{\E}[1]{\mathbb{E}\left[ #1 \right]}
\newcommand{\Var}[1]{\text{Var}\left( #1 \right)}
\newcommand{\Cov}[1]{\text{Cov}\left( #1 \right)}
\renewcommand{\L}[1]{\mathbb{L}\left[ #1 \right]}

%%% Miscellaneous
% Operators
\DeclareMathOperator{\R}{\mathbb{R}}
\DeclareMathOperator{\Z}{\mathbb{Z}}
\DeclareMathOperator{\N}{\mathbb{N}}
\DeclareMathOperator{\C}{\mathbb{C}}
\DeclareMathOperator{\F}{\mathbb{F}}
\let\O\relax
\DeclareMathOperator{\O}{\mathcal{O}}
\DeclareMathOperator*{\argmin}{argmin}
\DeclareMathOperator*{\argmax}{argmax}
% Commands
\newcommand{\set}[1]{{\left\{#1\right\}}} % braces for set notation
\newcommand{\powerset}[1]{\mathscr{P}\left( #1 \right)} % Power Set
\newcommand{\floor}[1]{\left\lfloor #1 \right\rfloor}  % Floor
\newcommand{\ceil}[1]{\left\lceil #1 \right\rceil}  % Ceiling
\newcommand{\conj}[1]{\bar{#1}}
\newcommand{\blue}{\color{blue}}
\newcommand{\black}{\color{black}}
\newcommand{\Question}[1]{\color{black}\section{ #1 }\color{blue}}

%%% Title Page
\renewcommand\epigraphflush{flushright}
\renewcommand\epigraphsize{\Large}
%\setlength\epigraphwidth{0.6\textwidth}

\definecolor{titlepagecolor}{cmyk}{1,.60,0,.40}

\DeclareFixedFont{\titlefont}{T1}{ppl}{b}{it}{0.5in}
\DeclareFixedFont{\subtitlefont}{T1}{ppl}{b}{it}{0.3in}


% *** quiver ***
% A package for drawing commutative diagrams exported from https://q.uiver.app.
%
% This package is currently a wrapper around the `tikz-cd` package, importing necessary TikZ
% libraries, and defining a new TikZ style for curves of a fixed height.
%
% Version: 1.2.0
% Authors:
% - varkor (https://github.com/varkor)
% - AndréC (https://tex.stackexchange.com/users/138900/andr%C3%A9c)
\NeedsTeXFormat{LaTeX2e}
% `calc` is necessary to draw curved arrows.
\usetikzlibrary{calc}
% `pathmorphing` is necessary to draw squiggly arrows.
\usetikzlibrary{decorations.pathmorphing}

% A TikZ style for curved arrows of a fixed height, due to AndréC.
\tikzset{curve/.style={settings={#1},to path={(\tikztostart)
    .. controls ($(\tikztostart)!\pv{pos}!(\tikztotarget)!\pv{height}!270:(\tikztotarget)$)
    and ($(\tikztostart)!1-\pv{pos}!(\tikztotarget)!\pv{height}!270:(\tikztotarget)$)
    .. (\tikztotarget)\tikztonodes}},
    settings/.code={\tikzset{quiver/.cd,#1}
        \def\pv##1{\pgfkeysvalueof{/tikz/quiver/##1}}},
    quiver/.cd,pos/.initial=0.35,height/.initial=0}

% TikZ arrowhead/tail styles.
\tikzset{tail reversed/.code={\pgfsetarrowsstart{tikzcd to}}}
\tikzset{2tail/.code={\pgfsetarrowsstart{Implies[reversed]}}}
\tikzset{2tail reversed/.code={\pgfsetarrowsstart{Implies}}}
% TikZ arrow styles.
\tikzset{no body/.style={/tikz/dash pattern=on 0 off 1mm}}