% !TEX root = lectures.tex

\section{Lecture 6}
Recall the definition of a measure space. The $\mu(\emptyset) = 0$, then the only thing
this excludes is $\mu$ being $\infty$ for every set $A \in \mca$. Let's talk about a few examples of $\sigma$-algebras.
\begin{example}
    Let $X$ be a set which we will detail below. Then the following $\mca$ are $\sigma$-algebras under $X$:
    \begin{enumerate}
        \item $\mca = \mathcal{P}(X)$.
        \item $\mca = \{ A \subset X: \text{either $A$ is countable, or $A^c$ is countable.} \}$
        \item $X = (0, 1]$. Define a dyadic interval of rank $n$ as $(\frac{i}{2^n}, \frac{i + 1}{2^n}]$, where $0 \le i \le 2^n - 1$
        and let $D_n$ be the set of such intervals. $\mca = $ the arbitrary union of elements of $D_n$ (this is an ordinary algebra).
    \end{enumerate}
\end{example}

\begin{lemma}
    Let $\{A_{\alpha}: \alpha \in I\}$ be a collection of $\sigma$-algebras. Then $\bigcap_{\alpha \in I} A_{\alpha}$ is a $\sigma$-algebra.
    \begin{proof}
        If $A \in \bigcap_{\alpha \in I} A_{\alpha}$, meanis $A \in A_{\alpha}$ for all $\alpha \in I$, which means that $A^c \in A_{\alpha}$ for all $\alpha$,
        meaning that $A^c \in \bigcap_{\alpha \in I} A_{\alpha}$. The other axiom is verified equally trivially.
    \end{proof}
\end{lemma}
Take $\mathcal{C} \subseteq \mathcal{P}(X)$. The above lemma implies that there exists a smallest $\sigma$-algebra that contains $\mathcal{C}$.
We shall define 
\[ \sigma(\mathcal{C}) = \bigcap_{\sigma\text{-algebra} \mca: \mathcal{C} \subseteq \mca} \mca \]
Note that $\sigma(\sigma(\mcc))$.
If $\mcc_1 \subseteq \mcc_2$, then $\sigma(\mcc_1) \subseteq \sigma(\mcc_2)$. The probability point of view
is to call the elements of $\mca$ as events. We can then think about points in $X$ as outcomes; it participates in some events in $\mca$
and makes them true. Likewise, we can restrict a space to one set in $\mca$; then we can treat the new space as a conditional probability.
We ``revealed'' some information.

Suppose $X$ is a topological space (i.e. it has open subsets). Call $\mathcal{G} = \{\text{all open sets}\}$.
Then we call $\mathcal{B}(X) := \sigma(\mathcal{G})$ the Borel $\sigma$-algebra on $X$. A Borel set is a 
\begin{theorem}
Let $X = \R$. The Borel \sa is generated by each of the following collections.
\begin{enumerate}
    \item $e_1 = \{(a, b)\}$
    \item $e_2 = \{ [a, b]\}$
    \item $e_3 = \{ (a, b] \}$
    \item $e_4 = \{(a, \infty)$
\end{enumerate}
\begin{proof}
    It's clearly true that $\sigma(e_1) \subseteq \mathcal{B}(\R)$
    because each of the sets in $e_1$ is open.
    Furthermore, for any open set $O \subseteq \R$ it can be decomposed into a countable number of open intervals
    (to see countability, note that it's clear that the lengths of each of these are positive).
    So $\sigma\{\text{open sets}\} \subseteq \sigma(e_1)$.

    Now, let's show $\sigma(e_1) = \sigma(e_2)$. $[a, b]^c = (-\infty, a) \cup (b, \infty)$. We can decompose $\bigcup_{n \in \N, n > b} (b, n) = (b, \infty)$.
    We can do the same thing with $(-\infty, a)$. By the \sa properties, this means $[a, b] \in \sigma(e_1)$,
    so $\sigma(e_1) \supseteq \sigma(e_2)$. We could've also written $[a, b] = \bigcap_{n \in \N} (a - \frac{1}{n}, b + \frac{1}{n})$.
    Finally $(a, b) = \bigcup_{n = 1}^{\infty} [a + \frac{1}{n}, b - \frac{1}{n}]$.

    $(a, b]$ and $(a, \infty)$ aren't too conceptually different. The only trick is to make $A \setminus B = A \cap B^c$. In fact $(a, \infty)$ where $a \in \Q$ is even good enough,
    since for $a \in \R$, $(a, \infty) = \bigcup_{n = 1}^{\infty} (2^{-n} \floor{2^n a}, \infty)$.
\end{proof}
\end{theorem}
Recall the definition of measure space. $X$ with a \sa $\mca$ is a measurable space; adding a function which acts as a measure $\mu$,
we get a measure space. If a measure space has total measure $\infty$, it's an infinite measure space; if it's $1$, then it's a
probability space. For $\mca = \mathcal{P}(X)$ and $\mu(A) = |A|$ then this is the counting measure.
Consider the following measure, the Dirac measure:
\[ \delta_x(A) = \begin{cases}
    1 & x \in A\\
    0 & x \notin A
\end{cases} \]
Recall that under countable union of disjoint sets,
\[ \mu\qty(\bigcup_{i = 1}^{\infty} A_i) = \sum_{i = 1}^{\infty} \mu(A_i) \]
Note that we cannot hope for an uncountable measure, as $[0, 1] = \bigcup_{x \in [0, 1]} \{x\}$,
making $1 = 0 + 0 + \dots + 0$. Uncountable sums don't make much sense!
One can see that this is a measure. Another measure would be $\sum_{i \in I} a_i \delta_{x_i}$.