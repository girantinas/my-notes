% !TEX root = lectures.tex
\section{Lecture 1}
The notion of measure is going to generalize the real line's notion of \textbf{distance}. Recall that $\mathbb{Q}$ can be
constructed from the integers as follows, defining $\mathbb{N}_{+} = \{1, 2, 3, \dots\}$ and $\Z = \{ \dots, -2, -1, 0, 1, 2, \dots \}$
and we can write $\Q = \{ \frac{m}{n} : m \in \Z, n \in \N_{+} \}$. Recall also that $\Q$ is a countable set.

Recall that $\Q$ is a dense subset of the real line, which we will revisit. First, we define the notion of a \textbf{distance} (or \textbf{metric}) between two rational numbers, a function $d: \Q \times \Q \to [0, \infty)$:
\[ d\qty(\frac{m_1}{n_1}, \frac{m_2}{n_2}) = \frac{|m_1 n_2 - n_1 m_2|}{n_1 n_2} \]
the distance is also a rational number. However, not all Cauchy sequences in the rationals \textbf{converge} to a rational number, the metric space is not complete.
\begin{definition}
    A \textbf{Cauchy sequence} is a sequence $\{x_n\}$ such that for all $\epsilon > 0$, there exists a threshold $n_0 \in \N$ such that 
    if we have $n, m \geq n_0$, $d(x_n, x_m) < \epsilon$.
\end{definition}
We can construct the reals by filling in absences in the rationals. To see these holes, we will represent real numbers with their decimal representation. Now, everyone has a unique representation, except, $0.999\dots = 1.000\dots$.
There are only countably many ``awkward'' points here (terminating decimals are a subset of the rationals), so it's not a big issue. We
will just ban the $9$s version, i.e. $1/2 = 0.5000$. If I select values for the decimal places at random, then with probability 0 I get a repeating decimal (rational number). This means
the rationals are very slim among the reals.

Let's take $\pi$ and write it as a Cauchy sequence of rationals $(3, 3.1, 3.14, 3.141, \dots)$. Since $\pi$ is not rational,
we have that the Cauchy sequence doesn't converge to a rational number. This seems to be a way to construct real numbers; why don't we identify $\pi$ with this Cauchy sequence?
But this isn't the only Cauchy sequence that converges to $\pi$. Also from infinite series we know that $\frac{\pi}{4} = 1 - \frac{1}{3} + \frac{1}{5} - \frac{1}{7} + \dots$;
multiplying by 4 and taking partial sums forms another Cauchy sequences. However, defining the relation between two convergent Cauchy sequences (in the real numbers) $x$ and $y$
that $x \sim y$ if $x$ and $y$ have the same limit. One can check this is an equivalence relation. This implies:

\begin{definition}
    The set of real numbers, $\R$ is the collection of equivalence classes of Cauchy sequences of rationals where for two sequences $x, y$, $x \sim y$ if $d(x_n, y_n) \to 0$.
\end{definition}
Thus, $\R$ is the \textbf{completion} of the rational numbers $\Q$. Let's look at its properties
\begin{enumerate}
    \item The distance function is as follows. Take $\qty(x_n) \in X$, $\qty(y_n) \in Y$
    \begin{align*}
        d&: \R \times \R \to [0, \infty) \\
        d(X, Y) &= \lim_n d(x_n, y_n)
    \end{align*}
    This is also the distance function for any completion of a metric space (where on the right, the distance function is inherited from the original space).
    It turns out it doesn't matter what representation we use (we are modding out by everything in the equivalence classes).
\end{enumerate}

