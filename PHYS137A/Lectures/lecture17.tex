\section{Lecture 17: Operator Properties}

Recall that $\ket{\Psi}$ is a state vector and $\bra{\Psi}$ is its adjoint.
Furthermore, eecall the adjoint of an operator, where if
\[ \ket{\phi} = \hat{A} \ket{x} \]
then
\[ \bra{\phi} = \bra{x} \hat{A}^{\dagger} \]

\begin{note}[Adjoint/Inner product properties]
    Here are some properties of the inner product.
    \[ \bra{\Psi_1}\ket{\Psi_2}  = \int \Psi_1^*(\mbf{r}) \Psi_2(\mbf{r}) \]
    \[ \bra{\Psi_1}\ket{\Psi_2}^* = \bra{\Psi_2}\ket{\Psi_1}\]
    \[ \bra{\Psi_1}\ket{c \Psi_2} = c \bra{\Psi_1}\ket{\Psi_2} \]
    \[ \bra{c \Psi_1}\ket{\Psi_2} = c^* \bra{\Psi_1}\ket{\Psi_2} \]
    For orthonormal $\Psi_1, \Psi_2$,
    \[ \bra{\Psi_1}\ket{\Psi_2} = \delta_{12} \]
    Also, taking the adjoint of a scalar function is just:
    \[ f(\hat{A})^{\dagger} = f^*(\hat{A}^{\dagger}) \]
\end{note}

\begin{definition}[Inverses/Identity]
    The identity operator is defined as
    \[ \hat{I} \ket{\Psi} = \ket{\Psi} \]

    If $\hat{B}\hat{A} = \hat{A}\hat{B} = \hat{I}$, then $\hat{B} = \hat{A}^{-1}$.
\end{definition}
Then we can also define a unitary operator.
\begin{definition}[Unitary Operator]
    A linear operator is said to be unitary if $\hat{U}^{-1} = \hat{U}^{\dagger}$, i.e.
    \[ \hat{U} \hat{U}^{\dagger} = \hat{U}^{\dagger} \hat{U} = \hat{I} \]
\end{definition}
These can be said to be like rotations: they do not change things like commutators, inner products, and all
the physics we have done until this point.
\begin{definition}[Projection Operator]
    The projection operator $\hat{\Lambda}$ does the following If we have $\ket{\Psi} = \ket{\phi} + \ket{x}$,
    then
    \[ \ket{\phi} = \hat{\Lambda} \ket{\Psi}, \ket{x} = \qty(\hat{I} - \hat{\Lambda})\ket{\Psi} \]
\end{definition}
Here is an incredibly useful theorem. It uses the sum of an outer product $\ket{u}\bra{v}$.
\begin{theorem}
    The following is true:
    \[ \sum_n \ket{\Psi_n}\bra{\Psi_n} = \hat{I} \]

    \begin{proof*}
    \begin{align*} 
        \ket{\Psi} &= \sum_n c_n \ket{\Psi_n} \\
        \bra{\Psi_m}\ket{\Psi } &= \sum_n c_n \bra{\Psi_m} \ket{\Psi_n} = c_m \\
        \Psi(\mbf{r}, t) &= \sum_{n} \int \Psi^*_n(\mbf{r'}) \Psi(\mbf{r'}, t) \Psi_n(\mbf{r}) \dd{\mbf{r'}} \\
        \Psi(\mbf{r}, t) &= \int \Psi(\mbf{r'}, t) \qty(\sum_{n} \Psi^*_n(\mbf{r'}) \Psi_n(\mbf{r})) \dd{\mbf{r'}} \\
    \end{align*}
    So we must have $\sum_n  \Psi^*_n(\mbf{r'}) \Psi_n(\mbf{r}) = \delta(\mbf{r} - \mbf{r'})$,
    since it acts like a delta under the integral sign.

    Note that
    \begin{align*}
        \bra{x}\ket{\Psi} &= \int x^*(\mbf{r}, t) \Psi(\mbf{r}, t) \dd{\mbf{r}} \\
        \bra{x}\ket{\Psi}&= \int \int x^*(\mbf{r}, t) \Psi(\mbf{r}, t) \delta(\mbf{r} - \mbf{r}') \dd{\mbf{r}} \dd{\mbf{r'}} \\
        \bra{x}\ket{\Psi}&= \sum_n \int x^*(\mbf{r}, t) \Psi_n(\mbf{r}) \dd{\mbf{r}} \int \Psi_n^*(\mbf{r'}) \Psi(\mbf{r'}, t) \dd{\mbf{r'}} \\
        \bra{x}\ket{\Psi} &= \sum_n \bra{x}\ket{\Psi_n}\bra{\Psi_n}\ket{\Psi} \\
        \bra{x}\ket{\Psi} &= \bra{x} \qty(\sum_n \ket{\Psi_n} \bra{\Psi_n})\ket{\Psi}
    \end{align*}
    The only operator that always satisfies this is the identity, so $\sum_n \ket{\Psi_n} \bra{\Psi_n} = \hat{I}$.
\end{proof*} 
\end{theorem}