\section{Lecture 7: Hermitian Operators}

\subsection{Probability Current}
Like we stated before, Hermitian operators have all real eigenvalues, and any operator that is observable is Hermitian! Now continuing from last lecture,
we had (substituting the spatial part and assuming a real valued potential):
\begin{align*}
    \partialderivative{}{t} \int | \Psi |^2 \dd{\mbf{r}} &= \int \frac{1}{i \hbar} \qty[\Psi^* (\hat{H} \Psi) - (\hat{H} \Psi)^* \Psi] \dd{\mbf{r}} \\
    &= \int \frac{i \hbar}{2m} \qty[\Psi^* (\grad^2 \Psi) - (\grad^2 \Psi)^* \Psi] \dd{\mbf{r}} \\
    &= - \int \frac{i \hbar}{2m} \int \mbf{\grad} \cdot [\Psi^* \mbf{\grad} \Psi - (\mbf{\grad} \Psi^*) \Psi] \dd{\mbf{r}} \\
    &= -\int \mbf{\grad} \cdot \mbf{j} \dd{\mbf{r}} \\
\end{align*}
Where we define:
\[ \mbf{j} = \Psi^* \mbf{\grad} \Psi - (\mbf{\grad} \Psi^*) \Psi \]
the "probability current." Using Stokes' theorem:
\begin{align*}
    \partialderivative{}{t} \int | \Psi |^2 \dd{\mbf{r}} &= - \int \mbf{j} \cdot \dd{\mbf{s}} \\
    \partialderivative{P(\mbf{r}, t)}{t} + \mbf{\grad} \cdot \mbf{j}(\mbf{r}, t) &= 0
\end{align*}
So now the name of $\mbf{j}$ makes sense! This quantifies how much the current moves away from a given point.

\subsection{Expectation Values}
In classical mechanics, we have the average as:
\[ \langle \mbf{r} \rangle = \int_{\R^3} \mbf{r} P(\mbf{r}, t) \dd{\mbf{r}} \]
Now suppose we have an operator $\hat{r}$ that gives position as $\hat{r} \Psi = \mbf{r} \Psi$.
\[ \langle \hat{r} \rangle = \int_{\R^3} \mbf{r} |\Psi(\mbf{r}, t)|^2 \dd{\mbf{r}} = \int_{\R^3} \Psi^* \hat{r} \Psi \dd{\mbf{r}} \]
Now let's say we want the average of the momentum:
\begin{align*}
    \langle \hat{p} &= \int \Psi^* \hat{p} \Psi \dd{\mbf{r}} \\
    &= -i\hbar \int \Psi^* \grad \Psi \dd{\mbf{r}} \\
\end{align*} 

Note that $\hat{p}$ is Hermitian but $\hat{p} = - i \hbar \grad$ and so its conjugate gives you a positive sign? That's because we only give the complex conjugate
guarantees on the operator acting on a wave function! We will see more in homework.

If you comes two Hermitian operators, the product is Hermitian only if the two operators commute. However, operators (even Hermitian ones) do not commute in general.
For example, $\hat{x}$ and $\hat{p}_x$ do not commute.

\begin{definition}
    The commutator (bracket) is defined as:
    \[ [\hat{A}, \hat{B}] = \hat{A} \hat{B} - \hat{B} \hat{A} \]
\end{definition}
Clearly the commutator is 0 if and only if the two operators commute. This means they are unrelated and can be measured simultaneously!

\begin{align*}
    [\hat{x}, \hat{p}_x] &= \hat{x} \hat{p}_x - \hat{p}_x \hat{x} \\
    [\hat{x}, \hat{p}_x] \Psi &= \hat{x} \hat{p}_x \Psi - \hat{p}_x \hat{x} \Psi \\
    &= - i \hbar x \partialderivative{}{x} \Psi + i \hbar \partialderivative{}{x} (x \Psi) \\
    &= - i \hbar x \partialderivative{}{x} \Psi + i \hbar \qty(\Psi + x \partialderivative{\Psi}{x}) \\
    &= i \hbar \Psi
\end{align*}
So the commutator is $i \hbar I \neq 0$. In fact, in general, the commutator of a canonical Fourier pair will always be this.