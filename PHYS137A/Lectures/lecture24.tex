\section{Lecture 24: Spin} 

\begin{definition}[Spin]
    Elementary particles ahve an internal degree of freedom that behaves as an angular momentum and is called spin.
    The operator that gives you the spin of a state is $\hat{\mbf{S}}$.
\end{definition}

Since spin is angular momentum, we have:
\begin{align*}
    [\hat{S}_x, \hat{S}_y] &= i \hbar \hat{S}_z \\
    [\hat{S}_y, \hat{S}_z] &= i \hbar \hat{S}_x \\
    [\hat{S}_z, \hat{S}_x] &= i \hbar \hat{S}_y
\end{align*}
Also, the simultaneous eigenbasis of $\hat{S}^2, \hat{S}_z$ called $\ket{s m_s} = \chi_{s, m_{s}}$ acts
\begin{align*}
    \hat{S}^2 \ket{s m_s} &= s(s + 1)\hbar^2 \ket{s m_s} \\
    \hat{S}_x \ket{s m_s} &= m_s\hbar \ket{s m_s} \\
\end{align*}
$s$ can be $0, 1/2, 1, 3/2, \dots$. $m_s$ has $(2s + 1)$ allowed values: $(-s, -s + 1, \dots, s)$. Recall the matrix
representations of these operators. Let's take $s = 1$.
\begin{align*}
    \hat{S}_z &= \hbar \mqty(\dmat{1, 0, -1}) \\
    \hat{S}^2 &= 2 \hbar^2 \mqty(\dmat{1, 1, 1}) \\
    \hat{S}_{\pm} &= \hat{S}_x \pm i \hat{S}_y
\end{align*}
The eigenvectors in this basis are just the elementary basis vectors over $\R^3$. The eigenvalues of $\hat{S}_z$ are $\hbar, 0, -\hbar$ respectively.
All the eigenvalues of $\hat{S}^2$ are $2\hbar^2$.

\subsection{Wave Function with Spin}
Let's add another argument to the wavefunction $\Psi(\mbf{r}, t, \sigma)$. We can write a general state as:
\begin{align*}
    \sum_{m_s = -s}^{s} \Psi_{m_s}(\mbf{r}, t) \chi_{s, m_{s}}
\end{align*}
Then naturally, $|\Psi_{m_s}(\mbf{r}, t)|^2 \dd{\mbf{r}}$ is the probability of being in a ball of volume $\dd{\mbf{r}}$ around $\mbf{r}$
at time $t$ at spin state $S_z = m_s \hbar$. Then
\[ \sum_{m_s = -s}^s |\Psi_{m_s}(\mbf{r}, t)|^2 \dd{\mbf{r}} \neq 1 \]
gives you the probability irrespective of spin (i.e. for all spins)
\[ \int |\Psi_{m_s}(\mbf{r}, t)|^2 \dd{\mbf{r}} \neq 1 \]
gives you the probability is a particle with $m_s \hbar$ at time $t$.

In our above discussion, we assumed that position and spin $s$ are independent. This may not necessarily be the case (as we will see in entanglement).

\subsection{Spin = 1/2}
Some examples of systems with spin 1/2 are: electrons, protons, qubits. We take $s = \frac12$, $m_s = -\frac{1}{2}, +\frac{1}{2}$.
Note that the length of the vector is $\sqrt{\text{e-val of }\hat{S}^2}$. The eigenvalue $s(s + 1) \hbar^2 = \frac{3}{4} \hbar^2$,
so the length is $\frac{\sqrt{3}}{2}$. The $z$ projection can be $\frac{\hbar}{2}$ or $\frac{-\hbar}{2}$.

We can write this $\chi_{1/2, 1/2}, \chi_{1/2, -1/2}$ or $\ket{\uparrow}, \ket{\downarrow}$ or $\ket{+}, \ket{-}$. In
a sense, this algebra describes anything that can take two takes perfectly!
\begin{align*}
    S^2 \ket{+} &= \frac{3\hbar^2}{4} \ket{+} \\
    S^2 \ket{-} &= \frac{3\hbar^2}{4} \ket{-} \\
    S_z \ket{+} &= \frac{\hbar}{2} \ket{+} \\
    S_z \ket{-} &= -\frac{\hbar}{2} \ket{-} \\
    S_{\pm} \ket{\mp} &= \hbar \ket{\pm} \\
    S_{\pm} \ket{\pm} &= 0 \\
    S_x \ket{+} &= \frac{\hbar}{2} \ket{-} \\
    S_y \ket{-} &= \frac{-i\hbar}{2} \ket{+}
\end{align*}
Let's write the matrices:
\begin{align*}
    [S_z] &= \frac{\hbar}{2} \smqty(\pmat{3}) \\
    [S_x] &= \frac{\hbar}{2} \smqty(\pmat{1}) \\
    [S_y] &= \frac{\hbar}{2} \smqty(\pmat{2})
\end{align*}
If you have some state:
\[ \ket{\chi} = a \ket{+} + b \ket{-} \]
The probability of being spin up $|a|^2$ and being spin down $|b|^2$ where:
\[ |a|^2 + |b|^2 = 1 \]
(this made Patrick very happy). For spin $1/2$:
\begin{align*}
    S^2 &= \frac{3}{4} \hbar^2 I \\
    S_x^2 &= S_y^2 = S_z^2 = \frac{\hbar^2}{4} I \\
    S_{\pm}^2 &= 0
\end{align*}

\begin{definition}[Anticommutator]
    The Anticommutator for two operators $\hat{A}, \hat{B}$ is:
    \begin{align*}
        [\hat{A}, \hat{B}]_{+} &= \hat{A}\hat{B} + \hat{B} \hat{A}
    \end{align*}
\end{definition}

This means:
\[ [S_x, S_y]_{+} = 0 \]

Note that the Pauli matrices (the matrices above without the $\hbar$) combined with the identity froms a basis for all 2 by 2 matrices.