\subsection{Lecture 20}
\subsubsection{Information Theory and Channel Coding}
We now return to our Binary Symmetric Channel from before.

\[\begin{tikzcd}
	X & 0 && 0 & Y \\
	& 1 && 1
	\arrow["{1-p}", from=1-2, to=1-4]
	\arrow["p"'{pos=0.3}, from=2-2, to=1-4]
	\arrow["p"{pos=0.3}, from=1-2, to=2-4]
	\arrow["{1-p}"', from=2-2, to=2-4]
\end{tikzcd}\]

\begin{definition}[Rate of Transmission]
    The rate of information is defined as number of bits conveyed / number of bits transmitted.
\end{definition}

For example for our BSC, we could transmit each bit 3 times and have the channel follow majority rule. This leads to a rate of 
$R = 1/3$.
Note that our information-theoretic rate abstracts out the time (this is not megabits per second, for example).

We define a few notions of information theory in order to build one of the most powerful results.

\begin{definition}
    Given a channel, let $p_e$ be the maximum probability of an error (over all possible codewords).
\end{definition}

\begin{definition}[Capacity]
    The capacity of a channel is defined as the fastest rate you can transfer information such that decoding is asymptotically error-free (error rate approaches 0 as $n \to \infty$).
\end{definition}

\begin{theorem}[Capacity of BSC]
    The capacity of this channel is $C(p) = 1 - H(p)$. where entropy $H(p) = -p \log p - (1 - p) \log (1- p)$.

    \begin{proof*}
        Define $E_i$ to be an indicator for when there is an error in $i$th position.
        The sequence of errors/no-errors looks something like $0101100\dots$ where $\Pr{E_i = 1} = p$.
        We can re-use the source-coding theorem from before! For large $n$ the typical amount of these strings is
        $A = 2^{nH(p)}$.

        Suppose there are $B$ different codewords (input strings) of length $n$ that are separable at output. Call them
        $x_1, \dots, x_B$ and their outputs the random variables $Y_1, \dots, Y_B$. Let $x_i$ be composed of $n$ random bits (think of them as vectors).
        Let $S_1$ be the set of $A$ typical outputs corresponding to $x_1$. Then, the probability of an error:
        \begin{align*}
            p_e &= \Pr{Y_2 \in S_1 \cup Y_3 \in S_1 \cup \dots \cup Y_B \in S_1} \\
            &\leq B \cdot \Pr{Y_2 \in S_1} \\
            &= B A 2^{-n}
        \end{align*}
        where the last probability is true because $Y_2$ has $n$ random bits (since $x_2$ is arbitrarily random), so $\Pr{Y_2 \in S_1} = |S_1| 2^{-n} = A2^{-n}$.
        Let $B = 2^{nR}$. Then,
        \begin{align*}
            p_e &\leq 2^{nR} 2^{nH(p)} 2^{-n}\\
            &\leq 2^{n(R + H(p) - 1)}\\
        \end{align*}
        Thus, if we want the probability of an error to go to 0 as $n \to \infty$, then we need $R + H(p) - 1 < 0$ or $R < 1 - H(p)$.

        We've shown that if $R < C(p)$, then there exists a coding scheme such that $p_e \to 0$ as $n \to \infty$ (it is asymptotically error free).
        You need the converse to complete the proof: i.e. if $R > C(p)$, then
        no coding scheme is asymptotically error-free.
    \end{proof*}
\end{theorem}

Thus we see that our $R = 1/3$ example, there are $B = 2^{n/3}$ different codewords of length $n$ that are separable at output.

In fact, there is a more general theorem for all channels.
\begin{theorem}[Shannon's Channel Coding Theorem]
    If the rate of information transfer is below the capacity of a channel, then there exists a channel coding such that error is nominally 0.
\end{theorem}

Note that neither of these theorems doesn't offer practical coding method for finite $n$ or a required $p_e$.

\subsubsection{Hypothesis Testing}
We formulate the problem of seeing the output of a link and we want to make predictions of its input.
Suppose $X \in \{0, 1\}$ and $\Pr{Y \mid X}$ is known. Let $\hat{X}$ be the estimate of $X$. We wish to solve the following optimization problem.

\begin{itemize}
    \item Maximize Probability of Correct Detection (PCD), $\Pr{\hat{X} = 1 \mid X = 1}$
    \item Subject to a constraint on Probability of False Alarm (PFA), $\Pr{\hat{X} = 1 \mid X = 0} \leq \beta$
\end{itemize}

\begin{definition}[Receiver Operating Characteristic (ROC)]
    This is a graph of PCD for the solution of the above problem as a function $R(\beta)$.
\end{definition}

\begin{theorem}[Neyman-Pearson Theorem]
    To maximize given the constraint, we should predict:
    \[ \hat{X} = \begin{cases}
        1 & \text{if } L(Y) > \lambda \\
        1 & \text{with probability $\gamma$, if } L(Y) = \lambda \\
        0 & \text{if } L(Y) < \lambda \\
    \end{cases} \]
    where $L(y)$ is the likelihood ratio, i.e. $L(y) = \frac{f_{Y \mid X}[y \mid 1]}{f_{Y \mid X}[y \mid 0]}$ and $\lambda, \gamma$ are chosen
    such that $\Pr{\hat{X} = 1 \mid X = 0} = \beta$.
\end{theorem}

Let us see Hypothesis testing in action.

\begin{example}
    Consider a Gaussian Channel $Y = X + Z$, where $\Normal{Z}{0}{\sigma^2}$ is independent of $X$.
    The receiver wants to guess $X$ from the received signal $Y$ with $PFA \leq \beta$. In this context,
    \[ L(y) = \frac{\exp(-\frac{(y - 1)^2}{2 \sigma^2})}{\exp(-\frac{y^2}{2 \sigma^2})} = \exp(\frac{2y - 1}{2 \sigma^2})\]
\end{example}