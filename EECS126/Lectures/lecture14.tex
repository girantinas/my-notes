\subsection{Lecture 14}

Most of the lecture was going through some claims in previous lectures; I updated those sections accordingly.

\begin{definition}[Stopping Time]
    Stopping time is a random variable $\tau$ that has the property
    $\{ \tau \leq t \}$ is completely determined by the process $\{ X_s, s \leq t \}$.
\end{definition}

An example of a stopping time is as follows.

\begin{example}[Gambling]
    Suppose we start with \$10 and bet \$1 each game, where we win \$1 with probability $p < 1/2$ and lose our money otherwise.

    The time to go bust is a stopping time, since it does not depend on knowing the evolution of the Markov chain after that time.

    The last time to have \$5 is not a stopping time, since it depends on knowing when you go bust (the future of that stopping time).
\end{example}

This leads to the following property (called the "Strong Markov Property").

\begin{theorem}[Strong Markov Property]
    If $\tau$ is a stopping time,
    \[ \Pr{\{X_{\tau + t}, t \geq 0\} \in A \mid X_{\tau} = k, X_t, t \leq \tau} = \Pr{\{X_t, t \geq 0\} \in A \mid X_0 = k}\]
    Note that this is stronger than the normal Markov property ($\tau$ can be an RV instead of a constant).
\end{theorem}

For CTMCs and DTMCs that we discuss, the Strong Markov Property holds.