% !TEX root = lectures.tex
\section{Lecture 15}
\subsection{Invariant Polynomials}
Let $G \curvearrowright \C^n$ be a representation of a finite group,
i.e. it acts linearly on $\C^n$ as some automorphism group.
This induces a natural action $G \curvearrowright \C[x_1, \dots, x_n]$,
which is the identity on constants.
\begin{definition}
    $\C[x_1, \dots, x_n]^G$ is the subring of $\C[x_1, \dots, x_n]$
    which contains the polynomials $f$ such that $gf = f$ for all $g \in G$.
\end{definition}
As a simple example, if $G = \{\pm 1\}$ which acts on $\C[x, y]$ by simple multiplication,
for instance $-1 \mapsto \begin{pmatrix}-1& 0 \\ 0 & -1\end{pmatrix}$ and $1 \mapsto \begin{pmatrix}1 & 0\\ 0 & 1\end{pmatrix}$
this means
\[ \C[x, y]^G = \C[x^2, xy, y^2] = \C[u, v, w]/(uw = v^2) \]
\begin{theorem}[Hilbert]
    With these conditions, $\C[x_1, \dots, x_n]^G$ is a finitely-generated ring over $\C$.
\end{theorem}
\begin{definition}
    A ring $R$ is generated over a subring $k$ (not necessarily a field) by elements $f_1, \dots, f_n \in R$ if
    any subring of $R$ containing $k$ and $f_1, \dots, f_n$ is all of $R$.
\end{definition}
In other words, we can make any element of $R$ as polynomials in $k$ and $f_1, \dots, f_n$.
This is equivalent to last time, where $R = k[f_1, \dots, f_n]/I$, 
because it's equivalent to saying that there's a surjective map $k[f_1, \dots, f_n] \rightarrow R$ ($I$ is the kernel).

Also, one should note that a subring of a finitely-generated ring need not be finitely generated.
\begin{example}
    Consider $\C[x, xy, xy^2, \dots] \subset \C[x, y]$. The second ring is clearly finitely generated
    by the two generators $x$ and $y$. But suppose you had a finite basis for the first ring, where the term with power 1 of $x$ and largest power of $y$ was $xy^p$.
    We cannot make $xy^{p+1}$.
\end{example}
With this in hand, let's prove the other Hilbert's theorem.
\begin{proof*}
    % TODO: Figure out why homogenous is required.
    Let $I \subset \C[x_1, \dots, x_n]$ be the ideal generated by
    all non-constant homogenous (all terms of the same degree)
    $G$-invariant polynomials (call such polynomials HNCGI polynomials). By Hilbert's theorem, the ring is Noetherian,
    so $I = (g_1, \dots, g_r)$ is finitely generated (as an ideal over the ring $\C[x_1, \dots, x_n]$).
    We claim that we can take these WLOG to be HNCGI.
    To see this, by definition,
    $g_i = r_1^{(i)} f_1^{(i)} + \dots + r_{k(i)}^{(i)} f_{k(i)}^{(i)}$
    so we can just replace $I = (f_1, \dots, f_s)$ where all the $f$'s are HNCGI. We will claim by induction on the
    degree that any HNCGI $f$ is a polynomial in $f_1, \dots, f_s$.

    We can write $f = r_1 f_1 + \dots + r_s f_s$ for $r_i \in \C[x_1, \dots, x_n]$. We will apply
    the averaging operator which applies $Af = \frac{1}{|G|} \sum_{g \in G} g(f)$. This operator has
    three useful properties:
    \begin{enumerate}
        \item $\textrm{im} A \subset \C[x_1, \dots, x_n]^G$ because applying any group element
    will just permute the order of the sum, which doesn't change anything.
        \item For $f \in \C[x_1, \dots, x_n]$ we have $Af = f$, since every term gives $f$.
        \item We have for each product term:
        \[ A(r_1 f_1) = \frac{1}{|G|} g(r_1 f_1) = \frac{1}{|G|} \sum g(r_1) g(f_1) = \frac{1}{|G|} \qty(\sum g(r_1)) f_1\]
        Since $f_1$ is invariant.
    \end{enumerate}
    Thus, applying the average of both sides yields
    \[ f = A(r_1) f_1 + \dots + A(r_s) f_s \]
    By induction on degree, since $A(r_1), \dots, A(r_s)$ are $G$-invariant
    and have smaller degree than $f$ (since $f_i$ are homogenous and nonconstant).
    They may constants, but that's fine for our claim. If not,
    they might not be homogenous, but it's definitely a finite sum of homogenous things,
    which can each be written by induction as polynomials in $f_1, \dots, f_s$.
    Which means $f$ can be written as a $\C$-polynomial in $f_1, \dots, f_s$, so the ideal is
    finitely-generated (as a ring).
\end{proof*}

\subsection{Symmetric Polynomials}
\begin{example}
    Let $S_n \curvearrowright \C[x_1, \dots, x_n]$ by $\sigma: x_i \mapsto x_{\sigma(i)}$.
    Then $\C[x_1, \dots, x_n]^{S_n}$ is called the set of \textbf{symmetric polynomials}.
    Consider
    \[(X+x_1)(X+x_2)\dots(X+x_n) = e_0 X^n + e_1 X^{n - 1} + \dots + e_n \]
    Then 
    \begin{align*}
        e_0 &= 1 \\
        e_1 &= x_1 + x_2 + \dots + x_n \\
        e_2 &= x_1 x_2 + x_1 x_3 + \dots \\
        \vdots& \\
        e_n &= x_1 \dots x_n
    \end{align*}
    where all the $e$'s are symmetric polynomials. A symmetric polynomial is not necessarily one these,
    like $x_1^2 + x_2^2 + x_3^2$. But actually, any symmetric polynomial
    can be written in terms of these ``elementary symmetric polynomials.''
\end{example}

\begin{theorem}
    $\C[x_1, \dots, x_n]^{S_n} = \C[e_1, \dots, e_n]$
    \textbf{Proof. } By induction on degree on highest lexicographic monomial, $f - ae_1^{r_1-r_2}e_2^{r_2-r_3}\dots e_n^{r_{n-1} - r_n}$
    can cancel out an $a x_1^{r_1} \dots x_n^{r_n}$ term.
\end{theorem}
Also the $e_i$ are algebraically independent, where there are no relations between the generators.

\begin{theorem}
    $R = \C[x_1, \dots, x_n]$ is a free $R^{S_n}$-module of rank $n!$.
    \begin{proof}
        Consider the case of $n = 3$. $S_3 = \langle s_1, s_2 \mid s_1^2=s_2^2=1, s_1 s_2 s_1 = s_2 s_1 s_2 \rangle$
        where $s_1 = (1 2)$ and $s_2 = (2 3)$. The amount of simple reflections needed to generate a permutation
        we will call its length.
        
        We start with discriminant $\frac{1}{6}(x_1 - x_2)(x_1 - x_3)(x_2 - x_3)$
        and apply operators $\partial_1 f = \frac{f - s_1 f}{x_1 - x_2}$
        and $\partial_2 f = \frac{f - s_2 f}{x_2 - x_3}$.
        
        Here are some figures: TODO
    \end{proof}
\end{theorem}



