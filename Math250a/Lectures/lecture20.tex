% !TEX root = lectures.tex
\section{Lecture 20}
\subsection{Normal Extensions}
\begin{definition}
    $K$ is a splitting field over $k$ for $f$
    if all the roots of $f$ lie in $K$ (e.g. over $K$,
    $f(x) = \prod_{i = 1}^{\deg f} (x - \alpha_i)$)
    and $K$ is generated by the roots of $f$.
\end{definition}

\begin{definition}
    An algebraic extension $K$ is normal over $k$ if one of the following
    equivalent  
    \begin{enumerate}
    \item There is a set $\alpha_i \in K$
    such that $K \supset$ of the splitting field of $\textrm{Irr}(\alpha_i, k, x)$.
    \item Every irreducible polynomial in $k[X]$
    which has a root in $K$ splits into linear factors in $K$.
    \item Consider an embedding $\sigma_l: K \to k^a$. Then every $\sigma: K \to k^a$
    has image $\sigma_l(K)$, i.e. every embedding induces an automorphism of $K$.
    \end{enumerate}
\end{definition}

A normal field $K$ contains the 
splitting field for $\textrm{Irr}(\alpha, k, X)$
fo all $\alpha \in K$. TODO: fill in the details here

\begin{theorem}
    If $[K: k] = 2$, then $K$ is normal over $k$.
    \begin{proof}
        Let $\alpha$ be a root of $f = \text{Irr}(\alpha, k, x)$
        then $(X - \alpha) \mid f$ in $K[X]$, so $f/(X-\alpha) = (X - \beta) \in K[X]$,
        which means that $\beta \in K$.
    \end{proof}
\end{theorem}

For fields with characteristic greater than $2$,
we can complete the square.
\[ f = x^2 - ax + b = \qty(x - \frac{1}{2}a)^2 - \frac{1}{4} a^2 + b \]
so $\qty(x - \frac{1}{2}a)^2$
This means $K(\alpha) = K(\sqrt{1/4 a^2 + b})$, so
for a quadratic extension,
we can always adjoin a square root.
\begin{example}
    Consider the following two quadratic extensions:
    \[ \Q \subset \Q(\sqrt{2}) \subset \Q(2^{1/4}) \]
    Note that each of the extensions are normal with respect to the previous field,
    but $\Q(2^{1/4})$ is not normal in $\Q$, because there are solutions to $x^4 - 2 = 0$
    that are in $\C \setminus \R$, but $\Q(2^{1/4}) \subset \R$.
\end{example}

\begin{theorem}
    Compositum and intersection of normal extensions inside a given algebraic closure field $k^a$, is normal.
\end{theorem}

\begin{theorem}
    There is always a smallest extension for algebraic extension $k \subset K$
    which is normal, called $k'$. To do this, take $\bigcap_{K' \supset K} K'$
    such that $K'$ is normal over $k$.
    To make a normal cover, take elements of big $K[X]$ and adjoining all their roots (in the algebraic closure).
\end{theorem}

The degree of the normal closure of $k(\alpha)$ is at most $[k(\alpha): k]!$.

\begin{theorem}
    Every extension of a field of characteristic $0$ is separable.
    \begin{proof}
        If we have extension $k(\alpha)$,
        we need to make sure $f = \textrm{irred}(\alpha, k , x)$
        has no multiple roots. Then
        $f'(\alpha) = 0$ either means $f' \mid f$ in $k$, which cannot be the case
        or $f' = 0$. But if $f = x^d + a_d x^{d- 1} + \dots \neq 0$,
        then $f' = dx^{d - 1} + \dots$ is not $0$ because $d \neq 0$.
        This is a contradiction, so therefore such a polynomial cannot have multiple roots.
    \end{proof}
\end{theorem}

\begin{theorem}
    Any group of units of a finite field is cyclic.
\end{theorem}

\begin{theorem}
    If $K$ is finite and separable then $\exists \alpha$ such that $K = k(\alpha)$.
    \begin{proof}
        Without loss of generality $K = k(\alpha, \beta)$.
        Let $\sigma_1, \dots, \sigma_n$ be the maps $K \to k^a$ over $k$.
        If $k$ is finite, then $k(\alpha, \beta) = k(\gamma)$
        for any generator $\gamma$ of $K^*$. Otherwise, we claim
        that there exists $r \in k$ such that $k(\alpha + r \beta) = k(\alpha, \beta)$.
        We will choose $r$ such that $\sigma_i (\alpha + r \beta)$ are all distinct.
        Then $k(\alpha + r \beta) = k(\alpha, \beta)$, because TODO
        \[ g(x) = \prod_{i \neq j}  (\sigma_i \alpha + x \sigma_i \beta) - (\sigma_j \alpha + x \sigma_j \beta)\]
        is nonzero, so $g(r) \neq 0$ for some $r$, because the characteristic is $0$ (there are only finitely many roots).
    \end{proof}
\end{theorem}

\begin{theorem}
    If $K$ has a primitive element, then there are only finitely many proper fields $k \subset K' \subset K$.
\end{theorem}
TODO: Look at proof of this fact