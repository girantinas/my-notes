% !TEX root = lectures.tex
\section{Lecture 14}
\subsection{Eisenstein's Criterion}
\begin{theorem}[Eisentein]
    Let $R$ be a UFD. Let $f = \sum_{k = 0}^n a_k X^k \in R[X]$.
    If there exists $p \in R$ prime
    such that $p \mid a_0, a_1, \dots a_{n - 1}$, but $p \not\mid a_n$
    and $p^2 \not\mid a_0$, then $f$ is irreducible.
    \begin{proof*}
        Suppose that $f = gh$ where $g = \sum_{k = 0}^m b_k X^m$ and $h = \sum_{k = 0}^{\ell} c_k X^k$.
        Since $p$ is prime and $p \mid a_0 = b_0 c_0$, but $p^2 \not\mid b_0 c_0$,
        exactly one of them is divisible by $p$ (say $p \mid b_0, p \not\mid c_0$).
        Let $b_r$ be the lowest coefficient of $g$ such that $p \not\mid b_r$ (this exists because $p$ does not divide $a_n = b_m c_{\ell}$).
        Then, $a_r = b_r c_0 + b_{r - 1} c_1 + \dots$. But $a_r$ is divisible
        by $p$ and all the lowest $b$ coefficients is divisible by $p$, so $p \mid b_r$, which is a contradiction.
    \end{proof*}
\end{theorem}
This result is also true for the fraction field. To recall why this is true, suppose $f = gh$ where $g, h \in k[x]$.
Then we could write $f = \text{cont}(f) \frac{g}{\cont(g)} \frac{h}{\cont(h)}$,
where $\text{cont}(f) \in R$ and $\frac{g}{\cont(g)}, \frac{h}{\cont(h)} \in R$ by being content-1.

\begin{example}
    Consider $f(X) = X^{p - 1} + X^{p - 2} + \dots + 1$. We claim that $f$ is irreducible.
    We change variables $Y = X - 1$.
    Then 
    \begin{align*}
        f &= \frac{X^p - 1}{X - 1} \\
        &= \frac{(Y + 1)^p - 1}{Y} \\
        &= \frac{Y^p + \binom{p}{p-1} Y^{p - 1} + \dots + \binom{p}{1} Y }{Y} \\
        &= Y^{p - 1} + \binom{p}{1} Y^{p - 2} + \dots + \binom{p}{1}
    \end{align*}
    Since $p \mid \binom{p}{i}$ for $0 < i < p$ and $p^2 \not\mid \binom{p}{1} = p$, then Eisenstein's criterion applies.
\end{example}

\subsection{Noetherian Rings and Hilbert's Theorem}
\begin{theorem}
    Let $R$ be a commutative ring. Then the following are equivalent.
    \begin{enumerate}
        \item Every ideal is finitely-generated.
        \item Every ascending chain $I_1 \subset I_2 \subset \dots$ eventually terminates (eventually $I_N = I_{N + 1} = \dots$).
    \end{enumerate}
    \begin{proof}
        \textbf{(1) $\implies$ (2)}
        Let $I_1 \subset I_2 \subset \dots$. Consider their union $I = \bigcup I_i$;
        this is still an ideal. By assumption, $I = (a_1, \dots, a_n)$ for some $a_i \in R$. Then each $a_i$ is contained in
        some finite $I_{n_i}$. Then, just take $N = \max_i n_i$, then $I_N \supset I_{n_i} \ni a_i$, so $I \subset I_N$
        and thus the ideals must terminate after that.

        \textbf{(2) $\implies$ (1)}
        Assume that there isn't an ideal which isn't finitely generated. Then there exist an infinite set of elements
        $\{a_i\}_{i =1}^{\infty}$ such that definining $I_{i - 1} = (a_1, a_2, \dots, a_{i - 1})$,
        $a_i \notin (a_1, a_2, \dots, a_{i - 1})$, so $I_1 \subset I_2 \subset \dots$ doesn't terminate.
    \end{proof}
\end{theorem}
\begin{definition}
    A ring is called \textbf{Noetherian} if either of the above is true.
\end{definition}
\begin{example}
    Consider a non-example, the polynomial ring on infinitely many variables $k[X_1, X_2, \dots]$. Note that $(X_1, X_2, \dots)$ is
    not f.g. so it fails the first item and $(X_1) \subset (X_1, X_2) \subset \dots$ doesn't terminate.
\end{example}

\begin{theorem}[Hilbert]
    If $R$ is a Noetherian ring, then $R[X]$ is also Noetherian.
    \begin{proof}
        Let $I \subset R[X]$ and let $I_i = (a_i \mid \exists a_0 + \dots + a_i X^i \in I) \subset R$ (we are only capturing the leading coefficient in the
        the generator builder notation). Then $I_0 \subset I_1 \subset \dots$ is an ascending chain of ideals, because if $a_i \in I_i$ then $p_i(X) = a_0 + \dots + a_iX^i \in I$,
        so $Xp_i(X) =  a_0 X + \dots + a_iX^{i+1} \in I$, so then $a_i \in I_{i + 1}$. Since $R$ is Noetherian, there exists
        $I_r = I_{r + 1} = \dots$, a termination ideal. Let $S_i \subset I$ be a finite set of degree $i$ polynomials whose $X^i$ coefficient
        generate $I_i$ (this exists because $R$ is Noetherian). Calling $I_i = (a_i^1, a_i^2, \dots)$ (where the superscripts are indices), we would have
        $S_i = \{a_0^1 + \dots + a_i^1 X^i, a_0^2 + \dots + a_i^1 X^i\}$. By iteratively subtracting off the leading term, any polynomial in $I$
        is generated by $S_1 \cup \dots \cup S_r$. That is, if $f = b_0 + \dots + b_n X^n \in I$,
        then if $n \le r$, we can subtract them out by an appropriate element of $I_n$ to knock down the power by 1.
        If $n > r$, $S_r = S_{r + 1} = \dots$, so we can just take the relevant generator from $S_r$ and multiply by $X^{n - r}$
        to reduce the power by 1.
        So $I = (S_1, \dots, S_r)$ is finitely generated.
    \end{proof}
\end{theorem}
If $R=k$ is a field, then $k[X]$ is a PID and thus Noetherian. If $R = k[Y]$ so $R[X] = k[X, Y]$; there are arbitrarily large ideals,
but each one is finitely-ginerated. To see the first part, $(X^n, X^{n - 1}Y, X^{n - 2}Y^2 , \dots , Y^n)$ cannot be generated by $n$ elements.
\begin{example}
    Here is an example of the proof in action. Let $R = \Z[Y]$ and $I = (2, 1 + YX, Y + X^3) \subset R[X]$.
    Then $I_0$ has all the coefficients of degree $0$ polynomials, so $I_0 = (2)$.
    Furthermore, $I_1$ has all the coefficients of degree $1$ polynomials, wherein we have $YX$ and $2X$,
    so $I_1 = (2, Y)$. Similarly for quadratics $I_2 = (2, Y)$. Now for cubics, we can make $1$, so $(1) = R = I_3 = I_4 = \dots$.
    Then, we construct $S_0 = \{2\}, S_1 = \{2X, 1+YX\}, S_2 = \{ 2X^2, X + YX^2 \}, S_3 = \{ Y + X^3 \}$.
    Let $f = (3 + Y) + (3 + 2Y) X + YX^2 + X^3 + X^4 \in I$. Then we can subtract off:
    \begin{align*}
        f &= (3 + Y) + (3 + 2Y) X + YX^2 + X^3 + X^4 \\
        f - X(Y+X^3) &= (3+Y) + (3+Y)X + YX^2 + X^3 \\
        f - X(Y + X^3) - (Y + X^3)&= 3 + (3+Y)X + YX^2 \\
        f - X(Y + X^3) - (Y + X^3) - (X + YX^2) &= 3 + (2+Y)X \\
        f - X(Y + X^3) - (Y + X^3) - (X + YX^2) - (2X + 1 + YX) &= 2 \\
        f - X(Y + X^3) - (Y + X^3) - (X + YX^2) - (2X + 1 + YX) - (2) &= 0
    \end{align*}
\end{example}
A corollary is that every finitely-generated ring over a Noetherian ring is Noetherian.
\begin{definition}
A ring
$S$ is finitely generated over $R$ if $S = R[X_1, \dots, X_n]/I$. for some ideal $I$.
\end{definition}

Let $G$ as an action on $\C^n$, a representation of a finite group. We can extend this to an action $G$ on $\C[X_1, \dots, X_n]$.
\begin{theorem}
    $\C[X_1, \dots, X_n]^G$ is a finitely generated ring.
\end{theorem}

For example, let $G = \{\pm 1\}$ acting on $\C[X_1, X_2$ where $-1 \cdot X_1 = -X_1$ and $-1 \cdot X_2 = -X_2$.
Then $\C[X_1, X_2]^G = \C[X_1^2, X_1 X_2, X_2^2] = \C[u, v, w]/(uw = v^2)$.