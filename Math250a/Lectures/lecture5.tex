% !TEX root = lectures.tex
\section{Lecture 5}
\subsection{Modules over Arbitrary Rings}
For a ring $R$, a left-module is an abelian group $M$ with a pairing $R \times M \to M$
which we will apply as multiplication.
This action is associative and distributive, as usual.
\begin{theorem}
    If $0 \to M' \to M \xra{b} M'' \to 0$ is a short exact sequence and a map from a free module $c: F \to M''$,
    then there exists a map $d : F \to M$ that makes the diagram commute.
    \begin{proof}
        Write $F \cong \bigoplus_i Re_i$. Then, $c(e_i) = m_i$ for some $m_i$.
        Since the map $b$ is onto, there exists $n_i$ such that $b(n_i) = m_i$. Thus,
        define $d$ as the map sending $e_i \to n_i$ and extending by linearity.
    \end{proof}
\end{theorem}
As a special case, if $0 \to M' \to M \xra{b} F \to 0$ is exact, then we can take $c = \text{id}_F$,
so there exists $d: F \to M$ where the composition of $b$ and $d$ yields identity; this is a section.
So $M \cong F \oplus M'$.

\begin{definition}
    $P$ is projective if given $b: M \mono M''$
    and a map $c: P \to M''$,
    there exists $d: P \to M$ such that $bd = c$.
\end{definition}

\begin{example}
    Consider the following polynomial ring:
    \[ R = \frac{k[x_1, x_2, x_3, y_1, y_2, y_2]}{(\sum_{(y_1, y_2, y_3)} x_i y_i = 1)} \]
    gives us exact sequence $0 \to R \to R^3 \to P \to 0$.
\end{example}

The nice thing about looking at things categorically is that we can turn around the arrows involved.
    
If $R$ is an \textbf{injective} $R$-module when considering $0 \to E \to M \to M'' \to 0$,
the property from before holds with all the arrows reversed.

For example we showed that for a PID $R$,
if $M$ is a torsion module and $p \in R$ is a prime
such that $p^a M = 0$, which is equivalent to $M$ is an $R/p^a$-module.
We showed that then, $R' := R/p^a$ is a summand of $M$.

\subsection{Groups}
\begin{definition}
    A \textbf{group} is a set with one operation $G \times G \to G: (a, b) \mapsto b$.
    This operation is associative, has a unit, and has inverses.
\end{definition}
There's a school of thought that thinks this definition is not very good. How do they define a group?
\begin{definition}
    A \textbf{group} is a set of permutations (bijections) of a given set $S$.
    This set should be closed under composition and inverses.
\end{definition}
To see that these notions are equivalent, we can take $S = G$; then group multiplication is a group
action of $G$ on itself.
Since these actions have inverses (multiplication by $g^{-1}$),
all of them are permutations.

Every permutation can be written as a product of unique disjoint cycles (up to order of factors).
To see this, consider the following greedy algorithm:
\begin{enumerate}
    \item Take some element $a \in S$. See where it maps to in finite compositions of the permutation.
    \item Whatever it doesn't ever lead to, create a new cycle starting with it.
    \item Repeat this until you run out of elements.
\end{enumerate}
We will denote a cycle as $[a_1, \dots, a_r]$.
\begin{definition}
    A $G$-set $S$ is a set with action $A: G \times S \to S$ (a homomorphism
    from $G \to \text{Perm}(S)$ where $(gh)(s) = g(h(s))$), where $(g, s) \to g(s)$ where the submap $s \to g(s) $ is a permutation.
\end{definition}
\begin{definition}
    The action of $G$ on its $G$-set is \textbf{transitive} (or the set itself) if for any $s \in S$,
    we have $Gs = S$.
\end{definition}
Not all actions are transitive. For example, take $G = \Z/2$ and act on the set $\{1, 2, 3\}$,
which we denote as $\Z/2 \hookrightarrow \{1, 2, 3\}$. Consider the action sending $0 \to \text{id}$
and $1 \to [1, 2]$.
\begin{theorem}
    Every $G$-set is the disjoint union of transitive $G$-sets.
\end{theorem}
To see this, just decompose $S$ into its orbits, for example, the orbit of $1$ and $2$ are $\{1, 2\}$
and the orbit of $3$ is $\{3\}$.
\begin{definition}
    Consider $S$ is a $G$-set and $s \in S$. Then the \textbf{orbit} of $s$ is $Gs = \{ gs \mid g \in G \}$.
\end{definition}
Consider $G$ acting on $S$ transitively. What elements of $G$
have an element $s \in S$ as a fixed point? 
\begin{definition}
    The \textbf{Stabilizer} of an element $s$ as
    \[ \text{Stab}_G(s) = G_s = \{ g \in G \mid gs = s \} \]
    This is a subgroup of $G$.
\end{definition}
It turns out, once you figure out what the stabilizer is for one element for a transitive
action, you have uniquely determined the action.
\begin{definition}
    A \textbf{subgroup} $H \le G$ for group $G$ is a subset of $G$ which is itself a group.
    For strict containment, we have $H < G$.
\end{definition}
\begin{definition}
    A coset of $H \leq G$ is a $gH \subseteq G$, e.g. $gH = \{gh \mid h \in H\}$.
    The set of cosets is denoted as $G/H$.
\end{definition}
Two cosets $gH$ and $kH$ for $g, k \in G$ are either equal or disjoint.
If $gH \cap kH \neq \emptyset$, then for $h, h' \in H$
\begin{align*}
    gh &= kh' \\
    ghH &= kh' H \\
    gH &= kH
\end{align*}
As a corollary, we get that
\begin{theorem}
    If for finite $G$, $H \le G$, then $|H| \mid |G|$. 
    \begin{proof}
    $G = \bigcup_{g \in G} gH$, some set of which are disjoint,
    and all of the cosets are the same size.
    \end{proof}
\end{theorem}
Finally, it turns out we can identify these two things.
\begin{theorem}
The set of cosets of $H \le G$ is a $G$-set with action $g(g'H) = (gg')H$.
If $G \hookrightarrow S$ is transitive and $G_s = H$,
then there is a bijection from $S$ to the set of cosets of $H$ which preserves
the action of $G$.
\begin{proof}
    Note that $\text{Stab}_G H \in G/H$ is exactly $H$.
    $(g, s) \to gH$ is clearly a surjection, because the action is transitive.
    Furthermore, $gs = g's$, then $g'^{-1}g s = s$, so $g'^{-1} g H = H$,
    meaning that $gH = g' H$, so we get the same coset. So the map is an injection too.
    We can also multiply by arbitrary group elements and get the exact same structure, proving the theorem.
\end{proof}
\end{theorem}

