% !TEX root = lectures.tex
\section{Lecture 21}
\subsection{Galois Extensions}
\begin{definition}
    A finite extension $K/k$ is \textbf{Galois} if it is separable and normal.
\end{definition}
Every extension of $k$ embeds into $k^a$. Separable means the number of homomorphisms
from $K \to k^a$ over $k$ (fixing elements on $k$) means that this is degree of the extension.

Every element of $K$ is separable over $k$, i.e. all the roots of irreducible
polynomials are not repeated, i.e. $\text{Irr}(\alpha, k, X)$ has no multiple roots for all $\alpha$.

$K/k$ is normal if it's the splitting field of every element. Or any two maps $K/k \to k^a/k$ have the same image.

\begin{definition}
    Suppose $K/k$ is Galois, then $\gal(K/k) = \aut(K/k)$, where the automorphisms FIX $k$.
\end{definition}

\begin{theorem}
    If $K$ is any field and $G$ is a finite group of automorphisms of $K$,
    then $K/K^G$ is Galois with $\gal(K/K^G) = G$.
    \begin{proof}
        Let $\alpha \in K/k$ for field $k \le |G|$. Suppose $\alpha, \sigma_1(\alpha), \dots, \sigma_k(\alpha)$
        are the distinct images of $\alpha$ under the elements of $G$. Then,
        the polynomial $f(x) = \prod_{i = 0}^{k}(x - \sigma_i(\alpha)) = \text{Irr}(\alpha, K^G, X)$ is fixed by $G$.
        Thus, the coefficients are in $K^G$, so $f$ is separable.
         Further, $K/K^G$
        is normal because it's the splitting field of the family of all of these polynomials for all $\alpha$.

        Clearly $G \subset \gal(K/K^G)$ by definition.
        We know $K = K^G(\alpha)$ by the primitive element theorem for some $\alpha$. Then
        $G$ acts transitively on the roots, because if $f(\alpha) = 0$, then $f^\sigma(\alpha) = 0$
        for all $\sigma \in G$. Let $n$ be the degree of $f$; then
        Thus, $|G| \ge n = [K: K^G] = [K: K^G]_s = |\gal(K/K^G)|$ because the extension is Galois.
    \end{proof}
\end{theorem}

\subsection{Finite Fields}
The easiest example of a finite field is $\Z/p$. If we have a finite field,
there is clearly a ring homomorphism from the integers sending $1 \mapsto 1$.
$Z \to F$ has a kernel $(p)$ which has $p > 0$ prime. If it weren't prime, then the two divisors would map to two nonzero
things that multiply to zero, contradicting the fact that $F$ is a field.
This means that $F \supset \Z/p$ and no other $\Z/p'$, which means that $|F| = q = p^m$.
$F^*$ is cyclic of order $p^m - 1$. Furthermore, for all $x \in F^*$ has $x^{p^m - 1}= 1$
and $x^{p^m} - x = 0$ for all $x \in F$. Take the splitting field of this polynomial over $(\Z/p)^a$.
We claim that all the roots themselves form a field. Well, if we add or multiply two elements, clearly they are still roots
(by using $(a + b)^{p^m} = a^{p^m} + b^{p^m}$ in characteristic $p$). In addition, it's not too hard to see that inverses exist.
Thus, there exists a unique $F_{p^m} \subset (\Z/p)^a$ splitting field inside a given algebraic closure,
and thus unique up to isomorphism. When is $F_{p^m} \subset F_{p^n}$? That's exactly when $F_{p^n}$
is a vector space over $F_{p^m}$. Thus, we need $p^n = (p^m)^b$ for some $b$, i.e. $m \mid n$.

\begin{theorem}
    Let $q = p^m$ and $p$ prime and $F_q / F_p$ is Galois.
    $\gal(F_q/F_p)$ is cyclic, generated by $\varphi: x \to x^p$.
\end{theorem}

\subsection{Inseparable Extensions}
Let $K/k$ be a finite extension. Then we define
\[ [K:k] = [K:k]_s \cdot [K:k]_i \]
In other words, $[K: k] = [K: k]_i$ if and only if
for all $\alpha \in K$, defining $f(x) = \text{irr}(\alpha, k, x)$
means that $f'(\alpha) = 0$. This happens if and only if $f'(x) = 0$.
If $f(x) = x^n + a_1 x^{n - 1} + \dots$, then $nx^{n - 1} + a_i(n - 1) x^{n - 1} + \dots = 0$.
This means $p \mid n, p\mid a_i(n - 1), \dots$. This happens if and only if $f(x) = g(x^p)$
for some other polynomial in $k[x]$. Thus, all roots of $f$ are multiple with the same multiplicity.
The root of $g$ must be inseparable as well. Thus, this must be true over and over,
e.g. $x^{p^m} - a$ was the polynomial all along. So, $K = k(\alpha_1, \dots, \alpha_s)$
and $\alpha_i^{p^m_i} \in k$.

\subsection{Compass and Straightedge Constructions}
Take $\R^2 \cong \C$. Start with a finite set of points $M$
on the plane. Without loss of generality, let's assume $M = \overline{M}$, i.e. it's symmetric about the horizontal axis.
Then you can ask what points once can ``construct'' given $M$. We claim is a subfield.
It turns out
constructability by compass and straightedge just corresponds to
adjoining square roots of numbers already in $M$.

\begin{theorem}
    $\alpha \in \C$ constructible in $M$ if and only if the Galois
    closure of $\Q(\alpha)$ has degree a power of $2$ over $\Q$.
    \begin{proof}
        If the Galois closure of $\Q(\alpha)$
        has degree a power of two.
        Recall that if we have a $p$-group $G$, then its center is nontrivial
        (it is actually itself a $p$-group). Here we take $p = 2$.
        We know that
        since the center is abelian,
        $\Z/2 \subset \Z/2 \oplus \Z/2 \subset \dots \subset C(G)$.
        But then we can take $C_2(G) = G/C(G)$, creating a tower upwards towards $G$.
        Then, by our theorem, there are a bunch of fields between $\Q$ and $\Q(\alpha)$
        made by these groups, as $F^{C_i(G)}$.

        If $\alpha$ is constructible, 
    \end{proof}
\end{theorem}

\subsection{Galois Theory, revisited}
\begin{theorem}
    Suppose $K/k$ is a finite Galois extension with
    Galois group $G = \aut_k(K)$.
    Then there is a $1$-to-$1$, order-reversing
    correspondence (called a Galois correspondence)
    $K \supset F \supset k$ and $1 \subset H \subset G$
    given by $F = K^H, H = \aut K/F$.
    \begin{proof}
        If we start at $F$ then look at $\aut K/F$
        then we get $F = K^H$
        TODO

        Clearly $K$ is normal over $F$ and it is separable over $F$
    \end{proof}
\end{theorem}
