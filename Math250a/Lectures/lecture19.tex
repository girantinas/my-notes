% !TEX root = lectures.tex
\section{Lecture 19}
\subsection{Field Theory}
A field is a ring where every element has a multiplicative inverse.
Some familiar fields:
\begin{enumerate}
    \item $\Q \subset \R \subset \C$ are common fields.
    \item If $k$ is a field, then $k(x_1, \dots, x_n)$,
    the set of rational functions with coefficients in $k$,
    is a field.
    \item $k[[x]][x^{-1}]$ is the field of Laurent series.
    \item $\Z/p$ is a field for prime $p$.
\end{enumerate}

\begin{definition}
    Let $k$ be a field. Then
    there is a ring homomorphism from $\Z \to k$
    that sends $1 \mapsto 1_k, 2 \mapsto 1_k + 1_k, 3 \mapsto 1_k + 1_k + 1_k, \dots$.
    The kernel of this ring homomorphism must be a prime ideal of $\Z$,
    call it $(n)$. Then we define the \textbf{characteristic} as $\text{char} k = n$.
\end{definition}

Often we want to adjoint polynomials to our fields.
If $p(x) \in k[x]$ is irreducible,
there is a \textbf{field extension} $k(\alpha)$
where $\alpha$ is a root of $p$. Namely,
$k(\alpha) = k[x]/(p(x))$ is a field, because $p(x)$ is a maximal ideal.
And this is exactly $k(\alpha)$ with isomorphism $x \mapsto \alpha$.
\begin{definition}
    Consider a field $F$ and an extension $F \subset E$. $[E : F] = \text{dim}_F(E)$.
    If $k \subset F \subset E$, then $[E: F][F: k] = [E: k]$.
\end{definition}

\begin{theorem}
    $[k[x]/(p) : k] = \text{deg} p$.
\end{theorem}

\begin{definition}
    Let $k \subset F$ be a field extension.
    We call $\alpha \in F$ \textbf{algebraic} over $k$ if it's the root of some polynomial in $k[x]$.
    There is a unique lowest degree monic irreducible polynomial.  
\end{definition}

Note that $\Q \subset \Q(x)$ is not algebraic, because the indeterminate $x$ is not a root of any polynomial
in $\Q[x]$.

\begin{definition}
    $k \subset F$ is \textbf{algebraic} if every element $\alpha \in F$ is
    algebraic over $k$.    
\end{definition}

\begin{theorem}
If $F$ is a finite extension ($\text{dim}_k F < \infty$) of $k$,
then it's a algebraic over $k$.

\begin{proof}    
Then take $\alpha \in F$
and consider powers $1, \alpha, \alpha^2, \dots$.
These cannot all be linearly independent because the dimension as a
vector space is finite. Thus, there is a linear combination $\sum_{i = 1}^{\dim F} c_i \alpha^i = 0$,
thus $\alpha$ is a root of $p(x) = \sum c_i x^i$.
\end{proof}
\end{theorem}

\begin{theorem}
    If a field extension $F$ is algebraic and finitely-generated over $k$,
    then it's finite over $k$.
    \begin{proof}
        Write $F = k(\alpha_1, \dots, \alpha_n)$.
        Note that $k \subset k(\alpha_1) 
        \subset k(\alpha_1, \alpha_2)
        \subset \dots \subset F$.
    \end{proof}
\end{theorem}

The set of finite extensions of $\C(x)$
is exactly the set of Riemann surfaces.
The set of finite extensions of $\Z/p$ are all the finite fields
with the increasing powers of $p$.

\begin{theorem}
    Consider two extensions of $k$, $F$ and $E$ and
    maps $\sigma: F \to E$ over $k$, e.g. $\sigma(a) = a$
    for all $a \in k$. If $\alpha \in F$ is a root of $p(x) \in k[x]$,
    then 
    \[ 0 = \sigma(0) = \sigma(p(\alpha)) = \sigma(p)(\sigma(\alpha)) = p(\sigma(\alpha)),\]
    so $\sigma(\alpha)$ is a root of $p$.
\end{theorem}
As a corollary, is $k \subset F$ is algebraic, then $\sigma: F \to F$ is
always an isomorphism. To prove this, all nontrivial field homomorphisms are injective, because
the kernel of a ring homomorphism is always an ideal, and fields have no nontrivial ideals.
Furthermore, $\beta \in F$ satifies $\text{irr}(\beta, k, x) = p(x)$.
Saying $\beta_1, \dots, \beta_s$ are roots of $p$ in $F$.
But $\sigma$ doesn't change the degree of $p$,
so $\sigma(\beta_i) = \beta_{\pi(i)}$ for some permutation $i$. Thus, $\beta \in \text{image} \sigma$,
so the map is a surjection.

\begin{theorem}
    Given any field $k$,
    there exists an algebraic extension $k \subset k^a$,
    called the \textbf{algebraic closure}, 
    such that $k^a$ is \textbf{algebraically closed},
    which means there are no n
    ontrivial algebraic extensions or equivalently,
    every polynomial in $k^a[x]$ splits 
    as a product of linear factors (or has $\deg$ roots in $k^a$).

    \begin{proof}
        Find all polynomials in $k[x]$, and look at
        \[ \{ p_{\alpha}(x_{\alpha}) \text{ are the irreducible polynomials.} \}\]
        Now $k \subset k_1 = k[\{x_{\alpha}\}]/(\{p_{\alpha}\}) \subset k_2 \subset \dots$.
        If there are no more irreducible polynomials at some stage $j$, then $k^a = k_j$.
        Otherwise, we set $k^a = \frac{k[x_{\alpha}, x_{\beta}, \dots]}{(p_{\alpha}, p_{\beta}, \dots)}$
        to be the union.
        As long as the ideal on the bottom doesn't have $1$,
        we should be good; which it doesn't, otherwise we would've terminated at a finite stage.
        Furthermore, $k^a$ is unique up to isomorphism because $F \cong E$ by just taking
        a field homomorphism between them.
    \end{proof}
\end{theorem}
We can write
\[ \C[x]/(x^2 - 2) = \C[x]/(x-\sqrt{2}) \times \C[x]/(x + \sqrt{2}) = \C \times \C \]
this is not a field! Algebraic closure really says that we cannot make a field any bigger.

Furthermore, $\Z/p(s, t) \subset Z/p(s^{1/p}, t^{1/p})$
cannot be generated by two elements.

\begin{theorem}
For a field of finite characteristic $k$ and
$F = k(\alpha)$ algebraic, then there are at most finitely many fields
in between $k$ and $F$.

And conversely if there are at most finitely many fields, then
there is a single generator.
\end{theorem}

\begin{theorem}
Let $k \subset F$ be an algebraic field extension.
Define $\Sigma = \{ \sigma: F \to k^a \text{ over } k \}$.
Then $[F: k]_s := |\Sigma|$ is finite.
\begin{proof}
    First, it's clear that if $k \subset F \subset E$ algebraic,
    then $[E: k]_s = [E: F]_s [F: k]_s$.
    To specify a new map, the extra freedom we get is specified
    by $[E: F]_s$.

    $k(\alpha) = F$, then $\Sigma \leftrightarrow \{\text{roots in } k^a \text{ of } \text{irr}(\alpha, k, x)\}$,
    then $[k(x) : k] \le \text{deg}(\text{irr}(\alpha, k, x))$.
    TODO
\end{proof}
\end{theorem}

\begin{definition}
    $F$ as an algebraic extension over $k$ is separable if $[F:k] = [F:k]_s$.
\end{definition}

\begin{theorem}
    $F$ over $k$ is separable if and only if $E/k$ and $F/E$ are separable.
\end{theorem}