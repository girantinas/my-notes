% !TEX root = lectures.tex
\section{Lecture 8}
\subsection{Semi-direct Product}
\begin{theorem}
    If $N \triangleright G$ and $H \le G$ such that $H \cap N = \{1\}$ and $HN = G$,
    then $G$ is the semi-direct product, i.e. $G = N \times H$ as a set,
    and we have the multiplication $(n, h)(n', h') = (n h n' h^{-1}, h h')$. This is isomorphic to the direct product.
\end{theorem}

\subsection{Simplicity of $A_n$}
\begin{definition}
    The alternating group is the kernel of the map $\mu: \Sigma_n \to \{\pm 1\}$
    which maps
    \[ \mu: \sigma \mapsto \frac{\prod_{i < j} (x_i - x_j)}{\prod_{i < j} (x_{\sigma_i} - x_{\sigma_j})} \]
    also known as the ``even'' permutations.
\end{definition}
Since $\Sigma_n$ is generated by transpositions, $A_n$'s are made up of
even amounts of transpositions. Every product of odd cycles is in $A_n$, e.g. because $(1 2 3) = (1 2)(2 3)$.
In fact, $A_n$ 
\begin{theorem}
    If $n \ge 5$, then $A_n$ is a simple group.
    \begin{proof}
        We will induct on $n$. First, for $n = 5$, note that $|A_5| = \frac{5!}{2} = 60$. 
        We proceed by contradiction. Consider a Sylow $5$-subgroup of $\Sigma_5$, call it $S_5$.
        Note that $|S_5| = 5$, and $S_5 \cong \Z/5$.
        Note that this is exactly a proper cycle of length $5$. $[A_5 : \Sigma_5] = 2$, so
        if $S_5 \not\subset [A_5 \cap S_5 : S_5] = 2$,

        
        Take $N \triangleleft A_5$.
        The first possibility is that $5 \mid |N|$, then $N$ would be the unique Sylow $5$-subgroup, which is a contradiction.
    \end{proof}
\end{theorem}

