% !TEX root = lectures.tex
\section{Lecture 1}

\subsection{Rings}
Recall that an abelian group is set equipped with an operation that works like addition: you can add and subtract, it's commutative,
associative and monoidal.
\begin{definition}
    A set $R$ is a ring if it is an abelian group equipped with an associative ``multiplication'' operation which has a unit $1$, where $1a = a$
    and this multiplication distributes over addition.
\end{definition}
The smallest ring is the zero ring, where $1 = 0$ (and the only element is $0$). Other examples of rings are $\Z, \Q, \R, \C, \text{quaternions}$.
Less obvious are
the polynomial rings, e.g. $\C[x_1, \dots, x_n]$ or $M_n(\R)$ (the $n \times n$ matrices over $\R$) or $\Z[G]$ (linear combinations of elements of a group $G$).
Even fancier is derivative ring $\C[x_1, \dots, x_n, \partial_1, \dots, \partial_n]$, where $x_i$ commutes with $x_j$ and $\partial_i$ commutes with
$\partial_j$ and $\partial_i$ commutes with $x_j$ for $i \neq j$, but $\partial_i x_i - x_i \partial_i = 1$ (this is a re-arrangement of the product rule).

\begin{definition}
    Consider a commutative ring $R$. $I \subseteq R$ is an ideal if $I$ is a subgroup of $R$ (over the operation of addition) and it's closed under multiplication, e.g.
    for $r \in R$ and $i \in I$, $ri \in I$.
\end{definition}
Ideals are generated by coprime elements; if they share a factor, some reduction can occur a la gcd and Bezout's.
$R$ is going to stand for a commutative ring from henceforth.
\begin{definition}
    Consider a commutative ring $R$. $R$ is a domain (or integral domain or entire ring) if $ab = 0 \implies a = 0 \text{ or } b = 0$.
\end{definition}
\begin{definition}
    Consider a commutative ring $R$. $R$ is a principal ideal ring (or principal ring) if every ideal is generated by 1 element.
\end{definition}
A principal ideal domain is both a principal ring and a domain.
We work towards the following result.
\begin{theorem}
    Every finitely-generated module over a principal ideal domain is a direct sum of cyclic modules.
\end{theorem}
What do all of these words mean?
\begin{definition}
    A module (or representation) over a ring $R$ (or $R$-module) is an abelian group $M$ combined with the operation
    of scalar multiplication by elements of $R$ that distributes over addition. So for $r,s \in R, m,n \in M$, then $(r+s)(m + n) = rm + rn + sm + sn \in M$.
\end{definition}
All vector spaces are modules over their field. The integers mod 12 is a $\Z$-module with integer multiplication
as the scalar multiplication. Also $\C[x] \oplus \C[x]$ where $p (a, b) = (pa, pb)$. Furthermore, 

A product of rings $R_i$, $\prod_i R_i$ is a funny object.
\begin{definition}
    The product of rings $\prod_i R_i$ is the unique ring such that it has projection maps $\pi_j: \prod_i R_i \to R_j$ for any ring $S$ with maps $f_j: S \to R_j$ there exists a unique
    map $f: S \to \prod_i R_i$ such that $f_j = \pi_j \circ f$.
\end{definition}
The above property is called the universal property. The direct product of rings is just a ring where you just tuple together the ring elements to make a ring element.

The direct sum is similar, but with all the maps reversed. That is why it
is sometimes called the coproduct.
\begin{definition}
    An $R$-module $A$ is the direct sum of $R$-modules $M_i$, $i \in I$ if there are maps $\phi_i: M_i \to A$ (reverse projections) and given
    a module $B$ with maps $g_i: M_i \to B$, there exists a unique map $g: A \to B$ such that $g_i = g \circ \phi_i$.
\end{definition}
The claim is that $A$ is also a set of tuples, but $A = \{ m \in \prod_i M_i \mid m_i = 0 \text{ for all but finitely many $i$} \}$
% Quote: "You can add 727 elements..."
\begin{definition}
    A module is cyclic if it is generated by one element. This element is called the generator. It is typically denoted as:
    \[ Rm = (m) = \{ rm \mid r \in R \} \]
\end{definition}
\begin{definition}
    Consider an $R$-module $M$. If $m \in M$, then $\textrm{ann}_R(m) = \{ r \in R \mid rm = 0\}$.
\end{definition}
The claim is that $Rm \approxeq R / \textrm{ann}_R(m)$. Example $\C[x]/(x^{12} - 1)$.
\begin{definition}
    A free $R$-module is a direct sum of copies of $R$ as a module over $R$. We will denote this as $R^n = R \oplus \dots \oplus R$.
\end{definition}
So to classify finitely-generated modules, let's split them into free parts. Consider $R$ as a PID and $M$ as an $R$-module,
then define \[M_{\text{tors}} = \{m \in M \mid am = 0 \text{ for some $a \neq 0 \in R$}\}\] to be the torsion submodule of $M$.
One can easily check this is a submodule.

The following is an exact sequence, meaning that the image of each map is the kernel of the one after it.
\[ 0 \to M_{\text{tors}} \to M \to M / M_{\text{tors}} \to 0 \]
We claim that $M / M_{\text{tors}}$ is a free module. Consider $\overline{m} \in M/M_{\text{tors}}$. Then, $r\overline{m} = rm + M_{\text{tors}} \in M/M_{\text{tors}}$,
which after addition shows the claim.