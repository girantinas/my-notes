% !TEX root = lectures.tex
\section{Lecture 23}
\subsection{Complex Numbers are Algebraically Closed}
We shall prove that $\C$ is algebraically closed.
\begin{proof}
    Let $f(x) = x^n + \dots + a_n$ be monic with $a_n \neq 0$ (otherwise $f(0) = 0$ and we'd be done). Let $B \gg |a_1|, \dots, |a_n|$.
    Look at the circle $|x| = B$. Then, the first term dominates the sum
    so $f(x) \sim x^n$ on this circle.
    So as $x$ traverses the circle, $x^n$ traverses the circle with radius $B^n$.
    Now, we make $|x|$ smaller and smaller. The big circle $f(x)$ begins to contracts
    to be centered around $a_n \neq 0$, so in that course, the curve must pass through the point $0$.
\end{proof}
Modulo the Jordan curve theorem, this is a complete proof.
\begin{proof}
    Let $f(x)$ be an odd-degree polynomial. We know it always has a root in $\R$
    (by studying end behavior and the intermediate value theorem).
    Suppose that the complex numbers are not algebraically closed.
    Then $\R \subset \C \subset K$ is a Galois extension over $\R$.
    Call $\gal K/\R = G$ and let
    $H < G$ be a Sylow $2$-subgroup of $G$. Now, look at $\R \subset K^H \subset K$.
    The degree $[K : K^H]$ is the biggest power of two dividing $|G|$ (it's exactly $|H|$). This means that $[K^H : \R]$
    is odd. By the primitive element theorem, $K^H = \R(\alpha)$ for some $\alpha \in K$.
    But then \text{irr}$(\alpha, \R, x)$ is odd degree
    and thus has a root in $\R$, so it can only have degree $1$. Thus, $K^H = \R$.
    Therefore, $|G| = 2^{\ell}$ for some $\ell$. So now there exists some $H_1$
    such that $\C = K^{H_1}$ (by the
    fundamental theorem of Galois groups), where in $|H_1| = 2^{\ell - 1}$.
    Then, there exists $H_1 > H_2$ such that $|H_2| = 2^{\ell - 2}$. This means that
    $[K^{H_2} : \C] = 2$. This means that $K^{H_2} = \C(\alpha)$ where $\alpha^2 = a \in \C$.
    But any square root of a complex number is itself a complex number, so $K^{H_2} = \C$,
    which is a contradiction unless $\ell = 1, K = \C$.
\end{proof}

\subsection{Solvability by Radicals}
Take $f(x) \in k[x]$ irreducible with root $\alpha$,
then \textbf{solvability of $f$ by radicals}
means that $k(\alpha) \subset k(a_1^{1/n_1})(a_2^{1/n_2}) \dots$.
By this statement, we mean $k(\alpha) \subset k(b_1)(b_2)\dots$
and $b_1^{n_1} \in k, b_2^{n_2} \in k(b_1) \dots$.
\begin{theorem}
    Let $\alpha \in k^a$ and $f(x) = \text{irr}(\alpha, k, x)$.
    Then $f$ is solvable by radicals if and only if $\alpha \in K$
    for some $K/k$ Galois and $\gal K/k$ is solvable. (Recall that a group
    is solvable means there's a sequence of subgroups $G > H_1 > H_2 \dots$
    such that each $H_{i + 1}$ is normal over $H_{i}$
    and $H_{i}/H_{i+1}$ is abelian; if make the quotients finer
    they can be taken as cyclic).
\end{theorem}
To build this result, we take a more basic theorem.
\begin{theorem}
    If the characteristic of $k$ doesn't divide $n$ and $k$
    contains the $n$th roots of unity,
    then $K/k$ is cyclic, Galois, of degree $n$
    if and only if $K = k(b)$ such that $b^n \in k$.
    \begin{proof}
        Let $\zeta \in k$ be the primitive $n$th root of unity.
        \begin{itemize}
            \item [($\Rightarrow$)] Suppose $K = k(b)$. Then $b$ is a root of $x^n - a$,
            and the other roots are $\zeta^i b$ for $i = 0$ to $n - 1$. Then $K$ is the splitting
            field of that polynomial and is thus normal. By the characteristic, the extension is separable. Thus, 
            the extension is Galois. Now, let $\sigma \in \gal(K/k)$. Then 
            $\sigma(b) = \zeta^{n_{\sigma}} b$. Then $\sigma \mapsto n_{\sigma}$
            is a map from the Galois group to $\Z/n$. This is an injection, because if $n_{\sigma} = 0$,
            then $b$ is fixed, but it generates the extension, so $\sigma$ must be the identity.
            This means the Galois group is a subgroup of a cyclic group and is thus cyclic.
            \item [($\Leftarrow$)] Suppose $K/k$ is cyclic Galois, e.g. $\gal K/k = \langle \sigma \rangle$.
            We need to find an element $b$ such that $\sigma(b) = \zeta b$, because then $K = k(b)$.
            To do this we need to write $\zeta = \frac{\sigma(b)}{b}$. We prove another theorem to do this.
            \begin{theorem}[Hilbert 90, Zahl Bericht 1897]
                Let $K/k$ be cyclic Galois.
                $\beta \in K$ has $N(\beta) = \prod_{\sigma \in \gal K/k} \sigma(\beta) = 1$
                if and only if $\beta = \frac{\sigma(\theta)}{\theta}$ for some $\theta \in K$.
                \begin{proof}
                    We write 
                    \[ \beta = \gamma + \beta \sigma(\gamma) + \beta \sigma(\beta) \sigma^2(\gamma) + \dots + \beta \sigma(\beta) \dots \sigma^{n - 2}(\beta) \sigma^{n - 1}(\gamma)\]
                    Then
                    \[ \beta \sigma(\beta) = \beta \sigma(\gamma) + \beta \sigma(\beta) \sigma^2(\gamma) + \dots +\gamma\]
                    Thus $\beta \sigma(\beta) = \beta$ (TODO: what?)
                    Then, it's true that the automorphisms of $K$ are linearly independent over $K$ (thus the expression is not $0$ for some $\gamma$).
                    In fact, in generaly, if $M$ is a monoid and there are distinct nonzero maps $\sigma_i$ such that $M \to K^{\times}$ then $\sigma_i$ are linearly independent.
                    If $n = 1$, this is true by definition. Otherwise suppose $\sum_i a_i \sigma_i = 0$ is the shortest linear dependence relation.
                    Then there exists $m \in M$ such that $\sigma_1(m) \neq \sigma_2(m)$ by distinctness
                    and we can write
                    \[ 0 = \sum_i a_i \sigma_1(m) \sigma_i(t) - \sum_i a_i \sigma_i(m) \sigma_i(t), \]
                    which removes the first term.
                    This is a shorter linear dependence relation, so we're done.
                \end{proof}
            \end{theorem}
            Now, note that $\zeta \in k$, so $\sigma(\zeta) = \zeta$
            and thus $N(\zeta) = \zeta^n = 1$, so we apply the theorem. Thus we're done.
        \end{itemize}
    \end{proof}
\end{theorem}

Now we can tackle the main theorem.
\begin{proof}
    If $f$ is solvable by radicals, this means that $\alpha \in K = k(b_1)(b_2)\dots$ such that $b_{i + 1}^{n_{i + 1}} \in k(b_1)\dots(b_i)$.
    Without loss of generality the $k$ contains roots of unity (it adds only a few more radicals
    and is a cyclic Galois extension).
    Then each quotient group (Galois of a sub-extension) is a cyclic group, creating a tower of cyclic quotients in $\gal K/k$.

    Suppose $G = \gal K/k$ is solvable and $\alpha \in K$. Then $K k(\zeta)/K$ 
    Without loss of generality $\zeta \in k$. Then $K \supset K_1 \dots K_n = k$, thus $K_n/K_{n + 1}$ is Galois
    and cyclic, so $K_i = K_{i + 1}(\beta)$, $\zeta \beta = \sigma \beta$, $\beta^n \in K_{i + 1}$
\end{proof}