\subsection{Lecture 14}

We briefly list algorithms used to solve LPs (but not detail them). They include:
\begin{itemize}
    \item Simplex (Dantzig '47), a greedy algorithm. Runs in exponential time worst-case; actually converges quickly in practice.
    \item Ellipsoid (Khachiyan '79), first poly-time algorithm for LP. For $m$ constraints, $n$ variables and $L$ precision, this takes poly(m, n) $\cdot L$ time. Not as usable.
    \item Interior Point (Karmarkar '84) Poly-time, but usable.
\end{itemize}

Maximizing an LP is basically just as hard as finding a feasible $x$ (suppose we could, we could just binary search
to find an optimal point). Furthermore, we can write LPs in the "primal" ($\alpha$) form in the "dual" ($\beta$) form.

\begin{example}
    Suppose we start with the following example:
    \[\min c^T x : Ax \leq b\]
    Let us try to rewrite it in the dual form. First, we write $x = x^+ - x^-$ under the constraint $x^+, x^- \geq 0$.
    Now we have non-negativity. Now how do we get inequalities to turn into equalities? We do the following:
    \[ \sum_{j = 1}^n a_{ij} x_j \leq b_j \]
    We introduce "slack" variables $s_j$ such that:
    \[ \sum_{j = 1}^n a_{ij} x_j + s_j = b_j, s_j \geq 0 \]
    So our entire LP becomes:
    \[ \max -c^T (x^+ - x^-): A(x^+ - x^-) + s = b, x^+, x^-, s \geq 0 \]
    Now let's go the other way. Suppose we start with:
    \[ \max b^T y : A^T y = b, y \geq 0 \]
    Then, note that the condition of equality is just equivalent to two inequality constraints.
    Thus we can write,
    \[ \min -b^Ty : A^Ty \leq b, A^Ty \leq -b, -y \leq 0 \]
\end{example}

\subsubsection{Maximum Flow}
Imagine you have some directed graph of "pipes" which each have a certain "capacity."
There is a single source vertex $s$ and sink vertex $t$ that all the "water" comes from and leaves.
What is the maximum rate of water flow
this network can support from the source to the sink?

The above formulation is termed the $s-t$ formulation of the max flow problem.

The natural algorithm is the following greedy algorithm; Keep finding $s \to t$ paths iteratively (perhaps using DFS)
and push flow amount that saturates the bottleneck (shortest) edge. This unfortunately does not find the correct solution.
It will find a flow, but it might not be optimal.

This is actually a linear program! Our algorithms above are enough to solve it. Consider some flow $f \in \R^{m}$ (assign a number to each edge).
There are certain constraints:
\begin{itemize}
    \item $\forall e \in E, f_e \geq 0$ (Nonnegativity)
    \item $\forall e \in E, f_e \leq u_e$ (Capacity constraint)
    \item $\forall v \in V, \sum_{e = (v, \cdot)} f_e - \sum_{e = (\cdot, v)} f_e$ (Conservation of flow)
    \item We wish to maximize the flows: $\max \sum_{e = (s, \cdot)} f_e = \max |f|$.
\end{itemize}

The first correct algorithm that solved this in better than exponential time was Ford-Fulkerson in the 50s.

\begin{algothm}[Ford-Fulkerson]
    This algorithm runs in pseudo-polynomial. It's polynomial in the max capacity $U$, number of vertices $n$, and number of vertices $m$.

    For all edges $e = (x, y)$, we will pretend $(y, x) \in E$ with $u_{(y, x)} = 0$ (if it isn't present already).
    Furthermore, if there is $f_e$ flow on $e$, we will pretend $-f_e$ flow on the reversed edge (i.e. if we push flow $f_e$, then
    the reversed edge will have capacity 0 + $f_e$). This allows us to "undo" the decision at some point.

    Now, we follow the normal greedy algorithm. Iteratively try to find an $s \to t$ path in the "residual graph" until you can't (this path is called
    an "augmenting path").

    \textbf{Runtime Analysis} In the original paper, they gave an example graph $G$ with $|V| = 10$ and $|E| = 48$ such that the algoritihm
    doesn't terminate. However, it hinged on using irrational numbers as capacities. Let us suppose they are integers from $1$ to $U$;
    we can then make an argument for termination. Note that this means that all the flows are going to always be integral; this means
    any capacity in the residual graph is an integer. This means when we find an augmenting path, we send at least 1 flow from $s$ to $t$;
    this means at most we'll have to do a DFS the max-flow amount of times ($|f^*|$). This yields runtime:
    \[\mathcal{O}((m + n) |f^*|) = \mathcal{O}((m + n)(n - 1)U) = \mathcal{O}(n(m + n)U) \]
    since in the worst case, there are $n - 1$ nodes coming out of $s$ with capacity $U$ each.

    Using BFS instead of DFS leads to so some savings and an even better $\mathcal{O}(m^2n)$. There are more improvements;
    we'll leave those for Wikipedia.

    \textbf{Proof of Correctness} We again start with some prerequisite lemmas.
    \begin{itemize}
        \item Any $s \to t$ flow can be decomposed into $\leq m$ cycles and $s \to t$ paths.
        
        \textbf{Proof} By induction on $m$. The base case (with nonzero flow) is a one-edge graph with $s$ to $t$ ($m = 1$). So it decomposes into a single-edge path from $s \to t$.

        We consider the inductive step where $m > 1$. Consider some flow $f$ from $s$ to $t$. Make a graph $H$
        where each edge exists if there is nonzero flow across that edge. Now let us do a DFS in $H$ from $s$.
        There are a few possible cases.
        \begin{itemize}
            \item If we reach $t$ in this DFS, we found a path from $s \to t$ in this graph. We look at the minimum flow edge along that path,
            and extract that path with that smallest value from $f$. Now, this graph has at least one fewer edge, so by inductive hypothesis,
            this decomposes into $m - 1$ paths and cycles; adding the original path back gives us all $m$ paths and cycles.
            \item If we find a cycle, then we can do the same thing as above; just extract the cycle.
            \item We get stuck at some vertex $v \neq t$, this would mean there is no conservation of flow (since you can't "get out" of this vertex).
        \end{itemize}
        
        \item Suppose $f$ is a non-max flow in $G$. Then $f^* - f$ is a feasible flow in $G_f$.
    \end{itemize}
    These are enough to show the correctness of Ford-Fulkerson (every augmenting path gives a better flow which is feasible).
\end{algothm}
