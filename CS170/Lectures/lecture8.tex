\subsection{Lecture 8}

\subsubsection{Minimum Spanning Trees}
We look at finding the minimum spanning tree in some connected graph $G$. First, we define these notions.

\begin{definition} [Tree]
    A tree is an undirected graph that is connected and acyclic.
\end{definition}

\begin{definition} [Spanning Tree]
    The spanning tree of some undirected graph $G = (V, E)$ is a subgraph of $G$ called $T = (V, E')$ with the same vertices and edges $E' \subseteq E$.
\end{definition}

Then, the notion of a minimum spanning tree is finding a spanning tree whose edge weights added up have minimal cost.

Here is a nice result about trees.
\begin{theorem}
    Consider some graph $T$. Any of the following two implies the third:
    \begin{enumerate}
        \item $|E(T)| = n - 1$.
        \item $T$ is connected.
        \item $T$ is acyclic.
    \end{enumerate}
    As a corollary, any two of these define a tree.
\end{theorem}

Here is a slow algorithm to tackle the problem.
\begin{algothm} [Brute Force MST]
    Take all subgraphs with $n - 1$ edges. Next, check if each is connected (using DFS), if it is connected, then calculate the weight.
    Finally take the minimum of the weights.

    The runtime is roughly:
    \[\mathcal{O}\qty(\binom{m}{n - 1} (n - 1 + n)) \leq \mathcal{O}\qty(\binom{n^2}{n - 1}n) \leq \mathcal{O}(n^n) \]
\end{algothm}

Next, let's discuss a faster algorithm; it's a meta-algorithm that we can use to make other algorithms
by filling in key implementation details.

\begin{algothm}[MST Meta-algorithm]
    \begin{algorithmic}
        \Function{MetaAlg}{$G$}
            \State $X \gets \emptyset$
            \While{$|X| < n - 1$}
                \State Pick $S \subsetneq V$, $S \neq \emptyset$ such that no edge in $X$ cross $(S, V\setminus S)$
                \State Let $e = (a, b)$ be one of the min-weight edges in $G$ crossing the partition
                \State $X \gets X \cup \{e\}$
            \EndWhile
            \State \Return $(V, X)$
        \EndFunction
    \end{algorithmic}

    \textbf{Proof of Correctness}
    We will show at all points in time, there exists an MST $T$ containing $X$. If we show this, this means that at the end of the while loop, since $T$ conatins $X$ and is the same size,
    $X$ becomes $T$. We proceed by induction on $|X|$.

    Base case: $|X| = 0$ works because a connected graph $G$ always has an MST, which trivially contains the empty edge-set.

    Inductive step: Suppose that for $X$ so far, it was contained by $T$. Now, we are adding one more edge $e = (a, b)$ where $a \in S$. There are two cases:

    Case 1: $T$ had the edge $(a, b)$. Then $X$ is still contained by $T$.

    Case 2: $T$ did not have the edge $(a, b)$. Note that since $T$ is a spanning tree, there must be a path from $a$ to $b$, i.e. it must
    pass the partition $V$ and $V \setminus S$ somewhere. If we combine this path with $(a, b)$, we get a cycle. Then, there must be a family of
    edges which all cross the partition; removing any one of them forms a tree. Suppose now we remove any edge $(x, y)$. We define
    \[ T' = T \cup \{(a, b) \} \setminus \{ (x, y) \} \]
    However, the weight of $T' \leq$ weight of $T$ since the weight of $(a, b)$ is at most the weight of $(x, y)$ because by construction it is the lightest edge
    across the partition. This means that $T'$ is also an MST, which also contains $X$, so the claim holds.
\end{algothm}

Now, we are left with a choice: how do we construct our paritioning set $S$?

\begin{algothm}[Prim's MST Algorithm]
    Prim's partitions the set based off a starting vertex $s$.
    \begin{algorithmic}
        \Function{Prim}{$G, s$}
            \State $X \gets \emptyset$
            \State $S \gets \{s\}$
            \For{$n - 1$ times}
                \State $(a, b) \gets \text{min-weight crossing} (S, V\setminus S)$, where $b \notin S$
                \State $S \gets S \cup \{b\}$
                \State $X \gets X \cup \{(a, b)\}$
            \EndFor
            \State \Return $(V, X)$
        \EndFunction
    \end{algorithmic}

    To implement this efficiently, we get the min-weight crossing by using a heap. We greedily take the cheapest edge that's
    not yet in our tree; this is precisely that min-weight crossing. Every time we add something new to $S$, all we should do it scan the edges and
    update it's distance from our tree accordingly. The implementation is ommitted here.

    \textbf{Runtime Analysis} The runtime becomes the exact same as Dijkstra's algorithm, since we update distances and pop things off the heap
    the same way (except updating distances is the distance to the tree, instead of doing an addition), so with a binary heap,
    the runtime is the same $\mathcal{O}((m + n) \log n) = \mathcal{O}(m \log n)$ since $m \geq n - 1$ since the graph is connected.
\end{algothm}

To present another approach, we first discuss a new data structure.
\begin{note} [Union-Find Data Structure]
    Suppose we want to maintain a partition of $\{1, \dots, n\}$ subjects to 2 operations.
    \begin{enumerate}
        \item find$(a)$ returns the name of the partition that $a$ is in.
        \item union$(a, b)$ merge the partitions containing $a$ and $b$.
    \end{enumerate}
\end{note}

Now we discuss an even more greedy approach.

\begin{algothm}[Kruskal's Algorithm]
    Kruskal's algorithm uses a global approach.
    \begin{algorithmic}
        \Function{Kruskal}{$G$}
            \State Sort $E$ in increasing order of weight.
            \State $X \gets \emptyset$
            \State Initialize a UnionFind data structure.
            \For{$(a, b) \in E$ (in increasing order of weight)}
                \If{find$(a) \neq$ Find$(b)$}
                    \State $X \gets X \cup \{e\}$
                    \State union$(a, b)$
                \EndIf
            \EndFor
            \State \Return $(V, X)$
        \EndFunction
    \end{algorithmic}
\end{algothm}