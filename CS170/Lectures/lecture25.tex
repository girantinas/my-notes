\section{Lower Bounds and Alternate Models of Computation}
\subsection{Lecture 25}
\subsubsection{Lower Bounds}

So far, we have tried to minimize computational resources to solve various problems.
In particular, we have cared about upper bounds; given some problem $P$, we can define some function $f_P(n)$
as the minimum number of resources (number of timesteps, number of bits of memory, etc) to solve $P$ on worst-case inputs of size $n$.

Thus, we can think of an algorithm as an upper bound of $f_P(n)$, i.e. there might exist a more efficient algorithm,
but the theoretically $f_P(n)$ must be at least the runtime/memory of that algorithm. How can we prove a lower bound on $f_P(n)$?

If the measure of complexity is the number of comparisons, we know that $f_{\text{sorting}}(n) = \Omega(n \log n)$
and $f_{\text{sorting}}(n) = \mathcal{O}(n \log n)$ because merge-sort exists. Thus, we have a very tight bound.

However, for non-comparison based algorithms, we want general time lower bounds. Turns out, this is pretty hard. For all we know, all NP-complete problems we mentioned prior may be able to be solved in linear time.
There are a few problems where $\omega(n)$ complexity are known through the time hierarchy theorem (which we will not discuss in depth). Here's what such a problem looks like.

\begin{example}
    Take some problem that looks like the following.
    \begin{itemize}
        \item Input is the source code to a program $P$ and a string $x$.
        \item If we run $P$ on input $x$, would it terminate in at most $|x|^3$ steps?
    \end{itemize}
    Clearly this takes $\Theta(|x|^3)$ on worst-case input if we were to just simulate it (where the program doesn't terminate at all; we cannot know this apriori due to the halting problem).
    It turns out due to this "time hierarchy theorem," it turns out we can put a lower bound on it that looks like $\Omega(n^{3 - \epsilon})$ steps.
\end{example}

It turns out it's much easier to prove lower bounds for alternate models of computation. Here are a few we will study.

\begin{itemize}
    \item Comparison Model - Sorting example discussed above
    \item Circuit Model
    \item Cell Probe Model
    \item Communication Model
\end{itemize}

\subsubsection{Circuit Model}
Suppose one has a circuit of $n$ input bits, with AND, OR, NOT gates. Consider the problem of integer multiplication. Suppose we have a collection $\{M_{2n}\}_{n = 1}^{\infty}$
which are functions that multiply two bit strings of size $n$ (i.e. two $n$-bit numbers).
\[ M_{2n} : \{0, 1\}^n \times \{0, 1\}^n \to \{0, 1\}^{2n} \]
The question is designing a family of Boolean circuits $\{C_{2n}\}$ each solving $M_{2n}$ so as to minimize some notion of complexity.

We mentioned prior that one can always write a circuit that computes the output of an algorithm where the size of the circuit is proportional to the runtime of the algorithm.
Thus, a trick to prove runtime lower bounds on algorithms would be to prove circuit size lower bounds.

Suppose we measure size $s$ of a circuit by the number of wires. It turns out that
\begin{theorem}
    The number of circuits of size $s$ is $2^{\mathcal{O}(s \log s)}$.
\end{theorem}

However, the number of functions on $n$ bits mapping to an output bit is is exactly $2^{2^n}$ ($2^n$ choices for each input, 2 choices for each output).
Thus if $s$ is polynomial in $n$, this means that $2^{\mathcal{O}(s \log s)}$ is very small to $2^{2^n}$ that almost every function cannot be computed by poly-sized circuits.
We could try to find one of these an exponential lower bound for the circuit size of certain problems.
However, coming up with a problem that takes exponential size to solve IS REALLY hard.

\begin{note}
    The highest known lower bound for any explicit function is size $\geq (3 + \frac{1}{86}) n$.
\end{note}

If we restrict depth, this becomes a little bit better.

\begin{example}
    Consider the parity problem, where we want to find the parity of some bitstring (the xor of all its bits); it turns out any $\mathcal{O}(1)$ depth
    circuit family for PARITY requires size $\geq \exp(\Omega(n))$.
\end{example}

\subsubsection{Cell Probe Model}

We analyze the cell probe model (for data structures). It looks like the following. Suppose you have some memory,
where each word in the memory takes up $w$ bits and a finite memory of size $S$. Now, suppose a data structure algorithm receives a query call;
it's going to read/write some of the cells of memory somehow. In a read call, we read all $w$ bits of some block; in a write call, we write all $w$ bits of some block.
After some finite amount of rounds of communication between the algorithm and memory, the query call will be resolved. Thus, we measure the cost of the computation, the query time,
(in lower bound) the number of rounds of communication.

It turns out there is a lot more headway made in this area.

\begin{example}
    \begin{itemize}
        \item We can use the cell probe model to prove that amortized complexity for union find is $\Omega(\alpha(m,n))$; meaning that the disjoint forest data structure is optimal (since it runs in $\mathcal{O}(\alpha(m,n))$).
        \item Consider the dynamic partial sum problem. We have $n$ items, where a query$(j) = \sum_{i = 1}^{j} x_i$ and update$(j, \delta)$ does $x_j \gets \delta$. It turns out $\min(t_{update}, t_{query}) = \Omega(\log n)$,
        but note that we can solve this with a balance binary search tree (store the sum of a node's subtree in the node); this already has $\mathcal{O}(\log n)$ updates and queries.
    \end{itemize}
\end{example}

How were these bounds proven? They generally come from communication lower bounds (like information theory).

\subsubsection{Communication Model}
Suppose there are two entities Alice and Bob who want to together compute $f(x, y)$, where Alice gets input $x \in X$ and Bob gets input $y \in Y$.
They send messages $m_1, m_2, \dots$ until one of them figures out the answer.

Some goals are to minimize the number of rounds as well as minimize the total bits communicated. There are also variants to the problem.

\begin{itemize}
    \item Deterministic
    \item Randomized
    \begin{itemize}
        \item Public coin (both know the same random string, without having to communicate it)
        \item Private coin (each has their own random string, unknownst to the other)
    \end{itemize}
\end{itemize}

\begin{example}
    Consider the EQ (Equality) problem. $X = Y = \{0, 1\}^n$ and we wish to determine whether $x$ and $y$ are equal bit strings. We consider the amount of bits of communication one needs to use.
    \begin{itemize}
        \item In the deterministic setting, it turns out you need at least $n$ communication, so the problem is $\Theta(n)$.
        \item In the randomized public coin settings, it turns out you have $\Theta(1)$ bits. Thus, we
        \item In the randomized private coin setting, it turns out you have $\Theta(\log n)$. Alice will look at the first $n^2$ primes and choose a random prime $p$ amongst those
        and send it to Bob. Then, we compute $x \mod p$ and set that. Then, Bob can compute $y \mod p$, and return yes if $(x - y)$ is 0. Since there are about $\frac{\log x}{\log \log x}$ prime factors of $x$
        the chance that $x$ and $y$ share the prime factor $p$ is very small. 
    \end{itemize}
\end{example}

\subsubsection{Lower Bounds for Streaming}
It turns out one can prove memory lower bouhds for streaming algorithms via communication complexity lower bounds.

\begin{theorem}[Lower Bound for Unique Counting]
    Any space-$S$ deterministic streaming algorithm for exact unique counting means there exists a deterministic communication protocol for EQ with total communication $\leq S$.

    \begin{proof*}
        Suppose such an algorithm $\mathcal{A}$ exists. Then we can have the following protocol to solve EQ. Alice creates a stream containing all $i$ such that $X_i = 1$.
        Then, Alice runs $\mathcal{A}$ on the stream. Then, Alice sends all $S$ bits of memory used by the algorithm to Bob. By the correctness of $\mathcal{A}$, Bob must be able to use $\mathcal{A}$ to query
        the memory and find the number of 1s in Alice's input, suppose this number is $t$. Bob streams an index of 1 into $\mathcal{A}$. If the answer is $t+1$, then Alice's bit at position 1 is 0; otherwise it is a 1.
        Repeating this over all indices allows Bob to reconstruct Alice's entire input. Then, Bob can simply check bit-by-bit if this matches his bit-string. This protocol only used $S$ bits of communication, proving the claim.
    \end{proof*}
\end{theorem}

But any communication algorithm needs at least $n$ bits to be communicated; thus any exact deterministic streaming algorithm for unique counting must use at least $n$ bits of memory (hash maps are optimal).

Here are two other results that arise. It's possible to show:

\begin{theorem}[Relaxing Unique Counting]
    \begin{itemize}
        \item Any approximate deterministic algorithm for unique counting requires $\Omega(n)$ bits of memory.
        \item Any exact randomized algorithm for unique counting requires $\Omega(n)$ bits of memory.
    \end{itemize}
\end{theorem}

Thus, the only way to get an efficient solution for this problem is using a randomized algorithm and approximation.