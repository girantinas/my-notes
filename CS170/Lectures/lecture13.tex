\subsection{Lecture 13}
\subsubsection{Knapsack Problem}
We consider the knapsack problem. Given as input an array $A[1 \dots n]$, of items.
Each item is a pair $(w, v)$ (weight and value). We also have a knapsack (bag) that can hold at most $W$ weight.

\begin{example}
There is a natural greedy approach to solving this.
\begin{align*}
    A &= [(10.001, 10.002), (10, 10), (10, 10)] \\
    W &= 20
\end{align*}
We greedily take the most "bang for your buck," biggest ratio of value to weight.
Greedy will take $(10.001, 10.002)$, while optimal is $[(10, 10), (10, 10)]$.

You can prove that you get at least half of the optimal solution with the greedy approach.
\end{example}


\begin{algothm} [Knapsack DP]
    First, we again think about recursive brute force, and then form a recurrence relation. Define $f(i, C)$ is
    the maximum value that you can make with the last $i$ items and capacity $C$. We want $f(1, W)$.
    The recurrence relation is:
    \[ f(i, C) = \begin{cases}
        0 & i = n + 1 \\
        f(i + 1, C) & w_i > C \\
        \max\{f(i + 1, C), v_i + f(i + 1, C - w_i)\} & \text{otherwise} \\
    \end{cases}\]

    Note that $i$ ranges from 1 to $n$ and $C$ ranges from $0$ to $W$, furthermore we only need the answers from $i + 1$ to get $i$. Thus,
    runtime is $\mathcal{O}(nW)$, memory is $\mathcal{O}(W)$ bottom-up.
\end{algothm}

This is pseudo-polynomial time. Why is that? Call $b$ the amount of bits in the input of your algorithm.
We define an algorithm that runs in polynomial time as one where the runtime is a polynomial in terms of $b$.

Note that the size of the input is $\Theta(n \log W)$ in bits (since every weight
is at most $W$). This means that $nW$ is polynomial in $n$ and exponential in $\log W$!

\subsubsection{Traveling Salesman Problem}
Now consider the following problem. There are $n$ locations with distances $D[i][j]$ = distance from $i$ to $j$,
which form a metric space (nonnegative and satisfy triangle inequality). We want to traverse this complete graph, visiting all verteices, starting
at 1 while minimizing total travel.

The naive brute force way is to try every ordering of visiting the vertices, this is $\mathcal{O}(n!)$.

\begin{algothm}[TSP]
    Define $f(i, S)$ as the min travel time to visit all locations in $S \subseteq \{1, 2, \dots, n\}$ when starting at $1$.
    We want $f(1, \{2, 3, \dots, n\})$.
    \[ f(i, S) = \begin{cases} 0 & S = \emptyset \\
        \min_{j \in S} f\qty(j, S \setminus \{j\} + D[i][j]) & \text{otherwise} \end{cases} \]
    \textbf{Runtime Analysis} We must calculate for $\mathcal{O}(n 2^n)$ entries in each set, with each maximum taking at most linear time, giving us:
    $\mathcal{O}(n^2 2^n)$.
    \textbf{Memory} Top-down memoization is $\mathcal{O}(n 2^n)$, but with bottom-up, it's
    something like
    \[ \max_{1 \leq k \leq n} n \binom{n}{k} = n \binom{n}{n/2} = \mathcal{O}(\sqrt{n} 2^n) \]
\end{algothm}

The example of chain matrix multiplication was also discussed, but I have ommitted it here; DPV does a pretty good job of explaining it.

\section{Linear Programming}
\subsection{Lecture 13, Continued}
\subsubsection{Linear Programs}
Linear programming is maximizing or minimizing some linear function subject to some linear inequality constraints.
In the language of linear algebra this is:

\begin{definition}[Linear Program]
    A linear program is an optimization problem which can be framed as:
    \[ \max_{x} c^T x : Ax \leq b \]
    where the $A, b, c$ are given, and the inequalities are elementwise.
\end{definition}

This may seem abstract, so let us give an example of a linear program.
\begin{example}
    I run a factory that sells foo for \$4/oz and bar for \$5/oz. In my warehouse, I have
    \begin{itemize}
        \item W = 10 oz water
        \item B = 6 oz butter
        \item G = 6 oz goo
    \end{itemize}
    Also, to make our products, we need to following ingredients:
    \begin{itemize}
        \item In order to make 1 oz of foo, it takes 2 oz water, 1 oz butter, 4 oz goo.
        \item In order to make 1 oz of bar, it takes 3 oz butter, 2 oz goo.
    \end{itemize}
    Call $x$ the oz of foo I make, and call $y$ the oz of bar I make.
    The objective is thus:
    \[ \max_{x, y} 4x + 5y \]
    subject to:
    \[ 2x + 0y \leq 10, 1x + 3y \leq 6, 4x + 2y \leq 6, -x \leq 0, -y \leq 0 \]
    Then define:
    \[ c = \begin{bmatrix}
        4 \\ 5
    \end{bmatrix}, b = \begin{bmatrix}
        10 \\ 6 \\ 6 \\ 0 \\ 0
    \end{bmatrix},
    A = \begin{bmatrix}
        2 & 0 \\ 1 & 3 \\ 4 & 2 \\ -1 & 0 \\ 0 & -1
    \end{bmatrix}\]
    Note that this satisfies the definition of LP we gave above.
\end{example}

We briefly discuss convexity in the context of LP, as well as some other properties.

\begin{definition}[Feasibility]
    A linear program is feasible if there is at least one vector $x$ that satisfies all the constraints.
\end{definition}

\begin{definition}[Bounded]
    A linear program is bounded if its solution is not $+ \infty$ or $- \infty$.
\end{definition}

\begin{definition}[Convexity]
    A region is convex if for any two points in the region, the line between them is also in the region.
\end{definition}

\begin{theorem}
    The feasible region in any LP is always convex. Furthermore, the solution to an LP always lies
    on a vertex of this convex polytope.
\end{theorem}

Furthermore, note that all of these constraints are "halfspaces" i.e. they are on one side of a bunch of lines. This defines
a convex polygon (or polytope in higher dimensional space) that we have to satisfy (a feasible region). We have to maximize in the $c$ direction. One
of the optimal solutions of this is always a vertex.

Even though the amount of vertices may be exponential, there exist polynomial-algorithms to solve LPs. We are not too
concerned about their details or implementations; we are recommended to take 270. Instead, we just need to understand how to formulate
our problems as LPs.

We briefly discuss duality. Remember in our example (without the nonnegative constraints) we had the following equations.
\begin{align*}
    2x + 0y &\leq 10 \\
    1x + 3y &\leq 6 \\ 
    4x + 2y &\leq 6
\end{align*}
Suppose I wanted to prove to you
that the maximum amount you could make was no more than 5 dollars. Multiply all the equations by some multipliers $z_1, z_2, z_3 \geq 0$ and add them:
\begin{align*}
    (2z_1 + z_2 + 4z_3)x + (3z_2 + 2z_3)y \leq z^T b
\end{align*}
What if $2z_1 + z_2 + 4z_3 = 4$ and $(3z_2 + 2z_3) = 5$. Then we have recovered our original $c$.
\begin{align*}
    c^x \leq z^T b
\end{align*}
this is the dual linear program (upper bound on our revenue).

\begin{definition} [Duality]
    Suppose our original "Primal" LP was:
    \[ p^* = \max_{x} c^T x : Ax \leq b \]
    the dual LP is:
    \[ q^* = \min_{y} b^T y : A^Ty = c, y \geq 0 \]
\end{definition}

Note that each $y$ is an upper bound to $c^T x$;
the dual tries to find the "tightest upper bound."

\begin{theorem}[Weak/Strong Duality]
    Weak duality tells us $p^* \leq q^*$.

    Strong duality tells us that if the primal LP is feasible and bounded, so is the dual and
    $p^* = q^*$. Furthermore, if the primal is unbounded, then the dual is infeasible.
\end{theorem}

Also, the dual of the dual is the primal.