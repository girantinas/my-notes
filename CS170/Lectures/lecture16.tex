\subsection{Lecture 16}
\subsubsection{Games}
\begin{definition}[Game]
    A game involves $p > 2$ players taking an action (in secret) from a set of possible actions.
    Once all the actions are taken, an outcome is revealed; each player gains some utility (score)
    for each possible outcome.
\end{definition}

We first note the most well-studied game: the Prisoner's Dilemma.

\begin{example}
    There are two apprehended suspects who the police are sure committed a crime and have a lesser charge,
    but they need a confession to get a greater charge.
    The police give the following deal to the suspects:
    \begin{itemize}
        \item If you stay silent: you get 1 year in prison.
        \item You tell on your partner: you get 0 years in prison, your partner gets 3 years.
        \item If both partners snitch: both get 2 years.
    \end{itemize}

    \begin{tabular}{c | c | c |}
        \empty & \text{snitch} & \text{silent} \\ \hline
        \text{snitch} & (-2, -2) & (0, -3) \\ \hline
        \text{silent} & (-3, 0) & (-1, -1) \\ \hline
    \end{tabular}

    Thus, any rational prisoner will tell on their partner, since they only gain from their decision (hold the row constant; moving left is always better for the column player.)

    We can think of this as an input to some algorithm, which might spit out the optimal strategy for the people.
\end{example}

\begin{definition}[Zero-Sum Games]
    A zero-sum game is a game where the payoffs for the 2 players sum to 0 for any outcome.
\end{definition}

A good example of a zero-sum game is rock, paper, scissors. We can tabulate here in the following "payoff matrix."
\begin{tabular}{ c | c | c | c |}
    \empty & \text{rock} & \text{paper} & \text{scissors} \\ \hline
    \text{rock} & (0, 0) & (-1, 1) & (1, -1) \\ \hline
    \text{paper} & (1, -1) & (0, 0) & (-1, 1) \\ \hline
    \text{scissors} & (-1, 1) & (1, -1) & (0, 0)  \\ \hline
\end{tabular}

We can just write this as the 1st number then, since all cells sum to 0:

\begin{tabular}{ | c | c | c |} \hline
    0 & -1 & 1 \\ \hline
    1 & 0 & -1 \\ \hline
    -1 & 1 & 0  \\ \hline
\end{tabular}

The row player thus wants to maximize the score and the column player wishes to minimize the score.

Now we discuss strategies of playing the game.
\begin{definition}[Strategies]
    There are two main types of strategies.
    \begin{enumerate}
        \item Pure Strategies: Always picking some particular action.
        \item Mixed Strategies: Some probability distribution over pure strategies.
    \end{enumerate}
\end{definition}

Returning back to rock-paper-scissors, suppose player 1 plays mixed strategy:
\[ x = (1/3, 1/3, 1/3) \]
Then suppose player 2 plays some other mixed strategy:
\[ y = (y_1, y_2, y_3) \]
Then, the expected payoff (using the reduced payoff matrix $U$ with one entry) is:
\[ \E{\text{payoff}} = \sum_{j = 1}^3 \sum_{i = 1}^3 \frac{1}{3} U(i, j) y_j = \sum_{j = 1}^3 y_j (-1/3 + 1/3 + 0) = 0 \]

Thus, we have the following.

\begin{definition}
    \begin{itemize}
        \item There exists a strategy $x$ for the row player which guarantees $\E{\text{payoff}} \geq 0$.
        \item Similarly, there exists a strategy $y$ for the column player which guarantees $\E{\text{payoff}} \leq 0$.
    \end{itemize}

    Thus, we say that the pair of strategies is in equilibrium. This means the value of this game is $0$ (the value that can be guaranteed by both parties).
\end{definition}

Let us look at more examples:

\begin{example}
    \begin{tabular}{c|c|c|}
        \empty & L & R \\ \hline
        T & 5 & -3 \\ \hline
        B & -1 & 1 \\ \hline
    \end{tabular}

    Let us try to find the equilibrium. What if the row player tries mixed strategy $x = (1/2, 1/2)$? Let's try pure strategies for the column player:
    \begin{itemize}
        \item L means the expected payoff is: \[ 5\frac{1}{2} - 1 \frac{1}{2} = 2 \]
        \item R means the expected payoff is: \[ -3\frac{1}{2} + 1 \frac{1}{2} = -1 \]
    \end{itemize}
    Now, lets have the column player try mixed strategy $y = (1/2, 1/2)$. Let's again try pure strategies:
    \begin{itemize}
        \item T means the expected payoff is: \[ 5\frac{1}{2} - 3 \frac{1}{2} = 1 \]
        \item B means the expected payoff is: \[ -1\frac{1}{2} + 1\frac{1}{2} = 0 \]
    \end{itemize}
    Thus, these strategies are not in equilibrium, because -1 is the minimization of the expected payoff (what the column player seeks to do)
    and 1 is the maximization of the expected payoff (what the row player seeks to do).

    Here is a better strategy: $x = (1/5, 4/5)$. Pure strats for column player:
    \begin{itemize}
        \item L means the expected payoff is \[5 \frac{1}{5} - 1 \frac{4}{5} = \frac{1}{5} \]
        \item R means the expected payoff is \[ -3\frac{1}{5} + 1 \frac{5}{5} = \frac{1}{5} \]
    \end{itemize}
    $y = (2/5, 3/5)$ also guarantees at most $\frac{1}{5}$ payoff.

    How can we find this magical pair of strategies that cause a payoff? Let us be more general.

    If $x = (x_1, x_2)$, the strategy L gives payoff $5x_1 - 1x_2$ and R gives payoff $-3x_1 + 1x_2$. Thus, we want to find:
    \begin{align*}
        &\max_x \min\{5x_1 - x_2, -3x_1 + x_2\} : x_1 + x_2 = 1 \\
        &= \max_{x, z} z : z \leq 5x_1 - x_2, z \leq -3x_1 + x_2, x_1 + x_2 = 1, x_1, x_2 \geq 0
    \end{align*}
    Where we introduce slack variable $z$, which will be the min at optimum.

    Note that we could write a similar LP for the column player. Get $y = (y_1, y_2)$, the strategy T gives payoff $5y_1 - 3y_2$ and
    the strategy B gives payoff $-y_1 + y_2$. Thus we want to find:
    \begin{align*}
        &\min_y \max\{5y_1 - 3y_2, -y_1 + y_2\} \\
        &= \min_{y, w} w: w \geq 5y_1 - 3y_2, w \geq -y_1 + y_2, y_1 + y_2 = 1, y_1, y_2 \geq 0
    \end{align*}

    Taking Jelani's word for it, these two LPs are duals! This means that they have the same optimum; thus they both yield the equilibrium strategy.
\end{example}

We talk about some more duality facts that can help us see this connection:
\begin{itemize}
    \item $x_i \geq 0$ in primal; the corresponding constraint with any $x_i$ in dual is inequality (instead of $=$)
    \item $Ax = b$ in primal; the corresponding dual variable does not have a nonnegativity constraint.
\end{itemize}

There are a few more unanswered questions. Firstly, let's generalize what we've seen in the specific game example.

Suppose our utility matrix is $U \in \R^{m \times n}$

The row player seeks to once again:
\[ \max_x \min_y \sum_{j = 1}^n\qty(\sum_{i = 1}^m x_i U(i, j)) y_j \]
note that the RHS is also $y^T U x$. Note that this is just a weighted sum of $y_j$, where the $y_j$'s all sum to 1. To minimize,
we just put all the probability on the smallest column. However, this is just a pure strategy. Thus, we have shown that the best response to
a fixed mixed strategy is a pure strategy.

This is also just an LP, so we can find them in polynomial time via linear programming. By duality, we have:

\begin{theorem}[Minimax Theorem]
    For any zero-sum game $U$, we have
    \[ \max_x \min_y \sum_{j = 1}^n\qty(\sum_{i = 1}^m x_i U(i, j)) y_j = \min_y \max_x \sum_{i = 1}^m\qty(\sum_{j = 1}^n y_j U(i, j)) x_i \]
\end{theorem}
