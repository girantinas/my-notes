\subsection{Lecture 12}

The first worry is that our dynamic programming examples
often return only a cost, instead of the optimal "solution" (i.e. maybe the edges in 
a path instead of just a length). However, we argue it is easy to modify dynamic programming algorithms to recover this.

Thinking about the dependency DAG of the subproblems, we can see how to do this. We simply find the shortest/longest path in a DAG,
which we already solved! To find the path explicitly, we can store the 'argmin' of the edge that minimized the shortest path, in
a new lookup table ('prev' or 'choices').

\subsubsection{Bellman-Ford}
We once again revisit the problem of single-source shortest paths on directed graphs.
We already know Dijkstra's algorithm; however, this algorithm does not work if some of the edge weights are negative.
Furthermore, if there is a negative cycle in your graph, then the shortest path to certain nodes may be $\infty$ (keep
following the cycle over and over to get an arbitrarily small length). Let us look at an algorithm which can find
the shortest paths if they are well defined (even if edge weights are negative) or find the existence of a negative cycle.

We first realize the following recursive thinking. Any shortest path from source $s$ to vertex $t$ has a last edge $v$ on
the path. However, the last recurrence relation will not work because of the cycles possible in the graph.
Furthermore, any shortest path to $t$, if it exists, has at most $|V| - 1 = n - 1$ edges (otherwise there is a cycle in that path we can remove which is not
negative). Thus, this is what we develop at first.

\begin{algothm} [Bellman-Ford (DP)]
    Define $f(t, k)$ as the shortest path from $s$ to $t$ using $\leq k$ edges. We want to find $f(t, n - 1)$. Let us develop a recurrence:
    \[ f(t, k) = \begin{cases}
        \infty & k = 0, t \neq s \\
        0 & k = 1, t = s \\
        \min\{f(t, k - 1), \min_{(v, t) \in E}\{w(v, t) + f(v, k - 1)\} & \text{otherwise}
    \end{cases} \]

    \begin{algorithmic}
        \Function{BF}{$G,s$}
            \State $T[1 \dots n][0\dots 1] \gets \infty$ for all values
            \State $T[s][0] \gets 0$
            \For{$k = 1$ to $n - 1$}
                \For{$t = 1$ to $n$}
                    $T[t][1] \gets T[t][0]$
                    \For{$(v,t) \in E$}
                        \State $T[t][1] \gets \min(T[t][1], w(v, t) + T[v][0])$
                    \EndFor
                \EndFor
                \For{$t = 1$ to $n - 1$}
                    \State $T[t][0] \gets T[t][1]$
                \EndFor
            \EndFor
            \State \Return $T[1 \dots n][1]$
        \EndFunction
    \end{algorithmic}

    \textbf{Memory Analysis} The memory table is $\mathcal{O}(n^2)$ size because $t$ can range from $1, n$ as can the path length. However,
    note that $f(\cdot, k)$ only depends on $f(\cdot, k - 1)$ values, so we need to only have to remember a 1-d array of $f(\cdot, k)$ values. So,
    we can reduce that to $\mathcal{O}(n)$ with bottom-up DP.

    \textbf{Runtime Analysis} The runtime is $n$ times total time to compute all $f(\cdot, k)$. This is:
    \[ n \sum_{t \in V} C \cdot (1 + \text{indeg}(t)) = \mathcal{O}(n^2 + mn) \]
\end{algothm}

However, this isn't the best we can do. Instead, the actual Bellman Ford runs yet faster on sparse graphs (in a textbook).

\begin{algothm}[Bellman-Ford]
    \begin{algorithmic}
    \Function{BFBook}($G, s$)
        \State $T[1 \dots n] \gets \infty$ for all values
        \State $T[s] \gets 0$
        \For{$k = 1$ to $n - 1$}
            \For{$e = (u, v) \in E$}
                \State $T[v] \gets \min(T[v], T[u] + w(u, v))$
            \EndFor
        \EndFor
        \State \Return $T$
    \EndFunction
    \end{algorithmic}

    \textbf{Runtime Analysis} Outer loop runs in $n$, inner in $m$; this gives us $\mathcal{O}(mn)$.

    Why are these equivalent? Well, first, note that this is just a different ordering of filling in the DP table $T$.
    This is because instead of using a specific node every time, we still look at all edges. Now, the question is, why is our replacement
    overwriting our DP table instead of making a second column? First we make the following claims (easy to check with induction):

    \begin{itemize}
        \item At all points in time, for all $v$, $T[v]$ is the length of some path from $s \to v$ (counting the empty path as a path with length $\infty$).
        \item For all $k$ and $v$, after going through the outer loop $k$ times, $T[v] \leq f(v, k)$.
    \end{itemize}

    Thus this means after all the iterations, $\delta(v) \leq T[v]$ (because all paths are at least the length of the shortest path), also $T[v] \leq f(v, n - 1) = \delta(v)$
    by definition. Therefore, we must have $T[v] = \delta(v)$, as we claim.
\end{algothm}

Now, we have our finished algorithm; where is the negative cycle detection we've been promised. Assume WLOG everyone is reachable from $s$. We show the following:

\begin{theorem}
    There exists a negative cycle if and only if $f(v, n) < f(v, n - 1)$ for some $v$ in the recurrence we defined above.

    \begin{proof*}
        The if condition is straightforward. Suppose that there was some vertex $v$ where $f(v, n) < f(v, n- 1)$ but there was no negative cycle.
        Because any path with at least $n$ edges has a cycle,
        this means that we could splice out the cycle (which must be positive by assumption) and
        make a path with $n - 1$ or less edges, which would mean $f(v, n) \geq f(v, n - 1)$,
        which is a contradiction.

        The only if is a bit trickier. We proceed by contrapositive. Suppose that for all $v$, $f(v, n) \geq f(v, n - 1)$.
        Now, consider some cycle $C = v_1 \to v_2 \to \dots \to v_r \to v_1$. Well, we know that for all $i$,
        \begin{align*}
            f(v_{i + 1}, n - 1) \leq f(v_{i + 1}, n) &\leq f(v_1, n - 1) + w(v_i, v_{i + 1}) \\
            \sum_{i = 1}^r f(v_{i + 1}, n- 1) - f(v_{i + 1}, n) &\leq \sum_{i = 1}^r w(v_i, v_{i + 1}) \\
            0 &\leq \sum_{i = 1}^r w(v_i, v_{i + 1})
        \end{align*}
        Thus the sum of the weights of $C$ is non-negative, meaning we cannot have a negative cycle, which is exactly the contrapositive
        of our only if claim.
    \end{proof*}
\end{theorem}

\subsubsection{Floyd-Warshall Algorithm}
Now we consider the problem of all pairs shortest paths. As input, we are given directed graph $G$. We should return the 2d array
$d$ such that for all $i,j$, $d[i][j] =$ total length of the shortest $i \to j$ path.

One way to do this would be to run Bellman-Ford $n$ times, once for each vertex. This gives us a total runtime of
$\mathcal{O}(n^2 m) = \mathcal{O}(n^4)$ if the graph is dense. We can do better. Assume without loss of generality that $G$ is the complete graph (if $e \notin E$, just 
pretend $w(e) = \infty$). We do the following:

\begin{algothm}[Floyd-Warshall Algorithm]
    Our recurrence relation is as follows. $f(i, j, k)$ is the length of the shortest path from $i$ to $j$, only using vertices from $\{1, \dots, k\}$. We want to find $f(\cdot, \cdot, n)$.
    \[ f(i, j, k) = \begin{cases}
        w(i, j) & k = 0, i \neq j \\
        0 & k = 0, i = j \\
        \min\{f(i, j, k - 1), f(i, k, k - 1) + f(k, j, k - 1)\} & \text{otherwise}
    \end{cases} \]
    The algorithm is ommitted for brevity. We can run a similar set of optimizations we did for Bellman-Ford.

    \textbf{Memory Analysis}
    The original memory is $\mathcal{O}(n^3)$, which can be made $\mathcal{O}(n^2)$ using bottom-up DP.

    \textbf{Runtime Analysis} We triple for loop over $k, i, j$ and then update the table in constant time every time. $\mathcal{O}(n^3)$ runtime.
\end{algothm}

\subsubsection{More DP Examples}
We consider the problem of longest increasing subsequence. We are given an input array $A$ of length $n$. We want to find the longest
subsequence (that is, non-contiguous elements in the same order as in the original array) that is increasing.

We again have to figure out a recurrence relation. At each index $i$, we have the choice of either taking that element into our
subsequence or not. Furthermore, sometimes we cannot; in particular, if the last element of the subsequence is too large, taking $i$
may cause it to stop increasing. Thus, we have the following recurrence relation, letting $f(i, last)$ be the length of the longest increasing
subsequence of $A[i \dots n]$ such that all values we use are $\leq A[last]$. We want to find:
\[ \max_{1 \leq i \leq n} \{ f(i, i + 1) + 1 \} \]
Instead, we can use a sentinel trick to make this a bit easier. Prepend $- \infty$ to $A$. Now,
$f(1, 2)$ is what we're solving for.

Now, for the actual recurrence relation:
\[ f(last, i) = \begin{cases}
    0 & i = n + 1 \\
    f(last, i + 1) & A[i] \leq A[last] \\
    \max\{f(last, i + 1), f(i, i + 1) \} + 1 & \text{otherwise}
\end{cases} \]
which can be done in $\mathcal{O}(n)$ with bottom-up dp, with runtime $\mathcal{O}(n^2)$.