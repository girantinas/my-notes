\subsection{Lecture 6}

\subsubsection{Topological Sort}

We consider the problem of topological sort. We take as input a directed acyclic graph (DAG).
We seek to find an ordering of $V$ such that $(u, v) \in E \implies u$ must come before $v$ in the ordering. Note that such an ordering of the graph
is not necessarily unique. Intuitively, it represents the order that dependencies must be filled in order to solve a problem.

%% TODO Insert example of Topo sort

\begin{algothm} ["Brute-force" Search]
    Try all permutations of the vertices. Return the first one that is a topological sort.

    The runtime is $\mathcal{O}{n!mn}$, because for each of $n!$ permutations, we have to check $m$ edges that the edge is respected, and finding each endpoint
    of the edge needs to be found in the sorted order.
\end{algothm}

To find a faster algorithm, let us define some terms.

\begin{definition}[Source, Sink]
    In a directed acyclic graph, a source is a vertex with no incoming edges. A sink is a vertex with no outgoing edges.
    Note that every DAG has a source and a sink.
\end{definition}

\begin{algothm}[Source Peel-Off]
    We can iteratively "peel-off" source vertices one at a time. It is possible to do this algorithm in linear time: $\mathcal{O}(m + n)$.

    To do so, have $n$ linked list nodes, one for each vertex. Then we keep an array $B$ where the $i$th entry of the array stores a pointer to node $i$.
    Then, we also keep an array $B$ of linked lists where the $i$th entry of that array is a doubly-linked list that has $i$ incoming edges.
    To peel off a source, you take anything in the $0$th entry of the array and take it out, call it $u$. Then, for each outgoing edge $(u, v)$, move $v$
    to the lower bucket in the array (since it now has one less incoming edge). Rinse and repeat.

    This is a bit too complicated. There is an easier linear-time algorithm.
\end{algothm}

Consider the following pre and postorder traversals (subscripts dropped for clarity):
\[ [[][]] [] [[]] \]
then, it's easier to see that the vertex associated with last closing bracket is a source; it has nowhere else that lead to it.
This leads to the following result:
\begin{theorem} [Finding a Source with DFS]
    The vertex $v$ with the largest postorder number must be a source.

    \begin{proof*}
        Suppose for the sake of contradiction that $v$ is not a source. That means there must exist a $u \in V$ such that $(u, v) \in E$.

        We know that the only possibilities are that $[_u ]_u [_v ]_v$ or $[_v [_u ]_u ]_v$. The latter means that $(u, v)$ is a back edge, which
        would mean there is a cycle. However there are no cycles since this is a DAG, so this cannot be the case. The former means that $u$ and $v$
        are disconnected, but there is clearly an edge between them or that $v$ was already visited, which is also not the case.

        By exhaustion of cases, we have a contradiction, so $v$ must be a source. \qed
    \end{proof*}
\end{theorem}

\begin{algothm} [Topological Sort]
    We begin by doing a DFS traversal on $G$. Then, we sort by postorder number. This exactly a topological sort of the graph.

    This works because peeling off the source causes the DFS to be the exact same, so the order of postordering numbers does not change.

    Note that postorder number is bounded between $2$ and $2n$, so it is easy to sort in linear time. Thus the runtime is dominated by DFS,
    $\mathcal{O}(n + m)$.
\end{algothm}

\subsubsection{Strongly Connected Components}

Next we try to find strongly connected components in a directed graph.

\begin{definition} [Strongly Connected]
    For some directed graph $G = (V, E)$, $u, v \in V$ are strongly connected if $u$ has a parth to $v$ and $v$ has a path to $u$.
\end{definition}

\begin{definition} [Strongly Connected Component (SCC)]
    A strongly connected component in a graph $G$ is a maximal subset of strongly connected vertices.
\end{definition}

%% TODO add another example of a graph

An easy algorithm to find SCCs would be to do $n$ DFS's from each vertex and check strong connectivity. However, this is fairly slow,
about $\mathcal{O}(mn + n^2)$.

\begin{theorem} [SCC Graph is a DAG]
    The graph made from treating the SCCs as nodes is acyclic (a DAG).

    \begin{proof*}
        Suppose there was a cycle. Then, all the vertices in those components forming a cycle are strongly connected,
        which would contradict the fact that the SCCs were maximal. Thus, there cannot be a cycle.
    \end{proof*}
\end{theorem}

We want to find a sink of this reduced DAG. In particular, running a DFS from the sink SCC, we can peel it off and recurse.
This means we want to find the SCCs in reverse topological sorted order. Imagine we had a subroutine that could find a vertex in a sink in constant time. Then, the runtime would be:
$\mathcal{O}(n + m)$ because we have to DFS over every single strongly-connected component, visiting every vertex and edge once.

To do this, first let us reverse the graph (make all the edges go the other direction). This will turn any source into a sink and vice versa.
Now we have to find a source SCC in the reversed graph $G_{rev}$. The claim is that the highest postorder number still works.

\begin{theorem}
    If $u$ has the highest postorder number, then $u$ is in the source SCC.

    \begin{proof*}
        Suppose that $u$ is not in the source SCC. That means there is another SCC that point into that SCC of $u$. This means that $(v, w) \in E$,
        where $w$ has a path to $u$ and $u$ has a path to $w$. Since $v$ thus has a path to $u$ but $u$ has the highest postorder number, the intervals cannot be disjoint.
        Furthermore, the intervals cannot be inside each other, as this would imply $u$ would have a path $v$; but that would mean they're in the same SCC, which they're not.
        Thus, the edge $(v, w)$ cannot exist, meaning that $u$ is in the source SCC.
    \end{proof*}
\end{theorem}

Finally our algorithm to find the SCCs is as follows:

\begin{algothm}
    \begin{enumerate}
        \item Reverse the graph and run topological sort on it. This will give you the ordering $S$ of sinks in the original graph.
        \item Run DFS on each sink $s$ in order. The vertices reached are in SCC of $s$, so we can remove them from the graph (mark them) and keep going.
        \item The SCCs removed are all the SCCs.
    \end{enumerate}

    This algorithm requires a topological sort and then small DFS's which amount to a single big DFS. This is
    $\mathcal{O}(m + n)$.
\end{algothm}