\section{Graph Algorithms}
\subsection{Lecture 1: Single-Source Shortest Paths}
We approach the SSSP problem with possibly negative edge weights. As input, we get the directed graph $G = (V, E, w)$ with weight function $w: E \to \R$
and start vertex $s \in V$. We also take for a set of edges $S$, $w(S) = \sum_{s \in S} w(s)$. The algorithm should output the lengths of the shortest paths from $s$ to any other $v \in V$ (or the shortest path tree).
We know a few algorithms for this.

\begin{itemize}
    \item Djikstra (with a Fibonacci Heap): $O(m + n \log n)$, which only works if $w \geq 0$.
    \item Bellman-Ford: $O(mn)$.
    \item Bernstein-Nanongkai-Wulff-Nilsen: $\tilde{O}(m \log W)$, where $|w| \leq W$.
\end{itemize}
The $\tilde{O}(f)$ means $f \cdot \text{Polylog}(f)$ We discuss the third one today.

We define a price function as a function $\phi : V \to \R$ and the associated price-reduced weight function as:
\[ w_{\phi}((u, v)) = w((u, v)) + \phi(u) - \phi(v) \]
Note the following observations. For any path $P = (v_0, \dots, v_r)$, $w_{\phi}(P) = w(P) + \phi(v_0) - \phi(v_r)$.
As a corollary, for all cycles $C$, $w(C) = w_{\phi}(C)$. This implies the shortest path in $G_{\phi} = (V, E, w_{\phi})$ is the same as in $G$.
Our goal is thus to find a $\phi$ such that for all edges, $w_{\phi} \geq 0$, then reduce to Djikstra.

\begin{theorem}
    A $\phi$ satisfying $w_{\phi} \geq 0$ exists if and only if there exist no negative cycles in the original graph $G$.

    \begin{proof*}
        Clearly if such a $\phi$ exists, $w_{\phi}(C) \geq 0$ for any cycle which is the same as $w(C)$ by our observations above.

        For the other direction, let $\phi(v) = d(s, v)$, where $d(s, v)$ is the length of the shortest path from $s$ to $v$.
        This means taking some neighbor $u$ of $v$, 
        \[ \phi(v) = d(s, v) \leq d(s, u) + w(u, v) \]
        This means that $w_{\phi}(u, v) = w(u, v) + \phi(u) - \phi(v) \geq w(u, v) + d(s, u) - d(s, u) - w(u, v) = 0$.
    \end{proof*}
\end{theorem}

The first person to use this technique was [G '75]. He showed the following result.

\begin{theorem} If you have an algorithm that finds a good $\phi$ in time $T(m)$ for special case
when $w(e) \geq -1$, then you can solve the general case of $|w| \leq W$ in time $O(T(m) \log W)$.

\begin{proof*}
    Round $W$ up to the nearest power of 2, $W = 2^k$. We proceed by induction on $k$. If $k = 0$, all the edge weights are already bigger than $-1$, so we're finished.

    For an inductive case, we use the following algorithm, using solve as as our subroutine:
    \begin{algorithmic}
        \Function{A}{$G = (V, E, w), k$}
            \If{$k = 0$} \State \Return Solve$(G)$
            \Else
            \State $\hat{w}(e) \gets \ceil{\frac{w(e)}{2}}$
            \State $\hat{\phi} \gets A(V, E, \hat{w}, k - 1)$
            \State $\phi \gets 2 \hat{\phi}$
            \State \Return Solve$(G_{\phi})$
            \EndIf
        \EndFunction
    \end{algorithmic}

    To prove correctness, we need to show that $G_{\phi}$ is solvable. Take some edge $e = (u, v)$.
    \begin{align*}
        w(e) &\geq 2 \ceil{\frac{w(e)}{2}} - 1 \\
        &= 2 \hat{w}(e) - 1 \\
        w_{\phi}(e) &= w(e) + \phi(u) - \phi(v) \\
        &\geq 2 \hat{w}(e) - 1 + 2 \hat{\phi}(u) - 2 \hat{\phi}(v) \\
        &= 2 \hat{w}_{\hat{\phi}}(e) - 1 \\
        &\geq -1
    \end{align*}
    Since the weights are nonnegative with the good $\hat{\phi}$.
\end{proof*}
\end{theorem}

Now, this is the novel portion.
We will focus on $G_s$ which is $G$ with a dummy vertex $s$ which has weight-0 to everyone else. We will focus on finding a $\phi$ for $G_s$ (which we will label as $G$ for brevity).

The algorithm has the following ingredients:
\begin{enumerate}
    \item The subroutine Low Diameter Decomposition. $LDD(G, D)$ takes in a graph $G$ where $w \geq 0$, and outputs $E^{rem} \subseteq E$ such that every strongly connected component of $G \setminus E^{rem}$
    has weak diameter at most $D$. The weak diameter is $\max_{u, v \in \text{same SCC}} d_G(u, v)$. Furthermore, $\Pr{e \in E^{rem}} \leq O\qty(\frac{w(e) \log^2 n}{D} + n^{-10})$. This is a fast algorithm, with running time $\tilde{O}(m)$.
    \item The subroutine Fix DAG Edges. This finds $\phi$ that makes $w_{\phi} \geq 0$ when $G$ is a DAG. Using an SCC graph, this will mean $w_{\phi}(e) \geq 0$ for $e$ going across SCCs.
    To implement this in linear time, just find the distance from a source with dynamic programming and just set the price function to be that.
    \item The subroutine Elim Neg. This finds a $\phi$ in time $O(\log n \cdot \sum_v (1 + \eta_G(v)))$ where $\eta_G(v)$ is the number of negative edges
    on the shortest path to $v$. This algorithm assumes all in-degrees and out-degrees are $O(1)$.
    \item The subroutine Scale Down. It takes in two numbers $\Delta$ and $B$. Assumes that $\eta(G) = \max_{v \in V} \eta_G(v) \leq \Delta$ and assumes all edges $w(e) \geq -2B$. It outputs $\phi$ such that $w_{\phi} \geq -B$.
\end{enumerate}

Furthermore, in the original graph we can assume WLOG, all degrees are $O(1)$, so we can actually use the third condition. We do this with graph blow-up on each vertex
$v$. Suppose $v$ has $x$ in-degree and $y$ out-degree. We can make a $x + y$-cycle with all edge weights $0$. Each vertex has one of $v$'s original edges, either an incoming or outgoing one.
This does not change shortest paths. All this does is blow up the number of vertices to $O(m)$ at most, which doesn't affect the runtime given.

Let's put them all together.
\begin{algothm}
    \begin{algorithmic}
        \Function{Main}{$G = (V, E, w)$}
        \State $B \gets 2n \text{ (rounded up to nearest power of 2)}$
        \State $\bar{w} \gets B w$
        \State $\phi_0 \gets 0$
        \For {$i = 1$ to $\log_2 B$}
            \State $\psi_i \gets$ ScaleDown$(G_{\phi_{i - 1}}, \Delta = n, \frac{B}{2^i})$
            \State $\phi_i \gets \phi_{i - 1} + \psi_i$
        \EndFor
        \State $w^* \gets \bar{w}_{\phi_{\log B}} + 1$
    \EndFunction
    \end{algorithmic}
\end{algothm}