\section{Lecture 14}

\subsection{Quantum Channel Generality}

We begin by recalling the quantum channels, and state that these are actually very general objects.

\begin{theorem}
    A superoperator $\mathcal{N}$
    is a completely positive, trace-preserving linear superoperator if and only if it is a quantum channel.

    \textbf{Proof} % To stop tcolorbox from shitting itself

    We have already shown the if step. To show the only if,
    consider $\mathcal{H}_R \cong \H_A$, where the $\H_R$ space is the one
    which will be acted on by the identity and $\H_A$ is the actual input space.
    Consider $\{\ket{j}\}$ as the orthonormal basis for this space.

    Consider $\ket{\Phi} = \sum_j \ket{j}_A \ket{j}_R$. Now, applying the superoperator tensored with the identity superoperator:
    \[ \qty(\mathcal{N} \otimes \textrm{Id}_R)\qty(\ketbra{\Phi}) = \sum_i p_i \ketbra{\Phi_i} \]
    for $\ket{\Phi_i} \in \H_B \otimes \H_R$. Because of positive semidefiniteness, we can take the square root of $p_i$
    and write $\ket{\Phi_i}$ as a linear combination of its basis elements:
    \[ \sqrt{p_i} \ket{\Phi_i} = \sum_{\alpha, j} K_i^{\alpha j} \ket{\alpha}_B \ket{j}_R \]
    We define the map
    \[ K_i = \sum_{\alpha, j} K_i^{\alpha j} \ketbra{\alpha}{j} \]
    Then, this means $\sqrt{p_i} \ket{\Phi_i} = K_i \ket{\Phi}$ (we contract over exactly the coefficients of $\ket{\Phi_i}$),
    so the output of the channel tensored with the identity yields
    \[ \sum_i p_i \ketbra{\Phi_i} = \sum_i K_i \ketbra{\Phi} K_i^{\dagger} \]

    Now consider another state $\ket{\psi} = \sum_j c_j \ket{j} \in \H_A$ and
    define $\ket{\overline{\psi}} = \sum_j c_j^* \ket{j} \in \H_R$. Then:
    \[ \bra{\overline{\psi}}\ket{\Phi} = \sum_{j', j} c_j' \braket{j'}{j}_R \ket{j}_A = \sum_j c_j \ket{j} = \ket{\psi} \]
    Now act
    \begin{align*}
        \bra{\overline{\psi}} \qty(\mathcal{N} \otimes \textrm{Id}_R)\qty(\ketbra{\Phi}) \ket{\overline{\psi}} &= \sum_i K_i \bra{\overline{\psi}} \ketbra{\Phi} \ket{\overline{\psi}} K_i^{\dagger} \\
        \mathcal{N}( \bra{\overline{\psi}} \ketbra{\Phi}  \ket{\overline{\psi}}) &= \sum_i K_i \ketbra{\psi} K_i^{\dagger} \\
        \mathcal{N}(\ketbra{\psi}) &= \sum_i K_i \ketbra{\psi} K_i^{\dagger}
    \end{align*}
    Now, by a simple superposition, this works for any density operator $\rho = \sum_i \lambda_i \ketbra{\psi_i}$
    \begin{align*}
        \mathcal{N}(\rho) &= \sum_j \lambda_j \mathcal{N}(\ketbra{\psi_j}) \\
        &= \sum_{i, j} \lambda_j K_i \ketbra{\psi_j} K_i^{\dagger} \\
        &= \sum_{i} K_i \qty(\sum_{j} \lambda_j\ketbra{\psi_j}) K_i^{\dagger} = \sum_i \lambda_i K_i \rho K_i^{\dagger}
    \end{align*}
    Finally, we will show that the Kraus operators are properly normalized using the trace-preserving property of
    the superoperator
    
    \begin{align*}
        \tr \mathcal{N}(\rho) &= \sum_i \tr(K_i \rho K_i^{\dagger}) \\
        \tr \rho &= \tr(\rho \qty(\sum_i K_i^{\dagger} K_i))
    \end{align*}
    Take $\rho = \ketbra{\psi}$. Then the above property implies
    \[ \mel**{\psi}{\sum_i K_i^{\dagger} K_i}{\psi} = \braket{\psi} \implies \sum_i K_i^{\dagger} K_i = I \]
    Thus, this is a valid quantum channel (in Kraus representation).
\end{theorem}

\subsection{Master Equations}
Recall that we could define discrete noiseless evolution by a unitary operator
\[ \rho \mapsto U \rho U^{\dagger} \]
Furthermore, we could define continuous noiseless evolution by the Von Neumann equation (the
density operator version of the Schrodinger equation)
\[ i \dv{\rho}{t} = [H, \rho] \]
Now, for a noisy (or open-system) discrete evolution, we now have quantum channels
\[ \rho \mapsto \mathcal{N}(\rho) \]
Now, we expect some kind of open-system continuous evolution of the form $\dv{\rho_A}{t} = f(\rho_A)$.
Unfortunately, this is not usually possible to do, because the evolution is not dependent on purely on initial condition;
information can leave the $A$ system and then come back later. Call $\H_A \cong \H_E$, where the $E$ system
is the environment. At $t = 0$, suppose the state is $\ket{\psi}_A \ket{0}_E$. We will write
\[ U(t_0) = e^{-iHt_0} = SWAP, U(t_0) \ket{i} \ket{j} = \ket{j} \ket{i} \]
So at $t = t_0$, $\rho_A(t_0) = \ketbra{0}$. All the information has escaped out of $A$. But afterwards,
$\rho_A(2 t_0) = \ketbra{\psi}$. The information flowed back into $A$. At time $2t_0$ we needed the state at $0$,
so the system is not memoryless and we cannot create an evolution law in the manner we described.

However, if $|\H_E| \gg |\H_A|$ and the environment is evolving sufficiently ``fast,'' then information
that escapes the system and doesn't immediately return is gone forever to the environment. To model this,
we replace $\H_E$ by some ``clean copy'' in a fixed state after every $\Delta t_{forget} \ll 1$. If we are fine with coarse-grained evolution,
then we will look at updating the state in increments $\delta t$
where $\Delta t_{forget} \ll \delta t \ll \Delta t_{system}$, where the last term is the time for the
system to evolve significantly. We will figure this as
\[ \mathcal{N}_{\delta t}(\rho(t)) = \rho(t + \delta t) = \rho(t) + O(\delta t) \]
By Taylor expansion:
\[ \mathcal{N}_{\delta t} = \textrm{Id} + \delta t \mathcal{L} \]
where the superoperator $\mathcal{L}$ is called the \textbf{Lindbladian} of the system. By
rearranging and taking limits:
\[ \dv{\rho}{t} = \mathcal{L}(\rho) \]
Assuming that $\mathcal{L}$ is independent of time, then the solution to the equation is
\begin{align*}
    \rho(t) &= \lim_{n \to \infty} \qty(Id + \frac{t}{n} \mathcal{L})^n \qty(\rho(0)) \\
    \rho(t) &= e^{t \mathcal{L}}(\rho(0))
\end{align*}
But what does the Lindbladian really look like? Note that $\mathcal{N}_{\delta t}$ is a quantum channel,
so it has a Kraus representation:
\[ \mathcal{N}(\rho(t)) = \sum_{i} K_i \rho K_i^{\dagger} = \rho + O(\delta t) \]
The easiest way to write this is $K_0 = I + \delta t \qty(-iH + K)$ where $H$ and $K$ are Hermitian (this is the split into
Hermitian and anti Hermitian parts). Furthermore, $K_i$ cannot have any constant terms: they're all accounted for.
To get a linear term in $K_i \rho K_i^{\dagger}$, we need order 1/2 terms for $i \neq 0$:
$K_i = \sqrt{\delta t} L_i$. But by normalization:
\begin{align*}
    \sum_i K_i^{\dagger} K_i &= I \\
    I &= I + \delta t(iH + K) + \delta t(-iH + K) + \sum_{i > 0} L_i^{\dagger} L_i \delta t \\
    0 &= 2 \delta t K + \delta t \sum_{i > 0} L_i^{\dagger} L_i \\
    K &= -\frac{1}{2} \sum_{i > 0} L_i^{\dagger} L_i \\
    \dv{\rho}{t} &= \frac{\mathcal{N}_{\delta t} (\rho) - \rho}{\delta t} \\
    \dv{\rho}{t} &= \frac{1}{\delta t} \qty[\sum_i K_i \rho K_i^{\dagger} - \rho] \\
    \dv{\rho}{t} &= (-iH - \frac{1}{2} \sum_{i > 0} L_i^{\dagger} L_i)\rho + \rho (iH - \frac{1}{2}\sum_{i > 0} L_i^{\dagger} L_i) + \sum_i L_i \rho L_i^{\dagger}
\end{align*}
Let's clean this up.
\begin{theorem}
    Continuous-time noisy evolution of a state, where each step is quantum channel,
    is described by the \textbf{Lindblad master equation}:
    \[ \dv{\rho}{t} = -i [H, \rho] + \sum_{i > 0} \qty(L_i \rho L_i^{\dagger} - \frac{1}{2} L_i^{\dagger} L_i \rho - \frac{1}{2} \rho L_i^{\dagger} L_i) \]
\end{theorem}
The first term is what one gets from normal Unitary evolution from a Hamiltonian $H$. The second term
$ \sum_{i > 0} \qty(L_i \rho L_i^{\dagger})$ is the effects of noise,
where $L_i$ are considered quantum jump operators where the jump is described by $\ket{\psi} \mapsto L_i \ket{\psi}$.
The last two terms preserve the normalization of $\rho$ overall. This was a lot of formalism, let's explore an example.
\begin{example}
    Consider a photon emission by an excited electron state. We will
    model our $A$ system as the electron and the photon as noise. We model $\ket{0}$ as the ground state of the electron
    and $\ket{1}$ as the excited state of the electron. We say the eigenvalue of the energy level $\ket{0}$ is $0$ and $\ket{1}$ is $E$,
    giving the Hamiltonian as $H = E \ketbra{1}$. We will model $L_1 = \sqrt{\Gamma} \ketbra{0}{1}$, where $\Gamma$ is the photon emission rate.
    Now, let's try to solve the master equation. The initial state is:
    \[ \rho = \begin{pmatrix}
        \rho_{00} & \rho_{01} \\ \rho_{10} & \rho_{11}
    \end{pmatrix} = \rho_{00} \ketbra{0} + \rho_{01} \ketbra{0}{1} + \rho_{10} \ketbra{1}{0} + \rho_{11} \ketbra{1}{1} \]
    Then note that the first and last terms commute with the Hamiltonian and one of the cross terms is:
    \begin{align*}
        H \rho &= E \ketbra{1}{1} \rho_{01} \ketbra{0}{1} = 0 \\
        \rho H &= \rho_{01} \ketbra{0}{1} E \ketbra{1}{1} = E \rho_{01} \ketbra{0}{1}
    \end{align*}
    the other cross term will be negative of this, so we have
    \begin{align*}
        -i [H, \rho] &= -i \qty(H \rho - \rho H) \\
        &= \begin{pmatrix}
            0 & iE \rho_{01} \\
            -iE\rho_{10} & 0
        \end{pmatrix}
    \end{align*}
    Now, the other terms:
    \[ L_1 \rho L_1^{\dagger} = \Gamma \ketbra{0}{1} \rho \ketbra{1}{0} = \Gamma \rho_{11} \ketbra{0}{0} \]
    \[ -\frac{1}{2} L_1^{\dagger} L_1 \rho = -\frac{\Gamma}{2} \ketbra{1}{1} \rho = -\frac{\Gamma}{2} \qty(\rho_{10} \ketbra{1}{0} + \rho_{11} \ketbra{1}{1}) \]
    \[ -\frac{1}{2} \rho L_1^{\dagger} L_1 = -\frac{\Gamma}{2} \rho \ketbra{1}{1} = -\frac{\Gamma}{2} \qty(\rho_{11} \ketbra{1} + \rho_{01} \ketbra{0}{1})\]
    Putting it together,
    \[ \dv{\rho}{t} = \begin{pmatrix}
        \Gamma \rho_{11} & iE\rho_{01} - \frac{1}{2} \Gamma \rho_{01} \\
        -iE \rho_{10} - \frac{1}{2} \Gamma \rho_{10} & - \Gamma \rho_{11}
    \end{pmatrix} \implies \rho(t) = \begin{pmatrix}
        \rho_{00}(0) + \qty(1 - e^{-\Gamma t})\rho_{11}(0) & e^{\qty(iE - \frac{\Gamma}{2})t }\rho_{01} (0) \\
        e^{\qty(-iE - \frac{\Gamma}{2})t }\rho_{10} (0) &  e^{- \Gamma t} \rho_{11}(0)\\
    \end{pmatrix} \]
    At any time, it can drop down, so as $t \to \infty$, the density approaches $\ketbra{0}$. The cross-terms gain a phase based on the eigenenergy,
    which is what that Hamiltonian would do on its own.
\end{example}