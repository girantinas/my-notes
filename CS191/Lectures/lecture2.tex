
\section{Lecture 2}

\subsection{Axioms of Quantum Mechanics}

We list some axioms of Quantum Mechanics. Consider an electron with $k$ energy levels, $\ket{0}, \ket{1}, \dots, \ket{k - 1}$.

\begin{note}[Superposition Principle]
    If there are $k$ distinguishable (eigenstates) of a system, then the state of a system can be written as:
    \[ \ket{\psi} = \sum_{j = 0}^{k - 1} \alpha_j \ket{j} \]
    where $\alpha_j \in \C$ and $\sum_{j} |\alpha_j|^2 = 1$.
\end{note}

This forms a Hilbert space, i.e. a Complex inner product space (but we will often think of all amplitudes as real). The $\{ \ket{j} \}_{j = 0}^{k - 1}$ forms
a basis for this state space. We can think of
    \[ \ket{\psi} = \begin{pmatrix}
        \alpha_0 \\ \alpha_1 \\ \vdots \\ \alpha_{k - 1}
    \end{pmatrix}, \ket{0} = \begin{pmatrix}
        1 \\ 0 \\ \vdots \\ 0
    \end{pmatrix}, \ket{1} = \begin{pmatrix}
        0 \\ 1 \\ \vdots \\ 0
    \end{pmatrix}, \dots \]

For inner products, we use Dirac's Bra-Ket notation. As we have already seen, the ``kets" are regular vectors and the
``bras" $\bra{\psi} = \ket{\psi}^{\dagger}$ are elements of the dual vector space (which can be thought of as conjugate transposes). This means:
\[ \bra{\psi} = \ket{\psi}^{\dagger} = \sum_j \qty(\alpha_j \ket{j})^{\dagger} =  \sum_j \alpha_j^* \bra{j} \]
where $(\cdot)^*$ is the complex conjugate.

Now define $\ket{\phi} = \sum_j \beta_j \ket{j}$. We can take inner products by using the following notation:
\[ \langle \psi, \phi \rangle = \braket{\psi}{\phi} = \qty(\sum_{i} \alpha_i^* \bra{i})\qty(\sum_{j} \beta_j \ket{j}) = \sum_{i,j} \alpha_i^* \beta_j \braket{i}{j} = \sum_{j} \alpha_i^* \beta_j\]
Because $\braket{i}{j} = 1$ if and only if $i = j$ (they form an orthonormal basis).

We generally use $k = 2$, call the Hilbert space generated $\mathcal{H}$. We typically think about chaining together (tensor-producting)
this Hilbert space with itself $n$ times. This is called a $n$-\textbf{qubit} state. A general state can then be written as:
\[ \ket{\psi} = \sum_{x \in \{0, 1\}^n} \alpha_x \ket{x} \]
with $\alpha_x \in \C$ and $\sum_{x} |\alpha_x|^2 = 1$.

\begin{note}[Measurement Principle]
    Pick an orthonormal basis $\mathcal{U} = \ket{u_0}, \ket{u_1}, \dots, \ket{u_{k - 1}}$. The outcome of a measurement is $j$ with
    probability $\qty|\braket{u_j}{\psi}|^2$. In this process, the state is also perturbed and turned into the state $\ket{u_j}$
\end{note}

Look at last lecture for examples of measuring in different bases, with real amplitudes one can think about qubit states geometrically.
The basis $\{ \ket{+}, \ket{-} \}$ serves us well.

\subsection{Bell Inequalities}
Let us look more closely at combining two qubits, each with states $\alpha_0 \ket{0} + \alpha_1 \ket{1}$, $\beta_0 \ket{0} + \beta_1 \ket{1}$.
We (tensor) product them together, producing a state:
\[ \ket{\psi} = \alpha_0 \beta_0 \ket{00} + \alpha_0 \beta_0 \ket{01} + \alpha_0 \beta_0 \ket{10} + \alpha_0 \beta_0 \ket{11} \]
but most states are not a product of two states.

The Bell basis states are a common example of states which are \textbf{entangled}, e.g. cannot be written as ``product states."
\[ \ket{\Phi^{\pm}} = \ftwo \ket{00} \pm \ftwo \ket{11}, \ket{\Psi^{\pm}} = \ftwo \ket{01} \pm \ftwo \ket{10}  \]
These four states form an orthonormal basis for two qubits.

Suppose your system was in the state $\Phi^+$ and we did a partial measurement on the qubit
Then with probability $1/2$ we collapse to $\ket{00}$ and with probability $1/2$ we collapse to $\ket{11}$. Note that we could achieve this
in a classical sense too, with correlated (``glued'') coin flips.

Furthermore, the Bell states are rotationally invariant. 
\begin{theorem}
    In any basis, we can write the Bell States as:
    \[ \ket{\Phi^+} = \ftwo \ket{00} + \ftwo \ket{11} = \ftwo \ket{vv} + \ftwo \ket{v^{\perp} v^{\perp}} \]
\end{theorem}

Let's prove this. Suppose $v = \alpha \ket{0} + \beta \ket{1}$. Then without loss of generality, we can write $v^{\perp} = -\beta^* \ket{0} + \alpha^* \ket{1}$.
This means that:
\begin{align*}
    \ket{vv} + \ket{v^{\perp} v^{\perp}} &= \qty(\alpha \ket{0} + \beta \ket{1})\qty(\alpha \ket{0} + \beta \ket{1}) + \qty(-\beta^* \ket{0} + \alpha^* \ket{1}) \qty(-\beta^* \ket{0} + \alpha^* \ket{1}) \\
    &=  \ftwo \qty(\ket{00} + \ket{11})
\end{align*}
where some algebra is elided. Note that we could achieve this
in a classical sense too, with correlated coin flips that are rotated.

To go beyond classical computation, we consider two qubit measurements.
The first player measures in the standard basis and the second player measures in a new basis, $\{ \ket{v}, \ket{v^{\perp}} \}$, rotated at an angle $\theta$ from the standard basis.
The probability that these two measurements are inequal is $\sin^2 \theta$ (for example, if the first measurement is $0$, then the state $\ket{00}$, so the component of $\ket{v}$ in the $\ket{0}$ direction is $\cos \theta$).

However, classically, the probability that one observes a different outcome is proportional to $\theta$.

So John Bell's experiment is as follows. Alice is given a uniformly random bit $x$ and Bob is given a uniformly random bit $y$. They must each report back
a bit $a$ and $b$ respectively.
Alice and Bob ``win'' the game if $xy = a + b \pmod{2}$.

They can play the game in two ways: either classically or quantumly. Classically, they cannot communicate (apart from maybe the ``glued'' coin).
In the quantum setup, 
Alice and Bob share a Bell state.
If Alice chooses a bit $0$, they measure their qubit in the standard basis, otherwise they measure it in a basis rotated by $\pi/4$. If Bob
chooses a bit $1$, they measure their qubit in a basis rotated by $\pi/8$, otherwise they measure in a basis rotated by $-\pi/8$. Call their measured
bits $a$ and $b$ respectively.

We then mention the following two facts:
\begin{enumerate}
\item No classical strategy can win with probability $> 75\%$. A randomized strategy can do no better than a deterministic strategy since the opponent's strategy is known.
The best deterministic strategy is to report $a = 0$ and $b = 0$ (or $a = 1$ and $b = 1$), because $xy = 0$ with probability $75\%$ (if at least one of the bits is 0); trying to force the answer to be $1$ will give you a lower probability of success.
You can do no better. The glued coin doesn't help you either; the best it could do is give you a shared source of randomness.
\item In each of the 4 cases, the probability winning in a quantum setup is $\cos^2 \pi/8 \approx 85\%$. For example, take the case
when $x$ and $y$ are both 0. Then they need to both measure a 1 or both measure a 0. The probability Alice measures a 0 is $1/2$ and then collapses the state to 
a $\ket{00}$. The probability that Bob then sees a $0$ is $\cos^2 \frac{\pi}{8}$ because of the rotation, giving us $\frac{1}{2} \cos^2 \frac{\pi}{8}$. Likewise, the probability Alice measures a 1 is $1/2$ and then
collapses the state to a $\ket{11}$. The probability that Bob then sees a $1$ is $\cos^2 \frac{\pi}{8}$, so overall the probability is $2 \cdot \frac{1}{2} \cdot \cos^2 \frac{\pi}{8} = \cos^2 \frac{\pi}{8}$.
The other cases are similar.
\end{enumerate}

which clearly shows the quantum setup gives us something not present in the classical one.
