\section{Lecture 11}

\subsection{Describing Noisy Quantum States}
Recall that in a (noisy) classical system, we define a probability distribution over $N$ states
$\{p_i\}_{i = 1}^N$ where each $p_i$ tells you the probability of being in state $i$. Similarly,
we can define probabilities $p_i$ of being in a state $\ket{\psi_i}$. Suppose we measure in the $\ket{m}$
basis, so the probability of measuring $m$ is
\[ \Pr{m} = \sum_i p_i \Pr{m \mid i} = \sum_i p_i \qty|\braket*{m}{\psi_i}|^2 \]
But, we can rewrite this quadratic term as a trace:
\[ \braket*{\psi_i}{m}\braket*{m}{\psi_i} = \Tr\qty(\ket{m}\ketbra*{m}{\psi_i} \bra{\psi_i}) \]
Using the fact that trace is linear, we can see:
\[ \Pr{m} = \Tr\qty(\qty(\sum_i p_i \braket{\psi_i}{\psi_i}) \ketbra{m}{m}) = \Tr(\rho \ketbra{m}{m})\]
Where we wrote the \textbf{density operator} as:
\[ \rho = \sum_i p_i \ketbra{\psi_i}{\psi_i} \]
Note that $\rho$ consists of $d^2$ matrix elements, where $d$ is the dimension of the Hilbert space.
However, the space of ensembles is infinite-dimensional, because there are an infinite amount of states
and we can define any probability distribution we'd like on them.
The reason for this is becuase different ensembles can have the same $\rho$ oeprator.

\begin{example}
    Take the probability distribution $p(\ket{0}) = 0.5$, $p(\ket{1}) = 0.5$, which gives us density
    \[ \rho = \frac{1}{2} \ketbra{0}{0} + \frac{1}{2} \ketbra{1}{1} = \frac{1}{2} I \]
    But now take the distribution $p(\ket{+}) = 0.5$, $p(\ket{-}) = 0.5$, which gives us density
    \[ \rho = \frac{1}{2} \ketbra{+}{+} + \frac{1}{2} \ketbra{-}{-} = \frac{1}{2} I \]
\end{example}

Seeing this, it is natural to say that a density matrix $\rho$ defines a noisy quantum state; if $p_i \neq 0$ for more than one $i$, then
this is
called a \textbf{mixed states}. If you have two systems
with the same density matrix, they are indistinguishable. A state $\ket{\psi} \in \mathcal{H}$ is called a \textbf{pure state}.
Its density matrix is $\rho = \ketbra{\psi}{\psi}$. One may note that this is a \textbf{projector}, or
orthogonal projector which takes a state and projects it onto its component in the $\psi$ direction. Formally:

\begin{definition}
    A projector $P$ is a Hermitian matrix with only two eigenvalues, $0$ and $1$, i.e.
    \[ P = P^{\dagger} = P^2 \]
\end{definition}

The nice thing
about this is that we don't need to worry about the global phase indistinguishability, as that will cancel in the
projector expression.

Let's look at a few properties of 
$\rho$.
\[ \rho^{\dagger} = \sum_i \qty(\ketbra{\psi_i}{\psi_i})^{\dagger} = \sum_i p_i \ketbra{\psi_i}{\psi_i} = \rho \]
Thus, $\rho$ is Hermitian. Furthermore, it is actually positive semi-definite:
\[ \bra{\psi} \rho \ket{\psi} = \sum_i p_i \qty|\braket{\psi_i}{\psi_i}|^2 \geq 0 \]
Thus, its eigenvalues are all nonnegative. Finally
\[ \Tr(\rho) = \sum_i p_i = 1 \]

\begin{theorem}
It turns out, any trace-1 PSD matrix $\rho$ is the density matrix of infinitely many
ensembles $\{(p_i, \ket{\psi_i})\}$.

\begin{proof*}
By the spectral theorem, we can write
such a matrix as
\[ \rho = \sum_j \lambda_j \braket{\psi_j}{\psi_j} \]
Since $\rho$ is PSD, then $\lambda_j \geq 0$ and since it has trace 1, $\sum_{i} \lambda_i = 1$.
Thus, this is the density operator for the ensemble $\{(\lambda_j, \ket{\psi_j})\}$.

By uniqueness of eigendecomposition, this is in fact the unique ensemble in which all the states
are orthonormal.
\end{proof*}
\end{theorem}

Now, let's visit the postulates of Quantum Mechanics in terms of density matrices.
\begin{theorem}
    \begin{enumerate}
    \item \textbf{Superposition}: For pure states, the original superposition principle states that states live in a Hilbert space $\ket{\psi} \in \mathcal{H}$. For density operators,
    we'd say that a state is described as a PSD trace-one matrix $\rho$ acting on $\mathcal{H}$.
    \item \textbf{Measurement}: For pure states, recall that we measure in an orthonormal basis $\{\ket{m}\}$ such that
    $\Pr{m} = \qty|\braket{\psi}{m}|^2$ and $\ket{\psi}$ collapses to $\ket{m}$. But to deal with more general measurements,
    e.g. measuring on a certain qubit, we have a projection-valued measure (PVM). We say we have a set of projectors
    $P_m$ such that $\sum P_m = I$. For a measurement on a one-qubit state, $P_m = \ketbra{m}{m}$ and for a measurement on the first qubit but not the second,
    $P_m = \braket{m}{m} \otimes I$, wherein measurement has
    \[ \Pr{m} = \mel{\psi}{P_m}{\psi}, \ket{\psi} \to \frac{P_m \ket{\psi}}{\sqrt{\mel{\psi}{P_m}{\psi}}} \]
    Now for density operators, this turns into
    \[ \Pr{m} = \sum_i p_i \mel{\psi_i}{P_m}{\psi_i} = \Tr\qty(\rho P_m) \]
    The update rule is a bit harder. Measuring $m$ gives information about which $\ket{\psi_i}$ we were in.
    \[ \rho \to \sum_i \Pr{\ket{\psi_i} \mid m} \frac{P_m \bra{\psi_i} \ket{\psi_i} P_m}{\mel{\psi_i}{P_m}{\psi_i}} \]
    We need to find the conditional probability that we were in state $\psi_i$ given a measurement of $m$. This yields:
    \[ \Pr{\ket{\psi_i} \mid m} = \frac{p_i \Pr{m \mid \ket{i}}}{\Pr{m}} = \frac{\mel{\psi_i}{P_m}{\psi_i}}{\Tr(\rho P_m)} \]
    This means the final update is:
    \begin{align*}
        \rho &\to \sum_{i} \frac{p_i}{\Tr(\rho P_m)} P_m \ket{\psi_i} \bra{\psi_i} P_m \\
        &= \frac{P_m \rho P_m}{\Tr(\rho P_m)}
    \end{align*}
    One can think about this as follows: project the operator $\rho$ into the subspace spanned by $P_m$,
    then ``normalize'' the trace by using the denominator.
    \item \textbf{Unitary Evolution}: For pure states, we evolve a state with a unitary by applying $\ket{\psi} \mapsto U\ket{\psi}$. For mixed states, we have:
    \[ \rho = \sum_i p_i \ketbra{\psi_i}{\psi_i} \mapsto \sum_i p_i U \ketbra{\psi_i}{\psi_i} U^{\dagger} = U \rho U^{\dagger} \]
    \end{enumerate}
\end{theorem}

With that formal stuff out of the way, let's think about a two state system. The spectral decomposition
of such an operator must have only two eigenvectors and with the trace 1 condition we have:
\[ \rho = \lambda \ketbra{\psi}{\psi} + (1 - \lambda)(I - \braket{\psi}{\psi}) \]
Without loss of generality, we can take $\lambda \geq \frac{1}{2}$ because we can just pick the biggest
we can parametrize $\lambda = \frac{1}{2} \qty(1 + r)$ where $0 \leq r \leq 1$ and we can think of
mixed states geometrically. Recall our parametrization of the Bloch sphere last lecture in terms of $\theta$ and $\phi$;
if $r = 1$ we get a pure state $\ketbra{\psi}{\psi}$, and if $r = 0$, then $\lambda = 1/2$ and the expression at the top is $\rho = \frac{1}{2} I$
regardless of the $\ket{\psi}$ we choose. Other mixed states lie in the inside of the sphere. Any $\rho \propto I$ is called a
``maximally mixed'' state.

\subsection{Reduced Density Matrices}
Suppose you have two coupled systems $\mathcal{H} = \mathcal{H}_A \otimes \mathcal{H}_B$. One can think
of $\mathcal{H}_A$ as a quantum computer or some other quantum system we wish to control; $\mathcal{H}_B$ is any other
noise/system that $\mathcal{H}_A$ can potentially interact with. Given some $\ket{\psi} \in \mathcal{H}_A \otimes \mathcal{H}_B$
we wish to only study its state on $\mathcal{H}_A$. But maybe $\ket{\psi} = \ket{\psi_A} \ket{\psi_B}$, then we'd be very lucky.

What do measurements on $A$ look like? They look like $P_m^{A} \otimes I^B$. The probability of observing $m$ is:
\[ \Pr{m} = \mel{\psi}{P_m^A \otimes I^B}{\psi} = \Tr\qty[(P_m^A \otimes I^B) \ketbra{\psi}{\psi}] \]
Pick some orthonormal bases $\ket{i}_A, \ket{j}_B$ for the two spaces. the trace is:
\begin{align*}
    \Pr{m} &= \sum_{i, j} \bra{i}_A \bra{j}_B P_m^A \otimes I^B \ketbra{\psi}{\psi} \ket{i}_A \ket{j}_B \\
    &= \sum_{i} \bra{i}_A P_m^A \qty(\sum_j \bra{j}_B\ket{\psi} \bra{\psi} \ket{j}_B) \ket{i}_A \\
    &= \sum_{i} \bra{i}_A P_m^A \Tr_B(\ketbra{\psi}{\psi}) \ket{i}_A \\
\end{align*}
Here we define 
\[ \Tr_B(X) = \sum_j \bra{j}_B X \ket{j}_B \]
Now, define the \textbf{reduced density matrix} $\rho_A = \Tr_B(\ketbra{\psi}{\psi})$. Then:
\[ \Pr{m} = \sum_i \bra{i}_A P_m^A \rho_A \ket{i} = \Tr(P_m^A \rho_A) \]
This is exactly our old measurement rule!

Finally, let's think about the relationship between $\rho_A$.
\[ \rho_A = \sum_i \lambda_i \braket{\psi_i}{\psi_i} \]
However, $\psi_i$ are orthogonal and form a basis, so we can combine the terms by $\ket{\psi_i}_A$:
\[ \ket{\psi} = \sum_i c_i \ket{\psi_i}_{A} \ket{\phi_i}_B \]
In general, the $\ket{\phi_i}$ are not orthogonal, but we can take out the $c$'s to normalize $\braket{\phi_i}{\phi_i} = 1$.
But on the other hand, by definition,
\[ \rho_A = \Tr_B(\ketbra{\psi}{\psi}) = \sum_{i, j}  c_i c_j \ket{\psi_i}\ket{\psi_j} \braket{\phi_j}{\phi_i} \]
But by comparing by the spectral decomposition, we must have the terms vanish if $i \neq j$, meaning:

Then the \textbf{Schmidt Decomposition} is
\[ \ket{\psi} = \sum_i \sqrt{\lambda_i} \ket{\psi_i}_A \ket{\phi_i}_B \]
where $\psi_i$ and $\phi_i$ are both simultaneously orthonormal. Thus, we must have:
\[ \rho_B = \sum_i \lambda \ket{\phi_i}\bra{\phi_i} \]
Thus they both have the same eigenvalues.
