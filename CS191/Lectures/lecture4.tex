
\section{Lecture 4}

\subsection{The Tensor Product}
Remember that we ``multiplied'' two states $\ket{0} \ket{0} = \ket{00}$. This combined two 2-state systems into a 4-state system.
But what is this mystical multiplication? The answer is the \textbf{tensor product}, denoted by the symbol $\otimes$. Really, it would be proper to say $\ket{0} \otimes \ket{0} = \ket{00}$,
in which we ``call'' the tensor product this nickname. We also saw that we could identify $\ket{00}, \ket{01}, \ket{10}, \ket{11}$ with vectors in $\C^4$,
in this sense, we say $\C^2 \otimes \C^2 \cong \C^4$, i.e. the spaces are isomorphic.

In a general setting, letting $\mathcal{H}_1$ be a system with $k$ levels and $\mathcal{H}_2$ be a system with $\ell$ levels, we say
\[ \mathcal{H}_1 \otimes \mathcal{H}_2 \text{ is the vector space spanned by } \{ \ket{i}\ket{j}\}_{0\leq i \leq k, 0 \leq j \leq \ell}\]
which is a vector space with dimension $k \cdot \ell$.

We can also talk about the tensor product of operators. When you tensor product two operators, they act on their spaces separately.
This is a circuit representation of $H \otimes Z$:

\begin{center}
\begin{quantikz}
    \lstick{} & \gate{H} & \qw \\
    \lstick{} & \gate{Z} & \qw
\end{quantikz}
\end{center}

In this circuit, $\ket{00} \mapsto H\ket{0} \otimes Z\ket{0} = \ket{+}\ket{0}$, keeping the tensor products separate.
On the other hand, note that the operator can be equivalently written as an element of $\C^{4\times 4}$.
\[ H \otimes Z = \begin{pmatrix}
    \ket{0 \cdot} \to \ket{0 \cdot} & \ket{1 \cdot} \to \ket{0 \cdot} \\
    \ket{0 \cdot} \to \ket{1 \cdot} & \ket{1 \cdot} \to \ket{1 \cdot}
\end{pmatrix} = \begin{pmatrix}
    \ftwo & \ftwo \\ \ftwo & -\ftwo
\end{pmatrix} \otimes \begin{pmatrix}
    1 & 0 \\ 0 & -1
\end{pmatrix} = \begin{pmatrix}
    \ftwo \begin{pmatrix}
        1 & 0 \\ 0 & -1
    \end{pmatrix} & \ftwo \begin{pmatrix}
        1 & 0 \\ 0 & -1
    \end{pmatrix} \\
    \ftwo \begin{pmatrix}
        1 & 0 \\ 0 & -1
    \end{pmatrix} & -\ftwo \begin{pmatrix}
        1 & 0 \\ 0 & -1
    \end{pmatrix} \\
\end{pmatrix} = \begin{pmatrix}
    \ftwo & 0 & \ftwo & 0 \\ 0 & -\ftwo & 0 & -\ftwo \\ \ftwo & 0 & -\ftwo & 0 \\ 0 & -\ftwo & 0 & \ftwo
\end{pmatrix} \]

The first matrix represents the parts of the subspace that the matrix acts on. The notation $\ket{b \cdot}$ corresponds to the subspace first bit being $b$ and the second bit being anything.

In addition, we will state one more fact about tensor products, which we will not prove:
\begin{theorem}
    The inner product multiplies under the tensor product.
    \[ (\bra{u} \otimes \bra{v})(\ket{x} \otimes \ket{y}) = \braket{u}{x} \braket{v}{y}\]
\end{theorem}

\subsection{No Cloning Theorem}
The no cloning theorem states that you cannot create a copy of a qubit. Formally,
\begin{theorem}[No Cloning]
    There is no unitary transform (quantum circuit) $U$ such that
    \[ U \ket{0} \ket{\psi} = \ket{\psi} \ket{\psi} \]
    for all possible states that $\ket{\psi}$ could take on.
\end{theorem}

But hang on, didn't we see a copying circuit before? Recall the following circuit:

\begin{center}
\begin{quantikz}
    \lstick{$\alpha \ket{0} + \beta \ket{0}$} & \ctrl{1} & \qw \rstick{$\alpha \ket{00} + \beta \ket{11}$} \\
    \lstick{$\ket{0}$} & \targ & \qw & \qw
\end{quantikz}
\end{center}

It sent $\ket{0} \mapsto \ket{00}$ and $\ket{1} \mapsto \ket{11}$, which uniquely defines
the transformation. But it doesn't work on a general state!
By linearity note that:
\[ \qty(\alpha\ket{0} + \beta\ket{1})\ket{0} = \alpha \ket{00} + \beta \ket{10} \mapsto \alpha \ket{00} + \beta \ket{11} \neq \qty(\alpha \ket{0} + \beta\ket{1}) \otimes \qty(\alpha \ket{0} + \beta \ket{1}) \]
This doesn't work.

To prove the no-cloning theorem another way, we then suppose for the sake of contradction that there exists a unitary $U$ that can clone. Then for two vectors:
\[ U \ket{\phi} \ket{0} = \ket{\phi}\ket{\phi}, U\ket{\psi}\ket{0} = \ket{\psi}\ket{\psi} \]
But a unitary cannot change the angle between two vectors. To begin with and end with, the inner product was:
\[ \braket{\psi}{\phi} \braket{0}{0} = \braket{\psi}{\phi} \braket{\psi}{\phi} \implies \braket{\psi}{\phi} = 0, 1 \]
Any unitary can only clone only orthogonal states or the same state. In general, you can only clone some orthonormal basis of your choosing,
like $\{\ket{0}, \ket{1}\}$.

\subsection{Superdense Coding}
Let's say Alice and Bob share a Bell state $\ket{\Phi^+} = \ftwo \ket{00} + \ftwo \ket{11}$. Alice has two bits of information $b_1, b_2 \in \{0, 1\}$
that she wants to send to Bob. Classically she can send this information in exactly two bits, no more, but with a quantum state, Alice
can do it with just one qubit. Based on the bits, she turns the state into one of the Bell states. Alice then sends Bob her qubit.
Now Bob can just measure the qubit in the Bell basis. But can we do better than a rate of 2 classical bits for one qubit? It turns out we can't.

\subsection{Quantum Teleportation}
Professor Vazirani assures us that Alice has been working hard in her lab and made a state $\alpha \ket{0} + \beta \ket{1}$
that takes a LONG time to make. Unfortunately, she does not have access to an apparatus needed to process this qubit into cool things.
Fortunately, she shares a Bell state with Bob and can exchange (classical) information with him, who does have such an apparatus.
As we will see, the following circuit allows Alice to share her state with Bob:

\begin{center}
\begin{quantikz}
    \lstick{$\alpha \ket{0} + \beta\ket{1}$}\qw & \gate[wires=2]{G'} & \meter{} & \qw\rstick{$b_2$}\\
    \lstick[wires=2]{$\ket{\Phi^+}$}\qw & & \meter{} & \qw\rstick{$b_1$} \\
    \qw & \gate[wires=1]{G(b_1, b_2)}  & \qw & \qw \rstick{$\alpha \ket{0} + \beta\ket{1}$} 
\end{quantikz}
\end{center}

where $b_1$ tells you whether to do a bit flip and $b_2$ tells you whether to do a phase flip in $G$.

Let's consider a simpler problem, Alice and Bob have two entangled bits $\alpha \ket{00} + \beta \ket{11}$ and they want Bob to create $\alpha \ket{0} + \beta \ket{1}$.
They can use the following circuit to solve this problem:

\begin{center}
\begin{quantikz}
    \lstick[wires=2]{$\alpha \ket{00} + \beta\ket{11}$}\qw & \gate{H} & \qw & \meter{} & \qw\rstick{$b_2$}\\
    \qw & \gate{ b_2 \text{ ? } Z \text{ : } I  } & \qw & \qw\rstick{$\alpha \ket{0} + \beta \ket{1}$} \\
\end{quantikz}
\end{center}

The state becomes:
\[ \frac{1}{\sqrt{2}} \qty(\alpha \ket{00} + \beta \ket{01}) + \frac{1}{\sqrt{2}} \qty(\alpha \ket{10} - \beta \ket{11}) \]
Upon measurement, we get $0$ and the state is $\alpha \ket{0} + \beta \ket{1}$, or we get $1$ and the state is $\alpha \ket{0} - \beta \ket{1}$.
She tells Bob this one bit $b_2$. If it were 1, Alice would tell Bob to apply a phase flip $Z$ to fix the quantum state.

Now, to address the original problem, we can reduce to this case by just getting $\alpha \ket{00} + \beta\ket{11}$.
The following subcircuit takes the two states and gives a $0$ if the state is $\alpha \ket{00} + \beta\ket{11}$ and a $1$
if the state is $\alpha \ket{01} + \beta\ket{10}$. In the second case, a bit flip on the second qubit gets us our target state.

\begin{center}
\begin{quantikz}
    \qw\lstick{$\alpha \ket{0} + \beta \ket{1}$} & \ctrl{1} & \qw & \qw \\
    \qw\lstick[wires=2]{$\sum_j \beta_j \ket{j, j}$} & \targ & \qw & \meter{} & \qw\rstick{$b_1$} \\
    \qw & \qw & \qw & \qw
\end{quantikz}
\end{center}

Let's prove this claim, using an index for brevity, where
the initial state of the (technically three) qubits is: $\sum_{i, j} \alpha_i \ket{i} \otimes \ket{j, j}$. Under CNOT:
\[ \text{CNOT} \sum_{i,j} \alpha_i \ket{i} \otimes \ket{j, j} = \sum_{i, j} \alpha_i \ket{i, i \oplus j, j}  \]
But then after measurement of $\ell$ in qubit 1, we have $i \oplus j = \ell \implies j = i \oplus \ell$ is the only term that survives:
\[ \sum_i \alpha_i \ket{i, i \oplus \ell} \]
But we want to end up with $\sum_i \alpha_i \ket{i, i}$. So we only have to do a bit flip on the third bit if $\ell = 1$.

Combining all these subcircuits together gives us the following circuit

\begin{center}
\begin{quantikz}
    \lstick{Alice sends $\ket{\psi} = \alpha \ket{0} + \beta\ket{1}$}\qw & \ctrl{1} & \gate{H} & \qw & \meter{} & \cwbend{2} & \rstick{$b_2$}\\
    \lstick[wires=2]{$\ket{\Phi^+}$}\qw & \targ & \qw & \qw & \meter{} & \cwbend{1} &  & \rstick{$b_1$} \\
    \qw & \qw & \qw & \qw & \gate{X} & \gate{Z} & \qw \rstick{Bob receives $\ket{\psi}$} 
\end{quantikz}
\end{center}

which allows Bob to successfully receive Alice's qubit. Using this circuit, we can teleport Alice's qubit anywhere!

Note that due to the principle of deferred measurement (covered in the next lecture), this circuit is equivalent to

\begin{center}
\begin{quantikz}
    \lstick{$\ket{\psi}$}\qw & \ctrl{1} & \gate{H} & \qw & \ctrl{2} & \qw\rstick{$b_2$}\\
    \lstick[wires=2]{$\ket{\Phi^+}$}\qw & \targ & \qw & \qw & \ctrl{1} & \qw & \qw\rstick{$b_1$} \\
    \qw & \qw & \qw & \targ & \qw & \gate{Z} & \qw \rstick{$\ket{\psi}$} 
\end{quantikz}
\end{center}