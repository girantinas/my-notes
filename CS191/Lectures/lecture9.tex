\section{Lecture 9}

\subsection{Grover's Algorithm}
We investigate the unstructured search problem. Suppose we have a boolean function $f: \{0, 1\}^n \to \{0, 1\}$, which
is guaranteed to not be the 0 function. We wish to find an $a$ such that $f(a) = 1$. Call $N = 2^n$.
One can think of a Boolean function as a table, mapping numbers from $0$ to $N - 1$. Deterministically you must brute force all entries in the worst case, so it takes $N$; 
with a randomized algorithm, on average you must look through half of them $N/2$, which is still exponential in $n$. This problem is hard (in \emph{NP-Complete}), as it's easy to
check given an $a$ that $f(a) = 1$, but it's hard to find that $a$. We discuss the hardest version of the problem, where $f$ is only 1 at a single point.

In the quantum world, we can prepare the following superposition:
\[ \frac{1}{\sqrt{N}} \sum_x \ket{x} \ket{f(x)} \]
However, when we measure, we only get $1$ with probability $\frac{1}{N}$, and we're no better than classical.
In fact, the best we can do is still exponential.
\begin{theorem}[BBBV Lowerbound]
    Any quantum algorithm for the unstructured search problem must take $\gtrsim \sqrt{N}$ time.
\end{theorem}

Grover's algorithm saturates this lower bound of $O(\sqrt{N})$. Recall that one thing we can do with quantum computation is
create a phase state: $\frac{1}{2^{n/2}} \sum_{x} (-1)^{f(x)} \ket{x}$. Consider the bit flip operator $X$;
its eigenvectors are $\ket{\pm}$ with eigenvalues $\pm 1$. Consider the following circuit which computes $f(x)$
and applies a CNOT to $\ket{f(x)}$ and $\ket{b}$:

\begin{center}
\begin{quantikz}
    \lstick{$\ket{x}$} & \gate[2]{U_f} & \qw\rstick{$\ket{x}$} \\
    \lstick{$\ket{b}$} & & \qw\rstick{$\ket{b \oplus f(x)}$}
\end{quantikz}
\end{center}

Now imagine if $\ket{b}$ is $\ket{-}$. Then, our output on the right is $(-1)^{f(x)} \ket{-}$ by linearity. This creates the phase state we want (we can throw away $\ket{-}$ since it's in a tensor product).
This leads to the following circuit, where $D$ is called the diffusion operator.

\begin{center}
\begin{quantikz}
    \lstick{$\ket{0^n}$} & \gate{H^{\otimes n}} & \gate[2]{U_f} & \gate{H^{\otimes n}}\gategroup[2,steps=3,style={dashed,
    rounded corners,fill=green!20, inner xsep=2pt},
    background,label style={label position=below,anchor=
    north,yshift=-0.2cm}]{{$D$}} & \ctrl{1} & \gate{H^{\otimes n}} & \qw & \dots \\
    \lstick{$\ket{-}$} & \qw & \qw & \qw & \targ & \qw & \qw & \qw & \dots
\end{quantikz}
\end{center}

What does it mean to control on a bunch of bits? In this case, this means that if all of the bits are 0, then leave it alone, otherwise, we flip.
Everything after the first Hadamard is considered ``one iteration" of Grover's algorithm.
We run this for $O(\sqrt{N})$ iterations, and we measure at the end. We claim that $a$ pops out with constant probability (with respect to $N$).

\begin{note}
    The diffusion operator $D$ can be expressed as follows:
    \[ D = \frac{2}{N}J - I = \begin{pmatrix}
        \frac{2}{N} - 1 & \frac{2}{N} & \cdots & \frac{2}{N} \\
        \frac{2}{N} & \frac{2}{N} - 1 & \cdots & \frac{2}{N} \\
        \vdots & \vdots & \ddots & \vdots \\
        \frac{2}{N} & \frac{2}{N} & \cdots & \frac{2}{N} - 1
    \end{pmatrix}\]
    where $J$ is the all 1's matrix.
\end{note}

Let's look at what the diffusion operator actually does. Consider 

\[\beta = \begin{pmatrix}
    \beta_0 \\
    \vdots \\
    \beta_{N-1}
\end{pmatrix} = D\alpha = D \begin{pmatrix}
    \alpha_0 \\
    \vdots \\
    \alpha_{N-1}
\end{pmatrix},\]

where each entry of $\beta$ is $\beta_i = -\alpha_i + \frac{2}{N} \sum_{j} \alpha_j$.
Then, noting that the mean is $\mu = \frac{1}{N} \sum_j \alpha_j$, we can express each of these entries as 
\[\beta_i = 2\mu - \alpha_i = \mu - (\alpha_i - \mu).\]
Now, we see that applying $D$ inverts every element about the mean. Let's use this property of $D$ to analyze the correctness of Grover's algorithm.

\subsubsection{Analysis of Grover's Algorithm}

At step 0, all the amplitudes are evenly distributed as $\frac{1}{\sqrt{N}}$. Applying $U_f$, we end up with the phase state,
which is $-1$ at $a$ and 0 everywhere else. This causes the mean of the amplitudes to decrease ever so slightly, so that after applying $D$, the rest of the amplitudes shift a small amount, and $a$'s amplitude goes up to nearly $\frac{3}{\sqrt{N}}$.
After another iteration, it goes up to $\frac{5}{\sqrt{N}}$, and it keeps on increasing by about $\frac{2}{\sqrt{N}}$
as long all the amplitudes are close to $\frac{1}{\sqrt{N}}$. In fact, we get an increase of at least $\frac{1}{\sqrt{N}}$ as long as the other amplitudes are above $\frac{1}{2\sqrt{N}}$. 

Thus, after $O(\sqrt{N})$
steps, the amplitude of $a$ takes up a constant fraction of the norm (in particular, the rest take up 
probability mass $\qty(\frac{1}{2\sqrt{N}})^2 \cdot (N - 1) \leq \frac{1}{4N} \cdot N = \frac{1}{4}$).
This means that we can measure $a$ with constant probability (at least $\frac{3}{4}$), as desired.

\subsubsection{Geometric Interpretation of Grover's Algorithm}
Here is a geometric way to view what Grover's algorithm does. Consider the two-dimensional subspace spanned by
$\ket{a}$ and $\ket{u} = \sum_x \frac{1}{\sqrt{N}} \ket{x}$. Note that these two vectors are not orthogonal, so we introduce $\ket e$, which is a vector orthogonal to $\ket{a}$ in this plane. 
Now, we can interpret this plane to have basis $\{\ket e, \ket a\}$ (i.e. the $x, y$ axes), and $\ket u$ is a vector lying within this 2D coordinate system. Then, geometrically, this is what Grover's does:
\begin{enumerate}
    \item Start in the state $\ket u$.
    \item Reflect about $\ket e$ (i.e. applying $U_f$).
    \item Reflect about $\ket u$ (i.e. applying $D$).
    \item Repeat steps 2-3 for each subsequent iteration.
\end{enumerate}
% We start in the state $\ket{u}$,
% and the $U_f$ flipping step reflects about $\ket{e}$. Then, we only institute what is essentially a phase flip about $\ket{u}$. 
One can think of steps 2-3 as walking $\ket{u}$ towards $\ket{a}$. Initially, the angle between $\ket{u}$ and $\ket{e}$ is $\theta = \sin^{-1} \frac{1}{\sqrt{N}} \approx \frac{1}{\sqrt{N}}$.
After one iteration, the new angle becomes $\theta + 2 \theta = 3 \theta$, and each subsequent iteration increases the angle by $2\theta$. Thus, in order to get to $\theta^{\prime} = \frac{\pi}{2}$ (which corresponds to $\ket a$), it takes approximately
\[\frac{\pi /2}{2 \theta} \approx \frac{\pi / 2}{2 / \sqrt{N}} = \frac{\pi}{4} \sqrt{N} = O(\sqrt{N})\]
iterations. 

\subsection{Vaidman's Bomb}
Suppose there is a quantum-superposition of bomb. If we look inside and there is a bomb, then it will blow up,
but if it's safe, then we want to open it. Consider sending a photon through it with state $\ket{\psi} = \cos k\theta \ket{0} + \sin k\theta \ket{1}$,
where the $\ket{0}$ state is a photon that passes through without looking at the bomb and $\ket{1}$ state is a photon that passes through while checking for the bomb.
You send this state through the box. If we measure the photon on the outside and there is a bomb,
the probability of an explosion is $\sin^2 \theta$. If there isn't a bomb, then for sure we will not explode. We repeat this $N$ times by rotating by $\theta$ such that $N \theta = \frac{\pi}{2}$.
Then the $N$th measurement is definite and tells you there is no bomb. The probability that we explode every time is at most $\theta^2$; taking a union
bound makes it at most $N \theta^2 = O(1/N)$.

This means intuitively, we make $1/N$ progress towards the probability 1 every time.