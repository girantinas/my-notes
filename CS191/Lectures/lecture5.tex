
\section{Lecture 5}

The midterm is coming next Tuesday. It should serve as a checkpoint for what we've learned so far.
The material in scope is the first 4 lectures plus the Hadamard transform we will define today.

\subsection{Quantum Circuits and Quantum Algorithms}
We will now begin quantum computer science in an algorithmic sense.

\begin{definition}[Quantum Circuit]
    A quantum circuit on $n$ qubits is a collection of $m$ gates (unitary transforms), with measurement at the end.

        \begin{quantikz}
            \lstick[wires=2]{$n$ qubits}& \qwbundle[
                alternate]{}& \gate[swap]{}\gategroup[2,steps=3,style={dashed,
            rounded corners,fill=blue!20, inner xsep=2pt},
            background,label style={label position=below,anchor=
            north,yshift=-0.2cm}]{{\sc circuit}}\qwbundle[
                alternate]{} & \gate{}\qwbundle[
                    alternate]{} & \qwbundle[
                alternate]{} & \qwbundle[
                    alternate]{} & \meter{} \\
            & \qwbundle[
                alternate]{} & \qwbundle[alternate]{} & \gate{}\qwbundle[alternate]{} & \gate{}\qwbundle[alternate]{} & \qwbundle[alternate]{} & \meter{}
        \end{quantikz}

    The depth $d$ of a quantum circuit is the maximum amount of gates any one qubit is passed through.
    At the end of it, we measure in the standard basis in some qubits.
\end{definition}

One way to think about a quantum circuit is ``rotating'' a state a lot of times and then measuring in the standard basis,
or we could think about a quantum circuit as measuring in some ``rotated'' basis.
In the quantum world, ``programming'' is really just designing such a quantum circuit. In a typical case,
we want $m = O(\text{poly}(n))$, i.e. on the order of some polynomial of the number of qubits.

Suppose we had a (classical) circuit that could compute a Boolean function $f: \{0, 1\}^n \to \{0, 1\}^k$:

\begin{quantikz}
    \lstick{$x$} & \gate{C_f} \qwbundle[
    alternate]{}& \qwbundle[alternate]{}\rstick{$f(x)$} &
\end{quantikz}

How can we create such a circuit in the quantum world? Well first, we'd like:

\begin{quantikz}
    \lstick{$\ket{x}$} & \gate{U_f} \qwbundle[
    alternate]{}& \qwbundle[alternate]{}\rstick{$\ket{f(x)}$} &
\end{quantikz}

So by linearity, \[U_f\qty(\sum_x \alpha_x \ket{x}) \to \sum_x \alpha_x \ket{f(x)}\]

Remember that in our analysis here, since $f$ is a Boolean function,
we assume that $x$ is a classical state, i.e. $\ket{x_1, x_2, \dots, x_n}$ with $x_i \in \{0, 1\}$.
Furthermore, such a circuit is always invertible (since it's always unitary!):

\begin{quantikz}
    \lstick{$\ket{f(x)}$} & \gate{U_f^{\dagger}} \qwbundle[alternate]{}& \qwbundle[alternate]{}\rstick{$\ket{x}$} &
\end{quantikz}

But this is clearly a problem! Suppose $f(x)$ was not injective, i.e. there existed two inputs $x \neq x'$
such that $f(x) = f(x')$. Clearly you cannot ``go backwards.'' This makes us think that maybe all Boolean functions
are not representable as quantum circuits. Thus we can only do it for $f$ that are bijections! If we put the input included with the output,
then we can go ahead and do this. Note that this does not break no-cloning because we are only cloning a pure binary state $x \in \{0, 1\}^n$.

\begin{quantikz}
    \lstick{$\ket{x}$} & \gate[2]{U_f} \qwbundle[alternate]{}& \qwbundle[alternate]{}\rstick{$\ket{x}$} & \\
    \lstick{$\ket{0}$} & \qwbundle[alternate]{} & \qwbundle[alternate]{}\rstick{$\ket{f(x)}$} &
\end{quantikz}

But note that all gates we use must also be reversible in the same. Recall
that any classical circuit can be produced from AND and NOT gates. The NOT gate is invertible, and has
quantum analogue $X$. But the AND gate is obviously not reversible.

However, it's very easy to make a reversible AND gate, we just use the same trick:
just keep the inputs at the output. However, in practice, the more elegant way to do this is the CSWAP gate.

\begin{quantikz}
    \lstick{$a$} & \qw\gategroup[3,steps=1,style={dashed,
    rounded corners,fill=blue!20, inner xsep=2pt},
    background,label style={label position=below,anchor=
    north,yshift=-0.2cm}]{{\sc cswap }} & \qw\rstick{$a'$}  \\
    \lstick{$b$} & \gate[swap]{} & \qw\rstick{$b'$} \\
    \lstick{$c$} & & \qw\rstick{$c'$}
\end{quantikz}

If $a = 0$, then $b' = b$ and $c' = c$, but if $a = 1$, $b' = c$ and $c' = b$. We claim this implements
and AND gate if we set $c = 0$. Then, $c' = 0$ if $a = 0$ and $c' = b$ if $a = 1$, i.e. $c' = a \land b$.
Furthermore, we can even implement a fanout, to ``clone'' an input.
One can see through a calculation that if $b = 1, c = 0$, then $a' = a, b' = \bar{a}, c = a$.

Note that we do not want a lot of these bits to actual AND them, they are useless to us. So we can add some ``garbage'' bits
to the output:
\begin{quantikz}
    \lstick{$\ket{x}$} & \gate[3]{U_f} \qwbundle[alternate]{}& \qwbundle[alternate]{}\rstick{$\ket{x}$} & \\
    \lstick{$\ket{0}$} & \qwbundle[alternate]{} & \qwbundle[alternate]{}\rstick{$\ket{f(x)}$} & \\
    \lstick{$\ket{0}$} & \qwbundle[alternate]{} & \qwbundle[alternate]{}\rstick{$\ket{g(x)}$} &
\end{quantikz}

But this is a disaster! Why? We shall explain soon. First, let's discuss how to fix it.
To fix this, we can just copy $\ket{f(x)}$, and then run $U_f^{\dagger}$ on everything else (including the original $\ket{f(x)}$).
This eliminates the garbage $g(x)$ (the output of the $U_f^{\dagger}$ is the just input padded with 0's), while preserving $\ket{f(x)}$.

\subsection{Principle of Deferred Measurement}
Let's go back to a basic quantum circuit.

\begin{quantikz}
    \lstick{$\ket{+}$} & \gate{H} & \meter{} & \qw\rstick{$0 \text{ w.p. } 1$}
\end{quantikz}

This is because:
\[ \ftwo \ket{0} + \ftwo \ket{1} \mapsto_H \ftwo\qty(\ftwo \ket{0} + \ftwo \ket{1}) + \ftwo\qty(\ftwo \ket{0} - \ftwo \ket{1}) = \ket{0} \]

Now, let's go to a more complicated circuit:

\begin{quantikz}
    \lstick{$\ket{+}$} & \ctrl{1} & \gate{H} & \meter{} & \qw\rstick{$0 \text{ w.p. } ???$} \\
    \lstick{$\ket{0}$} & \targ & \qw & \qw
\end{quantikz}

The output of the CNOT is fed into the $H$:
\begin{align*}
    \ftwo \ket{00} + \ftwo \ket{11} &\mapsto_H \ftwo \qty(\ftwo \ket{00} + \ftwo \ket{10}) + \ftwo \qty(\ftwo \ket{01} - \ftwo \ket{11} ) \\
    &= \frac12 \ket{00} + \frac12 \ket{10} + \frac12 \ket{01} - \frac12 \ket{11}
\end{align*}
Now we don't have the interference that happened before of the $\ket{1}$ cancelling out. So now, we see 0 and 1 with equal probability.

Let's generalize this.

\begin{theorem}[Principle of Deferred Measurement]
After qubits stop interacting, the state is the same as if you measured them instead. E.g., the following two circuits will conduct equivalent measurements on the beginning bits:

\begin{quantikz}
    \qwbundle[alternate]{} & \gate[2]{U} \qwbundle[alternate]{}& \gate{U'}\qwbundle[alternate]{} & \meter{}\qwbundle[alternate]{} \\
    \qwbundle[alternate]{} & \qwbundle[alternate]{} & \meter{}\qwbundle[alternate]{} &
\end{quantikz}


\begin{quantikz}
    \qwbundle[alternate]{} & \gate[2]{U} \qwbundle[alternate]{}& \gate{U'}\qwbundle[alternate]{} & \meter{}\qwbundle[alternate]{} \\
    \qwbundle[alternate]{} & \qwbundle[alternate]{} & \qwbundle[alternate]{} & \qwbundle[alternate]{}
\end{quantikz}

\end{theorem}

So now, going back to the garbage bits, we cannot just ``throw them away'' or measure them. This will mean you can no longer use those bits for computation!
As quantum information scientists, we have to keep our workspace ``clean,'' so to speak.

\subsection{Hadamard Transform}
Let's send a $n$-bit classical bit state $\ket{u}$ into a bunch of Hadamards tensored together. Call $H^{\otimes n} = H \otimes H \otimes \dots \otimes H$ (n times).
At the level of a singular qubit:
\[ H\ket{u_1} = \sum_{y \in \{0, 1\}} \frac{(-1)^{u_1 \cdot y}}{\sqrt{2}} \ket{y} \]
which can clearly be seen by the definition (we only have a negative sign if both $u$ and $y$ are 1). Similarly, by nearly ``squaring'' the above expression:
\[ (H \otimes H)(\ket{u_1, u_2}) = \sum_{\mathbf{y} \in \{0, 1\}^2} \frac{(-1)^{u_1 y_1} (-1)^{u_2 y_2}}{\qty(\sqrt{2})^2} \ket{y}\]

Thus, we can write:

\begin{quantikz}
    \lstick[wires = 3]{$\ket{\mathbf{u}}$} & \gate{H}\gategroup[3,steps=1,style={dashed,
    rounded corners,fill=blue!20, inner xsep=2pt},
    background,label style={label position=below,anchor=
    north,yshift=-0.2cm}]{{\sc Had }} & \rstick[wires = 3]{$\sum_{\mathbf{y} \in \{0, 1\}^n} \frac{(-1)^{\mathbf{u} \cdot \mathbf{y}}}{2^{n/2}}$} \\
    \qw & \gate{\vdots} & \qw \\
    \qw & \gate{H} & \qw 
\end{quantikz}

Lastly, one can compute the tensor product using the formula from last lecture:
\[ H \otimes H = \frac12 \begin{bmatrix}
    1 & 1 & 1 & 1 \\ 1 & -1 & 1 & -1 \\ 1 & 1 & -1 & -1 \\ 1 & -1 & -1 & 1
\end{bmatrix}\]

One can continue this with higher order products.






