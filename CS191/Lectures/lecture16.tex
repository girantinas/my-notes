\section{Lecture 16}

\subsection{General Conditions for Quantum Error Correction}
We start with a $\mathcal{H}_{code}$. This is the set of the actual quantum information
we want to protect. It is spanned by logical qubits $\{\ket{\tilde{i}}\}$ and we wish to embed this
Hilbert space into the physical qubits space $\H_{phys}$. Then we have an alphabet of errors:
$\Sigma = \{ I, E_1, E_2, E_3, \dots \}$, which we can think of as unitaries in the non-noisy setting or Kraus operators
in the quantum channel setting. For the Shor code, the $E_i$ were the 1-qubit Pauli operators. We
want to correct any error $E = \sum_i \alpha_i E_i$. When can we correct these errors?

For error correction to be able to happen,
we must have for all $L, K$ that if $\braket{\tilde{i}}{\tilde{j}} = 0$
then $\mel{\tilde{i}}{E_K^{\dagger} E_L}{\tilde{j}} = 0$. But this is not a sufficient condition. Namely
take the set of errors where:
\[ E_1 \ket{\tilde{0}} = E_2 \ket{\tilde{0}}, E_1 \ket{\tilde{1}} \perp E_2 \ket{\tilde{1}} \]
To recover $\ket{\tilde{1}}$ we need to do a syndrome measurement to see whether $E_1$ or $E_2$ occurred.
But because $E_1 \ket{\tilde{0}} = E_2 \ket{\tilde{0}}$, intuitively any measurement that does so will affect superpositions
$E_i\qty(\alpha \ket{\tilde{0}} + \beta \ket{\tilde{1}})$.

A stronger condition would be $ \mel{\tilde{i}}{E_K^{\dagger} E_L}{\tilde{j}} = \delta_{ij} \delta_{KL}$.
This is sufficient because we can measure perfectly which error $E_K$ actually occurred without affecting the logical state.
The projector would be onto the subspace $\textrm{span}\{E_K \ket{\tilde{i}}\}$. Any code that does this is called a 
\textbf{Nondegenerate code}.
But this is not necessary,
the Shor code doesn't follow this. For example, $Z_1 \ket{\tilde{\tilde{0}}} = Z_2 \ket{\tilde{\tilde{0}}}$. 

\begin{theorem}[Knill-Laflamme Conditions]
    Quantum error-correction is possible if and only if:
    \[ \mel{\tilde{i}}{E_K^{\dagger} E_L}{\tilde{j}} = \delta_{ij} M_{KL} \]
    in particular, the indices $i$ and $j$ must not affect $M_{KL}$.

    % average over all results of the measurement? what does this mean?
    \begin{proof*}
        \textbf{Only if.}
        Consider a quantum channel $\mathcal{N}(\rho) = \frac{1}{|\Sigma|} \sum_k E_k \rho E_k^{\dagger}$
        which has a uniform distribution over all errors. Recall the Stinespring representation,
        which wrote $\mathcal{N}(\rho) = \tr_E(V \rho V^{\dagger})$ where
        \[ V \ket{\tilde{i}} = \frac{1}{\sqrt{|\Sigma|}} \sum_k E_k \ket{\tilde{i}} \ket{k}_E \]
        Consider $\mathcal{R}$ be the recovery channel, which is both the syndrome measurement and the
        error-correcting step. We describe it with some Kraus operators $\sum_\mu R_{\mu}^{\dagger} R_{\mu} = I$.
        We can now give another isometry for the $\mathcal{R}$ channel as $V_R$ with environment $E'$:
        \[ V_R V \ket{\tilde{i}} = \frac{1}{\sqrt{\Sigma}} \sum_{k, \mu} R_\mu E_k \ket{\tilde{i}} \ket{k}_E \ket{\mu}_{E'}\]
        Then the composition of the two channels is:
        \begin{align*}
            \mathcal{R} \circ \mathcal{N}(\rho) &= \sum_{k, \mu} R_{\mu} E_k \rho E_k^{\dagger} R_{\mu}^{\dagger}
        \end{align*}
        If we succeeded, the final state of the system must be $\ket{\tilde{i}}$.
        \begin{align*}
            \sum_{k, \mu} R_\mu E_k \ket{\tilde{i}} \ket{k}_E \ket{\mu}_{E'} &= \ket{\tilde{i}} \ket{\chi}_{E \otimes E'} \\
            &= \ket{\tilde{i}} \sum_{k, \mu} \lambda_{k, \mu} \ket{k} \ket{\mu} \\
            R_{\mu} E_k \ket{\tilde{i}} &= \lambda_{k, \mu} \ket{\tilde{i}} 
        \end{align*}
        Now taking an inner product,
        \[ \sum_{\mu} \mel{\tilde{j}}{E_K^{\dagger} R_{\mu}^{\dagger} R_{\mu} E_L}{\tilde{i}} = \delta_{ij} \sum_{\mu} \lambda_{K, \mu}^* \lambda_{L,\mu}  = \delta_{ij} M_{KL}\]
        But the sum can be brought inside and the $R_{\mu}$ are normalized, the LHS is also $\mel{\tilde{j}}{E_K^{\dagger} E_L}{\tilde{i}}$.
        This proves that it's necessary.
        
        \textbf{If.} We will prove this by explicit construction of $R_{\mu}$, the recovery operator. Consider the matrix $M_{K, L}$.
        Let $\hat{e}_{\mu} = \sum_K e_{\mu, K} \hat{f}_K$, where the latter are the basis for the error space and the former
        are the eigenvectors of $M_{K, L}$. Further, let $c_{\mu}$ be the corresponding eigenvalues. Now, define $M_{\mu} = \sum_k e_{\mu, k} E_k$.
        Writing the result in terms of $M_{\mu}$:
        \begin{align*}
            \mel{\tilde{i}}{M_{\mu}^{\dagger} M_{\nu}}{\tilde{j}} = \delta_{ij} \delta_{\mu \nu} c_{\mu}
        \end{align*}
        Now, define
        \[ R_{\mu}  = \frac{1}{\sqrt{c_{\mu}}} \sum_i \ketbra{\tilde{i}} M_{\mu}^{\dagger} \]
        Suppose some error happened. Now the state is some superposition of errors $E_k$, which
        is also a superposition of $M_{\mu}$. The state then becomes:
        \[ \sum_{\mu} M_{\mu} \ket{\tilde{i}} \ket{\phi_{\mu}} \]
        which upon recovery becomes
        \begin{align*}
            &\sum_{\nu, \mu} R_{\nu} M_{\mu} \ket{\tilde{i}} \ket{\phi_{\mu}} \ket{\nu}
            &= \sum_{\nu, \mu, j} \frac{1}{c_{\nu}} \ketbra{\tilde{j}} M_{\nu}^{\dagger} M_{\mu} \ket{\tilde{i}} \ket{\phi_{\mu}} \ket{\nu} \\
            &= \sum_{\nu, \mu, j} TODO
            &= \sum_{\mu} \sqrt{c_{\mu}} \ket{\tilde{i}} \ket{\phi_{\mu}} \ket{\mu}
        \end{align*}
        Now, we just need to show that this is actually a channel.
        \begin{align*}
            &\sum_{\mu} R_{\mu}^{\dagger} R_{\mu}= \sum_{\mu, i, j} \frac{1}{c_{\mu}} M_\mu \ketbra{\tilde{i}} \ketbra{\tilde{j}} M_{\mu}^{\dagger} \\
            &= \sum_{\mu, i} \frac{1}{c_{\mu}} M_{\mu} \ketbra{\tilde{i}} M_{\mu}^{\dagger} \\
            &= \textrm{Proj}(\textrm{span}\{ M_{\mu} \ket{\tilde{i}}\})
        \end{align*}
        We can pad this with the identity by doing $R_0 = I - \sum_{\mu \geq 1} R_{\mu}^{\dagger} R_{\mu}$.
        Note that this operator does not affect states that we care about, since they live in that subspace spanned by the
        other $R_{\mu}$.
    \end{proof*}
\end{theorem}

\subsection{Stabilizer Codes}
Stabilizer codes are a very general way to come up with quantum error-correcting codes. We will first introduce the Pauli group.
\begin{definition}
    The Pauli group on one qubit is defined as $\mathcal{G}= \pm \{I, X, iY, Z\}$.

    The Pauli group on $n$ qubits is defined as $\mathcal{G}_n = \pm \{I, X, iY, Z\}^{\otimes n}$. There are $2^{2n + 1}$ elements (the plus and minus are at the end).
\end{definition}
These properties can be proven simply using casework.
\begin{theorem}
    The Pauli group $\mathcal{G}_n$ on $n$ qubits satisfies the following properties:
    \begin{enumerate}
    \item The Pauli group is indeed a group under matrix multiplication.
    \item For $M, N \in \mathcal{G}_n$, either $[M, N] := MN - NM = 0$,
    or $\{M, N\} := MN + NM = 0$. They either commute or anti-commute.
    \item For $M \in \mathcal{G}_n$, either $M^2 = I$ and $M = M^{\dagger}$, or $M^2 = I$ and $M = -M^{\dagger}$.
    \end{enumerate}
\end{theorem}
Now we are ready to define stabilizer codes. Choose a set $\{ M_i \} \subseteq \mathcal{G}_n$. For all $i, j$,
we want $[M_i, M_j] = 0$ and $M_i = M_i^{\dagger}$. This means the eigenvalues must be all $\pm 1$ and
there exists a simultaneous set of eigenstates that diagonalize them. These will generate a subgroup $\langle M_i \rangle = S \subset \mathcal{G}_n$
which is Abelian. For $M \in S$, Unless we take $M = I$, half of the eigenvalues of $M$ are 0 and half of them are 1 (they must have trace 0).
We will define $\H_{code} = \{ \ket{\psi} : \forall i M_i \ket{\psi} = \ket{\psi} \}$. The dimension of the
entire space $\H_{phys}$ is $2^n$. One can show each condition in $\mathcal{H}_{code}$ halves the dimension, giving
$2^{n - k}$, where $k = |\{M_i\}|$ is the number of generators. Now, as syndrome measurements,
we measure all the $M_i$. When we measure a $-1$, we have to correct that error, otherwise we do not.
\begin{example}
    The Shor code is actually an example of a stabilizer code. The syndrome measurements were 
    \[ \{ Z_1 Z_2, Z_2 Z_3, Z_4 Z_5, Z_5 Z_6, Z_7 Z_8, Z_8 Z_9, X_1 X_2 X_3 X_4 X_5 X_6, X_4 X_5 X_6 X_7 X_8 X_9 \}\]
    Note that all of these are members of the Pauli group and they all commute (all even numbers of $X$ and $Z$).

    \textbf{Exercise:} Show that $\ket{\tilde{\tilde{0}}}, \ket{\tilde{\tilde{1}}}$ are the only +1 eigenstates of this snydrome measurement.
\end{example}
