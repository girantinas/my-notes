\part{Quantum Computing and Algorithms}
\section{Lecture 1}

\subsection{Motivating Quantum Computing}
The classical unit of computation is a \textbf{bit}. How small can we shrink bits? Let's conduct a thought experiment. Let's suppose we could shrink them down to the size of a Hydrogen atom. The ``state''
of $\ket{0}$ being the ground state and $\ket{1}$ first excited state. However, electrons in general exist in superposition states! These states look like:
\[ \{ \alpha \ket{0} + \beta \ket{1}: \alpha, \beta \in \C, |\alpha|^2 + |\beta|^2 = 1 \} \]
But it gets weirder. According to quantum theory, when conducting a measurement on such a state, we end up getting:
\[ M = \begin{cases}
   0 & \text{wp } |\alpha|^2 \\
    1 & \text{wp } |\beta|^2 \\
\end{cases} \]
Furthermore, the act of measurement ``collapses'' the wavefunction to a state $\ket{0}$ or $\ket{1}$. Subsequent measurements will give that pure state deterministically.

Now, suppose we have a system of two such Hydrogen. There are now 4 basis states:
\[ \ket{\psi} = \alpha_{00} \ket{00} + \alpha_{01} \ket{01} + \alpha_{10} \ket{10} + \alpha_{11} \ket{11} \]
Effective computation now comes from extrapolating to $n$ such \textbf{qubits}. Now such a state would look like $\sum_{x \in \{0, 1\}} \alpha_x \ket{x}$.

This is pretty profound. Classical computers were designed to use nature (through silicon) in order to work for humans. But with all
this effective work that nature is doing behind the scenes, it seems that quantum computing is really the more powerful framework we should've asked for.

\subsection{Measurement}
Now suppose we do a ``partial'' measurement, e.g. only measuring the first bit. What will we get? It seems reasonable that the probability should be the sum of the probabilities of getting a 0 in the first qubit,
e.g. we get a 0 w.p. $|\alpha_{00}|^2 + |\alpha_{01}|^2$. The state collapses, but it must be renormalized so the coefficients can still be probabilities!
So the new state is actually \[\frac{\alpha_{00} \ket{00} + \alpha_{01} \ket{01}}{\sqrt{|\alpha_{00}|^2 + |\alpha_{01}|^2}}\]

Now suppose we are given a qubit in the state
\[ \ket{\psi} = \frac{1}{\sqrt{2}} \ket{0} + \frac{e^{i \theta}}{\sqrt{2}} \ket{1} \]
How can we figure out $\theta$ (phase estimation)? Well if we measure this, we will get either $0$ or $1$ with probability $1/2$ each. This will
tell us nothing about $\theta$. It turns out this is only a special case of measurement.

To understand what general measurement is, we first go back to our state representation.
What we really mean by a superposition is a linear combination of two vectors. We fix some basis $\ket{0}$ and $\ket{1}$, and a normalized state $\ket{\psi} = \alpha \ket{0} + \beta \ket{1}$
is a unit vector in a 2-dimensional complex vector space. Now we can think about a measurement in the following way:

\begin{definition}
    A \textbf{measurement} of some state $\ket{\phi}$ in some basis $\mathcal{U}$ is a projection onto one of the basis vectors $\ket{u}$. The value of the measurement is:
    $u$ with probability of the scalar projection squared, $\qty|\frac{\braket{u}{\psi}}{\braket{\psi}{\psi}}|^2$.
\end{definition}

So for example, let's stick to our 2-space and pick a new orthonormal basis $\{\ket{u}, \ket{u^{\perp}}\}$ and our state $\ket{\psi}$. Suppose $\ket{\psi}$ makes an angle of $\theta$ with the $\ket{0}$ axis and
makes an angle of $\mu$ with the $\ket{u}$ axis. By a simple diagram, it's clear from $\psi$'s projections that measurement in the standard basis yields $0$ with probability $\cos^2 \theta$ and in our new basis it yields $u$ with probability $\cos^2{\mu}$.

\begin{note}
    There is a bit of a subtlety here. We assumed that the amplitudes we are working with are real, but in general they can be complex. It turns out, all
of quantum computing can be formalized with only real amplitudes, but it gets more messy when interfacing with physics. For now,
we will assume real amplitudes only, but most results generalize to complex amplitudes.
\end{note}

Another common example of a basis is:
\[ \ket{+} = \ftwo \qty(\ket{0} + \ket{1}), \ket{-} = \ftwo \qty(\ket{0} - \ket{1}) \]

Measuring our original phase estimation in this new basis is exactly what we need! We just need to write it in the new
basis to figure out the amplitudes:
\begin{align}
    \ftwo \ket{0} + e^{i \theta} \ftwo \ket{1} &= \ftwo \qty(\ftwo \ket{+} + \ftwo \ket{-}) + e^{i\theta} \ftwo \qty(\ftwo \ket{+} - \ftwo \ket{-}) \\
    &= \frac{1}{2} (1 + e^{i \theta}) \ket{+} + \frac{1}{2} (1 - e^{i \theta}) \ket{-} \\
    &= \frac{1}{2} \qty(1 + \cos \theta + i \sin \theta) \ket{+} + \dots \\
\end{align}
so we get $\ket{+}$ from the measurement with probability
\[ \frac{1}{2} |1 + \cos \theta + i \sin \theta|^2 = \cos^2 \qty(\theta/2) \]
Now we can repeat the measurement (with other processed inputs) to get statistics and thus a good estimate on $\theta$.