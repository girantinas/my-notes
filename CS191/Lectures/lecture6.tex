
\section{Lecture 6}

\subsection{Hadamard Transform (Continued)}
Recall the Hadamard Transform from last lecture, made up of a bunch of 
Hadamard gates tensored together.

\begin{center}
\begin{quantikz}
    \lstick[wires = 3]{$\ket{\mathbf{u}}$} & \gate{H}\gategroup[3,steps=1,style={dashed,
    rounded corners,fill=blue!20, inner xsep=2pt},
    background,label style={label position=below,anchor=
    north,yshift=-0.2cm}]{{\sc Had }} & \rstick[wires = 3]{$\sum_{\mathbf{y} \in \{0, 1\}^n} \frac{(-1)^{\mathbf{u} \cdot \mathbf{y}}}{2^{n/2}} \ket{y}$} \\
    \qw & \gate{\vdots} & \qw \\
    \qw & \gate{H} & \qw 
\end{quantikz}
\end{center}

Recall that such a $\ket{u} \in \{0, 1\}^n$ is a computational basis vector. So in the standard basis,
it has a 1 in some position, and a 0 everywhere else. We will call this position the $u$th position.
Thus, this picks out the $u$th column of $H^{\otimes n}$. This means the $u$th column is just $\frac{(-1)^{\mathbf{u} \cdot \mathbf{y}}}{2^{n/2}}$ for all possible $y$.
Therefore, we have
\[[H^{\otimes n}]_{y, u} = \frac{(-1)^{\mathbf{u} \cdot \mathbf{y}}}{2^{n/2}}.\]
Note that these are just bitstrings, but when we use them as indices, we convert them to their decimal counterparts (and use 0-indexing).

One can view this as a Discrete Fourier Transform over $\Z_2^n$, which means it will be very nice for use later. Let's revisit the special case of 2x2.

\[
H^{\otimes 2} = \frac12 \begin{pmatrix}
1 & 1 & 1 & 1 \\
1 & -1 & 1 & -1 \\
1 & 1 & -1 & -1 \\
1 & -1 & -1 & 1 \\
\end{pmatrix}
\]

Let's think about how to act on one of our favorite states, $\Phi^+ = \ftwo \ket{00} + \ftwo \ket{11}$.
The $\ket{00}$ maps to the first column, $\frac12 \qty(\ket{00} + \ket{01} + \ket{10} + \ket{11})$ and
the $\ket{11}$ maps to the last column, $\frac12 \qty(\ket{00} - \ket{01} - \ket{10} + \ket{11})$.
By linearity, this means:
\[ H^{\otimes 2}\Phi^+ = \ftwo \frac12 \qty(\ket{00} + \ket{01} + \ket{10} + \ket{11}) + \ftwo \frac12 \qty(\ket{00} - \ket{01} - \ket{10} + \ket{11}) = \ftwo \ket{00} + \ftwo \ket{11}   \]
That means this Bell state through the Hadamard transform just returns the same thing.

With some similar algebra, it turns out another Bell state can be mapped as:
\[ H^{\otimes 2}\qty(\ftwo \ket{01} + \ftwo \ket{10}) = \ftwo \ket{00} - \ftwo \ket{11} \]

We can use the Hadamard to do something interesting:
\begin{center}
\begin{quantikz}
    \lstick{$\sum_{x} \alpha_x \ket{x}$} & \gate{Had} & \meter{}
\end{quantikz}
\end{center}

Suppose the state before measurement is $\sum_y \beta_y \ket{y}$. Then measuring in the standard basis gives $y$ with probability $|\beta_y|^2$.
The question is, what are some interesting input states we can put into the Hadamard? Thinking about this question lays the groundwork for quantum algorithms.

\subsection{Building Blocks for Quantum Algorithms}
How do we program a quantum computer? Let's again try to emulate a classical computer. We saw last lecture, we
can emulate a classical circuit $C_f$ using $U_f$:

\begin{center}
\begin{quantikz}
    \lstick{$\ket{x}$} & \gate[3]{U_f} \qwbundle[alternate]{}& \qwbundle[alternate]{}\rstick{$\ket{x}$} & \\
    \lstick{$\ket{0}$} & \qwbundle[alternate]{} & \qw\rstick{$\ket{f(x)}$} & \\
    \lstick{$\ket{0}$} & \qwbundle[alternate]{} & \qwbundle[alternate]{}\rstick{$\ket{0}$} &
\end{quantikz}
\end{center}

Now, suppose given a function $f: \{ 0,1\}^n \to \{0,1\}$, we want to make a circuit that produces:
\[ \sum_x \frac{(-1)^{f(x)}}{2^{n/2}} \ket{x} \]
Our first attempt is:

\begin{center}
\begin{quantikz}
    \lstick{$\ket{0^n}$}\qwbundle[alternate]{} & \gate{H^{\otimes n}}\qwbundle[alternate]{} & \gate[3]{U_f}\qwbundle[alternate]{} & \qwbundle[alternate]{} \rstick[wires=2]{$\sum_x \frac{1}{2^{n/2}} \ket{x} \ket{f(x)}$}   \\
    \lstick{$\ket{0}$} & \qwbundle[alternate]{} & \qwbundle[alternate]{} & \qw\\
    \lstick{$\ket{0}$} & \qwbundle[alternate]{} & \qwbundle[alternate]{} & \qwbundle[alternate]{} \rstick{$\ket{0}$}
\end{quantikz}
\end{center}

We're close, but we need to somehow obtain the $(-1)^{f(x)}$. We can do this by applying a phase flip $Z$ to the middle $\ket{f(x)}$ bit, which negates every term where $f(x) = 1$, i.e.,
the desired output becomes:
\[ \sum_x \frac{(-1)^{f(x)}}{2^{n/2}}\ket{x} \ket{f(x)} \]
Now, to erase $\ket{f(x)}$, we can do the usual trick of running $U_f^{\dagger}$ on everything to undo the transform. Then, we measure to yield our intended answer (as $U_f^{\dagger} \ket x \ket{f(x)} = \ket x$). The circuit that puts this all together is depicted below:

\begin{center}
    \begin{quantikz}
        \lstick{$\ket{0^n}$}\qwbundle[alternate]{} & \gate{H^{\otimes n}}\qwbundle[alternate]{} & \gate[3]{U_f}\qwbundle[alternate]{} & \qwbundle[alternate]{} & \gate[3]{U_f^{\dagger}}\qwbundle[alternate]{} & \qwbundle[alternate]{}\rstick[2]{$\sum_x \frac{(-1)^{f(x)}}{2^{n/2}} \ket{x}$}  \\
        \lstick{$\ket{0}$} & \qwbundle[alternate]{} & \qwbundle[alternate]{} & \gate{Z} & \qw & \qw\\
        \lstick{$\ket{0}$} & \qwbundle[alternate]{} & \qwbundle[alternate]{} & \qwbundle[alternate]{} & \qwbundle[alternate]{} & \qwbundle[alternate]{}\rstick{$\ket{0}$}
    \end{quantikz}
\end{center}

\subsection{Motivation for Quantum Algorithms}

Before building up quantum algorithms, we thought about the Extended Church-Turing Thesis, which roughly stated that any ``reasonable'' model of computation is
polynomial-time simulable on a (probabilistic) Turing Machine. Clearly, simulating $n$ qubits with a classical computer requires tracking $2^n$ amplitudes,
which is an exponential-time process. However, quantum computers violate this thesis and early quantum algorithms showcased this fact. Recently,
Google performed a ``Quantum supremacy'' experiment, where they argued there was a \textbf{experimental} speedup from a classical computer running the same task.

\subsection{Bernstein-Vazirani Algorithm}
Suppose there is a secret function $f(x) = u \cdot x$ where we don't know $u$,
but we have oracle access to it. The algorithm to solve this can be stated simply.

\begin{enumerate}
    \item Create a phase state $\sum_x \frac{(-1)^{f(x)}}{2^{n/2}} \ket{x} = \sum_x \frac{(-1)^{u \cdot x}}{2^{n/2}} \ket{x}$
    \item Feed it through the Hadamard Transform (it is its own inverse), and then measure. This will yield $u$.
\end{enumerate}

The quantum circuit that performs this algorithm is depicted as follows:

\begin{center}
    \begin{quantikz}
        \lstick{$\ket{0^n}$}\qwbundle[alternate]{} & \gate{H^{\otimes n}}\qwbundle[alternate]{} & \gate{U_f}\qwbundle[alternate]{}\ctrl{1} & \gate{H^{\otimes n}}\qwbundle[alternate]{} & \qwbundle[alternate]{}\rstick{$\ket u$} \\
        \lstick{$\ket{-}$} & \qw & \targ & \qw & \qw & \qw\rstick{$\ket -$}
    \end{quantikz}
\end{center}

where the $\ket -$ control performs the same operation as the $Z$ gate described previously (i.e. generates the $(-1)^{f(x)}$).

But how would you figure this out classically? To uniquely find $f$, you'd need to put in a full basis of $x$, which necessitates at least $n$ queries to uniquely determine $u$. On the other hand, in the quantum setting, Bernstein-Vazirani requires only two query accesses (assuming you can invert $f$). This is a substantial speedup.

You can generalize this algorithm with one called \textbf{Recursive Fourier Sampling (RFS)}. This is a similar algorithm that works on
multi-dimensional vectors, allowing you to compute gradients (see that here we computed the gradient of $f$). For the problem that RFS solves, the comparison of runtimes between classical algorithms and quantum algorithms is shown below:
\begin{itemize}
    \item Classical Algorithms satisfy the recursion $T(n) > nT\left(\frac{n}{2}\right) + n$. Solving this recurrence yields a runtime of 
    \[T(n) = \Omega(n^{\log n}),\]
    which is \emph{superpolynomial}.

    \item Our Quantum Algorithm (i.e. RFS) satisfies the recursion $T(n) = 2T\left(\frac{n}{2}\right) + O(n)$. Solving this recurrence yields a runtime of 
    \[T(n) = O(n \log n),\]
    which is \emph{polynomial}.
\end{itemize}
Here, we can see the power of quantum algorithms for computational optimization.

