\section{Lecture 13}

\subsection{Shannon Entropy}
Suppose we have some (discrete) classical probability distribution $\{p_i\}_{i = 1}^n$. We'd like to somehow
characterize the amount of uncertainty in this distribution. One natural notion would be to count the number of outcomes which
have probability that is nonzero. However, this notion does not acknowledge a difference between all the $p_i$'s being equal
and the case where $p_1 > 0.99$ and the rest are tiny. The former definitely has more uncertainty than the latter.
The better notion is \textbf{Shannon Entropy}, which we will motivate before introducing.

One should take $N$ independent samples and take the limit as $N \to \infty$. By the (weak) law of large numbers
with probability tends to 1, the fraction of outcome $i$, $N_i/N \to p_i$. We define a ``typical outcome'' to be a sample where
this fraction is true (up to some $\epsilon$ wiggle room). The number of typical outcomes, by combinatorics:
\begin{align*}
    N_{typical} = \sum_{\text{typical } \{N_i\}}\frac{N!}{\prod_{i} N_i!}
\end{align*}
The number of typical $\{N_i\}$ grows at most polynomially with $N$, because for a typical $N_i$,
$N_i/N \in p_i \pm \epsilon$, so $N_i \in Np_i \pm N \epsilon$, so there are roughly $(2N \epsilon)^n$ choices,
which is still polynomial ($n$ is the number of possible outcomes, remember). However, the probability of the factorials is
going to turn out to be exponential, so we will ignore the summation.
\begin{align*}
    \log N_{typ} &\approx \log N! - \sum_i \log N_i! \\
    &\approx N \log N - \sum_i N_i \log N_i \\
    &= \sum_i N_i \log N - \sum_i N_i \log N_i \\
    &= - \sum_i N_i \frac{N_i}{N} \\
    &= - N \sum_i p_i \log p_i
\end{align*}
This motivates the following definition.

\begin{definition}
    The Shannon Entropy of a probability distribution $\mathbf{p} = \{p_i\}_{i = 1}^N$ is
    \[ H(\mathbf{p}) = -\sum_i p_i \log p_i \] 
\end{definition}

Intuitively, the Shannon entropy denotes how many bits are required to communicate a typical sample from the distribution.
From another direction, we can think of the probability of getting a typical outcome.
\begin{align*}
    p_{typ} &= \prod_i p_i^{N_i} \\
    \log p_{typ} &= \sum_i N_i \log p_i \\
    &\approx -N H(\mathbf{p})
\end{align*}
But then, we take
\begin{align*}
    \log p_{typ} N_{typ} &= \log p_{typ} + \log N_{typ} \\
    &\approx -N H(\mathbf{p}) + N H(\mathbf{p}) = 0 \\
    p_{typ} N_{typ} &\approx 1
\end{align*}
So as a sanity check, this makes sense because the sum of the probabilities of all of the typical outcomes should be 1. This gives us the sketch of the Shannon source-coding theorem.

\begin{theorem}[Source-coding]
   It requires $N H(\mathbf{p}) + o(N)$ bits to communicate a typical outcome of $N$ samples from $\{p_i\}$. 
\end{theorem}

Thus, one can say that it takes amortized $H(\mathbf{p})$ bits of information per sample to communicate a typical string.

\subsection{Entanglement Entropy}
Let's extend our anaylsis to quantum. Take $\ket{\Psi} \in \mathcal{H}_A \otimes \mathcal{H}_B$ and suppose Alice and Bob share a state
$\ket{\Psi}^{\otimes N}$ and Alice wants to send her part of $\ket{\Psi}^{\otimes N}$ to Charlie
so that with high probability a state shared by Charlie and Bob becomes $\tilde{\ket{\Psi}}_{(N)} \approx \ket{\Psi}^{\otimes N}$.

If the initial state is a product state $\ket{\Psi} = \ket{\psi}_A \ket{\phi}_B$,
then 0 qubits are needed. Charlie can prepare $\ket{\psi}^{\otimes N}$ in his lab and Bob can contribute his state.

Recall that we derived
\[ \ket{\Psi} = \sum_i \sqrt{\lambda_i} \ket{\psi_i}_A \ket{\phi_i}_B, \rho_A = \sum_i \lambda_i \ketbra{\psi_i}, \rho_B = \sum_i \lambda_i \ketbra{\phi_i} \]
We define the typical subspace $\mathcal{H}_{typ} \subseteq \mathcal{H}_A$ as
\[ \mathcal{H}_{typ} = \textrm{span}\left\{ \ket{\psi_{i_1}} \ket{\psi_{i_2}} \dots  \ket{\psi_{i_N}} : \frac{N_i}{N} \approx \lambda_i \right\} \]
We define $P_{typ}$ as the projector onto $\mathcal{H}_{typ}$. Applying the projector,
\[ \bra{\Psi}^{\otimes N} P_{typ} \ket{\Psi}^{\otimes N} = \sum_{\text{typical } \{i_j\} }  \prod_{j = 1}^N \lambda_{i_j} = \sum_{ i} p_i \approx 1 \]
The dimension of the typical subspace can be recovered by noting that it's just the number of typical strings, which we already found was:
\[ \log \dim(\mathcal{H}_{typ}) \approx N H(\mathbf{\lambda}) \]
Now we can define the protocol. Alice measures whether $\ket{\Psi}^{\otimes N}$ are in $\mathcal{H}_{typ}$ The new state becomes:
\[ \ket{\tilde{\Psi}_{(N)}} = \frac{P_{typ} \ket{\Psi}^{\otimes N}}{\sqrt{\bra{\Psi}^{\otimes N} P_{typ} \ket{\Psi}^{\otimes N} }} \approx \ket{\Psi}^{\otimes N}\]
We can now send this to Charlie using $N H(\mathbf{\lambda})$ qubits and we're done describing the state.
This motivates the following quantum entropy definition (taking operator $\log$ by its infinite series):
\begin{definition}
    The Von Neumann Entropy of a density matrix $\rho$ is:
    \[ S(\rho) = -Tr(\rho \log \rho) = - \sum_i \lambda \log \lambda_i\]
\end{definition}
If we have a pure state $\ket{\Psi} \in \mathcal{H}_A \otimes \mathcal{H}_B$, then $S(\rho_A) = S(\rho_B) = H(\mathbf{\lambda})$, and we term this
the entanglement entropy.

\subsection{Quantum Channels}
We've already talked about noisy states and noisy measurement. How can we define a noisy quantum evolution? It turns out it can be described by a few things. Start with
some state $\rho_A \in \mathcal{H}$.
\begin{enumerate}
    \item Take a fixed state $\ket{0} \in \mathcal{H}_A$ and give new state $\rho_A \otimes \ket{0}\bra{0}_X$
    \item Take an arbitrary unitary evolution $U_{AX}$ and give new state $U_{AX}\qty(\rho_A \otimes \ketbra{0}) U_{AX}^{\dagger}$.
    \item Write the entire Hilbert space as $\H \simeq \H_A \otimes \H_X \simeq \H_B \otimes \H_E$. Then throw away $\H_E$, yielding
    \[ \rho_B = \tr_E\qty[U_{AX} (\rho_A \otimes \ketbra{0}_X) U_{AX}^{\dagger}] = \mathcal{N}(\rho_A) \]
    We call $\mathcal{N}$, which takes an operator and spits out an operator, a \textbf{superoperator}.
\end{enumerate}
We can also the first two steps as $V = U_{AX} \ket{0}_X$, where $V: \H_A \to \H$ and is almost unitary, namely:
\[ V^{\dagger} V = I_A, VV^{\dagger} \neq I_B \otimes I_E \]
So, the map $V$ preserves inner product and is termed an $\textbf{isometry}$. Our channel thus becomes:
\[ \mathcal{N}(\rho) = \tr_E(V \rho V^{\dagger}) \]
In particular, this is called the \textbf{Stinespring representation} of the channel. We can also write with Kraus operators:
\[ V \ket{\Psi}  = \sum_i K_i \ket{\Psi} \ket{i}_E \]
where $V^{dagger}V = I$ implies $\sum_i K_i^{\dagger} K_i = I$. The channel then becomes
\[ \mathcal{N}(\rho) = \tr_E\qty(\sum_{i, j} K_i \rho K_j^{\dagger} \otimes \ket{i}\bra{j}) = \sum_i K_i \rho K_i^{\dagger} \]
This is called the \textbf{Kraus representation} of the channel.

Let's consider a few examples of channels.
\begin{example}
    \begin{enumerate}
        \item The Dephasing Channel.
        \[ K_0 = \sqrt{1 - p/2} I, K_1 = \sqrt{p/2} Z \]
        Then the channel is \[\mathcal{N}(\rho) = \qty(1 - \frac{p}{2}) \rho + \frac{p}{2} Z \rho Z^{\dagger}\]
        This means that with probaiblity $p/2$, the phase gets flipped, else we do not.
        \[ \rho = \begin{pmatrix}
            a & b \\ c & d
        \end{pmatrix} \implies \mathcal{N}(\rho) = \begin{pmatrix}
            a & (1 - p) b \\ (1 - p) c & d
        \end{pmatrix} \]
        If $p = 1$, then 
        \[ \mathcal{N}(\rho) = \begin{pmatrix}
            a & 0 \\ 0 & d
        \end{pmatrix} \]
        Then, we get a mixture of $\ket{0}$ and $\ket{1}$; all phase information is lost.
        \item The Depolarizing Channel.
        \[ K_0 = \sqrt{1 - \frac{3p}{4}} I, K_1 = \sqrt{\frac{p}{4}} X, K_2 = \sqrt{\frac{p}{4}} Y, K_3 = \sqrt{\frac{p}{4}} Z \]
        Then the channel is 
        \[ \mathcal{N}(\rho) = \qty(1 - \frac{3p}{4}) \rho + \frac{p}{4} X \rho X + \frac{p}{4} Y \rho Y + \frac{p}{4} Z \rho Z\]
        which can be rewritten as
        \[ \mathcal{N}(\rho) = (1 - p) \rho + \frac{p}{2} I  \]
        which means as $p \to 1$, the state is less like $\rho$ and becomes closer to maximally mixed.
    \end{enumerate}
\end{example}

It turns out quantum channels are really pretty general objects. Here are some properties they satisfy:
\begin{enumerate}
    \item A quantum channel is linear.
    \[ \mathcal{N}\qty(\alpha \rho + \alpha \sigma) = \alpha \mathcal{N}(\rho) + \beta \mathcal{N}(\sigma) \]
    \item A quantum channel is trace-preserving.
    \[ \tr \mathcal{N}(\rho) = \tr \rho \]
    \item A quantum channel is positive. If $\rho \geq 0$, then
    \[ \mathcal{N}(\rho) = \sum_i K_i \rho K_i^{\dagger} \geq 0 \]
\end{enumerate}
2 and 3 combined say that $\mathcal{N}$ maps density matrices to density matrices. Is any superoperator satifying
these properties a quantum channel? The answer is no. The counterexample is the map that takes $\rho \mapsto \rho^T$.
Consider a state $\ket{\Phi^+} \in \H_A \otimes \H_B$. The density matrix of this state is:
\[ \rho_{AB} = \ketbra{\Phi^+} = \frac{1}{2} \qty(\ketbra{00} + \ketbra{00}{11} + \ketbra{11}{00} + \ketbra{11}) \]
We will apply the transpose to the first qubit.
\[ \rho_{AB}^{T_A} = \frac{1}{2} \qty(\ketbra{00} + \ketbra{10}{01} + \ketbra{01}{10} + \ketbra{11}) \]
But this is not a density matrix at all. The state $\ket{01} - \ket{10}$ has eigenvalue $-1/2$. So we need a fourth condition.

4. Completely positive. 
\[ \rho_{AR} \geq 0 \implies \mathcal{N} \otimes \textrm{Id}_R(\rho_{AB}) \geq 0\]
Note that 4 implies 3. In fact, we will see that next lecture, any completely positive, trace-preserving,
linear superoperator is a quantum channel.