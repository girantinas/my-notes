\section{Lecture 12}

\subsection{Measuring Devices}
What is so mysterious about quantum measurements? Well first off, it's weird that the measurement and unitary transformations
form two distinct quantum postulates, when they both describe physical processes. Well, we can actually try to describe measurement using unitaries.
In fact, any measurement device is itself a quantum system $\mathcal{H}_M$, and a measurement is just an interaction between this device
and our measured system $\mathcal{H}_A$. There should be some unitary $U_{meas}$ that describes the measurement.
What does this unitary look like? Well, we can see how this acts on basis elements and then on $\ket{\psi} = \alpha \ket{0} + \beta \ket{1}$,
denoting $\ket{??}$ as the state of the device before it's read anything.
         %dial sitting in the middle
\begin{align*}
    U_{meas} \ket{0} \ket{??} &= \ket{0} \ket{\text{meas. } 0} \\
    U_{meas} \ket{1} \ket{??} &= \ket{1} \ket{\text{meas. } 1} \\
    U_{meas} \ket{\psi} \ket{??} &= \alpha \ket{0} \ket{\text{meas. } 0} + \beta  \ket{1} \ket{\text{meas. } 1}
\end{align*}
Thus, the measurement device has gotten entangled with the state it's measuring. But,
we have a formalism with partial trace for figuring out what state the device is in:
\begin{align*}
    \rho_M &= \Tr_A \qty[U_{meas} \ket{\psi} \ketbra{??} \bra{\psi} U_{meas}^{\dagger}] \\
    &=\qty|\alpha|^2 \ketbra{\text{meas. } 0} + \qty|\beta|^2 \ketbra{\text{meas. } 1}
\end{align*}
But note, this is the same as an ensemble with $\Pr{\text{meas. } 0} = |\alpha|^2$ and $\Pr{\text{meas. } 1} = |\beta|^2$. However,
we used the measurement postulate last lecture to get the connection between density matrices and quantum systems.
Note that what we've done here is not a derivation of the measurement postulate, but it is a nice way of looking at what quantum measurement is. Similarly,
we find for $A$
\[ \rho_A = \qty|\alpha|^2 \ketbra{0} + \qty|\beta|^2 \ketbra{1} \implies \Pr{\ket{0}} = |\alpha|^2, \Pr{\ket{1}} = |\beta|^2 \]
This means that any future measurements on $\mathcal{H}_A$ that ignore $\mathcal{H}_M$ will agree with this ensemble. Thus, the idea
that the wavefunction ``collapses'' can be explained as unitary evolution with a measurement device.

But what if you don't throw away the measurement device?
\begin{align*}
    U_{meas} \ket{+} \ket{??} &= \ftwo \ket{0} \ket{\text{meas. } 0} + \ftwo \ket{1} \ket{\text{meas. } 1} \\
    U_{meas}^{\dagger} U_{meas} \ket{+} \ket{??} &= \ket{+} \ket{??}
\end{align*}
So, measurement IS reversible! The global state is not an ensemble, but actually a superposition, and thus reversible.

This may seem like it breaks the uncertainty principle,
but note that reversing the measurement requires erasing all trace of the outcome of the measurement from the rest of the universe
(all entanglements with other systems). So we can't look at it (or if we do, we have to erase our memory).

\subsection{Decoherence}
In a real setup, very quickly a measurement gets copied (really, entangled) into large quantum systems. For example, if your measurement device has an LED screen,
large number of photons have a different state depending on if the measurement device shows a 0 or 1.
So, one can think roughly as $\ket{\text{meas. } 0} = \ket{0}\ket{0}\dots \ket{0} = \ket{0^N}$. In practice $N \in [10^{20}, 10^{30}]$. But, if some ``qubit'' photon $\gamma$ leaves our lab,
we need to trace over it to figure out our state. This yields:
\[ \rho_{S\setminus \{ \gamma\}} = \qty|\alpha|^2 \ketbra{0^{N - 1}} + \qty|\beta|^2 \ketbra{1^{N - 1}} \]
This is a mixed state. We cannot use $U_{meas}^{\dagger}$ to undo things.
This phenomenon is known as \textbf{decoherence}.

\subsection{Us: Interpretations of Quantum Mechanics}
In our class, we distinguish between a \emph{measurement} and an \emph{observation}. A measurement is what we just described: a unitary interaction between a system and a measuring device.
An observation is a person actually looking at a measurement outcome. Let's try to discuss this as a unitary transform:

\[ U_{obs} \qty(\alpha \ket{0} \ket{\text{meas. } 0} + \beta\ket{1} \ket{\text{meas. } 1}) \ket{hmm} =  \alpha \ket{0} \ket{\text{meas. } 0} \ket{\text{sad}} + \beta\ket{1} \ket{\text{meas. } 1} \ket{\text{happy}}  \]
It's not obvious what it means for my brain to be in a superposition state. Thus, we still need a measurement postulate that superposition is experienced as a probabilistic process weighted by norm squared.

This relates to the interpretations of quantum mechanics. There are two main ones:
\begin{enumerate}
    \item $\ket{\Psi}$ is \emph{epistemic}. This means it's always fundamentally a quantum version of Bayesian probabilities, meaning that when I see a measurement, I have more information and thus only see one of the terms in my $\ket{\Psi}$.
    This one is closest to ``Copenhagen Interpretation" from 100 years.
    \item $\ket{\Psi}$ is \emph{ontic}. $\ket{\Psi}$ is physically real and is everything. This is often considered the many-world interpretation. The probabilities observed in the lab are not accounted for in this model.
\end{enumerate}

\subsection{POVM Measurements}
If measurements are really interactions with a device, why restrict $U_{meas}$ to only give us $\ket{\text{meas. } 0}$ or $\ket{\text{meas. } 1}$?
Let's let it be any unitary. Let's initialize the measurement device in the state $\ket{0}$.
\[ U_{meas} \ket{\psi} \ket{0} = \sum_{m} \ket{\psi_m} \ket{m} \]
where the $\ket{\psi_m}$ are not normalized. We can rewrite that since this is a linear map, its action on $\psi$ can be seen to be linear:
\[ \ket{\psi_m} = K_m \ket{\psi} \]
By unitarity,
\begin{align*}
    \bra{0} \bra{\phi} U_{meas}^{\dagger} U_{meas} \ket{\psi} \ket{0} &= \sum_{m_1, m_2} \bra{m_1} \bra{\phi} K_{m_1}^{\dagger} K_{m_2} \ket{\psi} \ket{m_2} \\
    &= \sum_{m_1, m_2} \bra{\phi} K_{m_1}^{\dagger} K_{m_2} \ket{\psi} \braket{m_1}{m_2} \\
    &= \sum_m \bra{\phi} K_m^{\dagger} K_m \ket{\psi}  \\
    \braket{\phi}{\psi} &= \bra{\phi} \qty(\sum_m K_m^{\dagger} K_m) \ket{\psi}
\end{align*}
For this to be true for any $\ket{\phi}$ and $\ket{\psi}$, we must have
\[ \sum_m K_m^{\dagger} K_m = I \]
The $K_m$'s are usually known as \textbf{Kraus operators}. Now to get a measurement, we just find the probability
according to the measurement postulate
\begin{align*}
    \Pr{m} &=  \qty|\bra{m} U_{meas} \ket{\psi} \ket{0}|^2 \\
    &= \sum_{m_1, m_2} \bra{m_1} \bra{\psi} K_{m_1}^{\dagger} \ket{m} \bra{m} K_{m_2} \ket{\psi} \ket{m_2}\\
    &= \sum_{m_1, m_2} \bra{\psi} K_{m_1}^{\dagger}  \braket{m_1}{m} \braket{m}{m_2} K_{m_2} \ket{\psi} \\
    &= \mel**{\psi}{K_m^{\dagger} K_m}{\psi} = \langle K_m^{\dagger} K_m \rangle
\end{align*}
So note, that $\Pi_m := K_m^{\dagger} K_m$ is all the matters. Note that an operator is in the form $A^{\dagger} A$ if and only if it is positive semi-definite.
Thus we have:
\[ \sum_m \Pi_m = I, \Pi_m \geq 0\]

Thus, there is no reason that these $\Pi_m$'s need to be projection operators at all. They just need to be PSD. If they were projection operators,
we'd get the \textbf{projection-valued measure (PVM)} from last time. Instead these are the \textbf{positive-operator-valued measure (POVM)}. It turns out,
this lets you learn about weak or noisy measurements. Consider the two operators
\[ \Pi_0 = \frac{3}{4} \ketbra{0} + \frac{1}{4} \ketbra{1}, \Pi_1 = \frac{1}{4} \ketbra{0} + \frac{3}{4} \ketbra{1} \]
These clearly add up to the identity. Then for the original state $\ket{0}$, the device enters a state where:
\[ \Pr{0} = \mel{0}{\Pi_0}{0} = 3/4, \Pr{1} = 1/4 \]
and if we instead started with a $\ket{1}$,
\[ \Pr{0} = \mel{0}{\Pi_1}{0} = 1/4, \Pr{1} = 3/4 \]
This is a noisy measurement because $3/4$ of the time, we get the correct answer and $1/4$ of the time, we get something completely wrong.

Remember, $\ket{\psi} \to K_m \ket{\psi}$ when $K_m^{\dagger} K_m = \Pi_m$; so a particularly nice Kraus operator
is $\Pi_m^{1/2}$. Taking our previous example, we have:
\begin{align*}
    &K_0 = \frac{\sqrt{3}}{2} \ketbra{0}{0} + \frac{1}{2} \ketbra{1}{1}, &K_1 = \frac{1}{2} \ketbra{0}{0} + \frac{\sqrt{3}}{2} \ketbra{1}{1}  \\
    &K_0 \ket{+} = \frac{\sqrt{3}}{2} \ket{0} + \frac{1}{2} \ket{1}, &K_1 \ket{+} = \ftwo \ket{0} + \frac{\sqrt{3}}{2} \ket{1}
\end{align*}